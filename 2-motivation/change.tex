
\section[Richtungswechsel: Von defensiv zu offensiv]
  {Richtungswechsel:\\Von defensiv zu offensiv}
\label{sec:motivation:change}

\authors{\DE \and \JF}{\LM \and \KUH \and \MW \and \BK}

Nach dem ersten Projektjahr führten wir eine Bestandsaufnahme
durch. Wir hatten für die Seite Rechnernetze viele Bausteine
zusammengetragen, die notwendig waren, um ein Netz zu überwachen und
die bei der Überwachung entstehenden Daten anonymisiert
weiterzuverarbeiten.

So konnten wir, und können wir noch heute, Daten, die aus
verschiedenen Sensoren im \acr{idmef} zusammengetragen werden,
anonymisieren und zur Analyse an eine Komponente senden, die diese
Daten in Zusammenhang bringen kann.

Wir hatten bereits einige möglicherweise zur Analyse geeignete
KI-Algorithmen getestet, jedoch gab es hierbei zwei Probleme, die
besonders deutlich wurden. Da unsere Projektgruppe etwa zur Hälfte aus
Studenten bestand, die den Bachelorabschluss anstrebten und damit nur
für den Zeitraum eines Projektjahres dem Projekt zur Verfügung
standen, war die Anzahl der Studenten nach dem ersten Jahr um die 50
Prozent reduziert. Zudem verließen uns vor allem die Studenten, die
sich mit den Analyse-Algorithmen beschäftigt hatten.

Hinzu kam, dass wir mehr und mehr vor dem Problem standen, dass sowohl
Angriffe auf Rechnernetze komplex zu simulieren sind, als auch, dass
wir nicht automatisiert testen konnten. Dies ist essentiell, um
Algorithmen zu testen, die Netzverkehr analysieren sollen.

Was wir gebraucht hätten, wäre ein Tool gewesen, welches es uns
ermöglicht einen Algorithmus zur Analyse zu installieren, um
abschliessend diverse Angriffe ablaufen zu lassen.

Hierbei gab es wiederum zwei besonders hervortretende Probleme. Zum
einen wäre es unrealistisch gewesen, Angriffe in einem Netz zu senden,
in dem sonst kein Netzverkehr entsteht. Denn Angriffe sind häufig vor
allem auch deshalb schwer zu erkennen, weil sie in der Menge des
sonstigen Datenverkehrs in einem Netz untergehen. Zum anderen mussten
diese Angriffe reproduzierbar und protokollierbar sein, um
verschiedene Analyse-Methoden später vergleichen zu können.

Wir waren also auf der Suche nach einem Werkzeug, das verschiedene
Angriffe automatisiert durchführt und diese protokolliert. Die
Angriffe sollten reproduzierbar sein und der Grad der Auffälligkeit
musste einstellbar sein. Darüber hinaus musste ein solches Tool in der
Lage sein, mit weiterem Netzverkehr umzugehen. Es musste möglich sein,
die Pakete, die zum Angriff gehören, zu identifizieren, sodass man
hinterher feststellen konnte, ob der Analyse-Algorithmus die richtigen
Pakete, als zu einem Angriff gehörend, erkannt hat.

Nachdem wir diese vier großen Probleme identifiziert hatten, begaben
wir uns auf die Suche nach einem Tool, das bereits konnte, was wir
brauchten. Jedoch konnten wir kein Programm finden, dass das bereits
konnte, was wir haben wollten. Zudem ergaben Gespräche mit Forschern
aus anderen Projekten, wie z.B. dem FIDeS-Projekt der Universität
Bremen, dass dieses Problem, vor dem wir nun standen, sowohl ein
bekanntes Problem, als auch ein immer noch bestehendes Problem ist. In
mehreren Projekten im Bereich der IT-Security gibt es deutliche
Probleme, an Echt-Daten zu gelangen, da diese in der Regel aus
datenschutzrechtlicher Sicht sensible Daten enthalten.

Darüber hinaus ergaben Gespräche mit anderen Projekten der Universität
Bremen, dass ein solches Tool wie oben beschrieben tatsächlich einige
Probleme lösen und von wissenschaftlicher Relevanz sein könnte.

Diese Aussichten motivierten uns sehr und wir beschlossen die Fronten
zu wechseln. Nach einem Jahr haben wir also noch einmal von vorne
angefangen, um ein Tool zu schreiben, das in der Lage ist
automatisiert, reproduzierbar und protokollierend ein Netz
anzugreifen.
