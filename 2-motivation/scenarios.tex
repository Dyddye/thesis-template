
\section{Betrachtete Szenarien}
\label{sec:motivation:scenario}

\authors{\DE \and \JF}{\WF \and \LM \and \MW \and \DH}

Um den Zweck des \f-Systems und die damit zu erreichenden Ziele
festlegen zu können, bedarf es einer genauen Darstellung der
Anwendungsszenarien.

Das \f-System soll in erster Linie Sicherheitsanalysten beim
systematischen Testen von \glspl{glos:ids}n unterstützen. Dabei ist
der Aufbau der Netze, die durch ein \acr{ids} überwacht werden, von
Organisation zu Organisation sehr unterschiedlich. Diesem Umstand muss
deshalb bei der Entwicklung des \f-Systems Rechnung getragen
werden. Diese Heterogenität bezieht sich allerdings nicht allein auf
die Rechnernetze. Auch die verwendeten \acr{ids} sowie deren
Regelsätze sind unterschiedlich. Der Markt bietet hier sowohl freie
Systeme (\zB \gls{glos:prelude-ids}) als auch kommerzielle Lösungen an.

Natürlich sollen \acr{ids} nicht nur getestet werden können,
die Ergebnisse der Tests müssen auch vergleich- und bewertbar
sein. Aufgrund des \f-Systems soll ein Sicherheitsanalyst die
Möglichkeit haben zu entscheiden, welche Konfiguration bzw. Regelmenge
für den jeweiligen Anwendungsfall optimal ist. Dazu ist es nötig, dass
das \f-System in der Lage ist, zwei grundlegende Aufgaben zu
erfüllen. Dem Anwender soll es mit einer leicht zu bedienenden
grafischen Oberfläche bei den Testangriffen auf das eigene Netz
assistieren, während eine weitere Komponente im Hintergrund die durch
den Angriff erzeugten \glospl{idmef-event} registriert und
aufbereitet.

Als zusätzliche Funktionalität kann sich das \f-System die vom
Anwender ausgeführten Angriffe merken, damit diese auch auf andere
Netze durch eine KI-Komponente semi- und vollautomatisch angewendet
werden können. Diese Angriffsautomatisierung darf jedoch nur soweit
gehen, dass die KI Vorschläge über eventuelle nächste Angriffsziele
innerhalb des Netzes macht.

Um die Wiederverwendbarkeit der mit dem \f-System durchgeführten
\acr{ids}-Tests zu gewährleisten, muss es einem Anwender möglich sein,
einzelne Angriffschritte in einem austauschbaren Format zu definieren,
damit andere sie wiederverwenden können. Eine dazu nötige
Planungskomponente kann dafür genutzt werden, Sicherheitstests für
Rechnernetze ähnlich der Unit-Tests aus der Softwaretechnik zu
entwickeln, die nach jeder Veränderung an der Netzinfrastruktur erneut
ausgeführt werden.

Ein letzter wichtiger Aspekt bei der Betrachtung der möglichen
Einsatzszenarien ist die Integration bereits bestehender Komponenten
zum Penetration-Testing innerhalb einer Organisation. Das
\f-System soll es ermöglichen, bisher verwendete Methoden einfach
an das System anzubinden. Daher spielt auch die Erweiterbarkeit des
\f-Systems eine wichtige Rolle. Hierzu haben wir eine Architektur
entwickelt, die das Ergänzen von Methoden erlaubt und das \glos{msf}
verwendet, welches weitere Schnittstellen zu diversen bereits
vorhandenen Werkzeugen mitbringt. 