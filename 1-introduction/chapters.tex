
\section{Kapitelübersicht}
\label{sec:intro:chapters}

\authors{\JF \and \DH}{\LM \and \MW \and \BK}

Im Folgenden soll ein Überblick über den Aufbau dieses Dokumentes
gegeben und die behandelten Themen kurz vorgestellt werden.

\textbf{Kapitel~\ref{chp:motivation}} beschäftigt sich mit dem
Richtungswechsel, der nach dem ersten Projektjahr vollzogen
wurde. Es wird beschrieben, warum das \f-System sinnvoll ist und
welche Anwendungsszenarien damit abgedeckt werden.

Im~\textbf{\ref{chp:related-work}. Kapitel} werden Verweise auf ähnliche
Arbeiten in der Anwendungsdomäne des \f-Systems gegeben.

Anschließend wird in \textbf{Kapitel~\ref{chp:objectives}} genauer definiert,
welche Ziele gemäß der betrachteten Szenarien mit dem \f-System
erreicht werden sollen.

\textbf{Kapitel~\ref{chp:methods}} stellt die Methoden
vor, mit denen einzelne Komponenten entwickelt wurden und beschreibt
auch die Vorgehensweise bei der Veröffentlichung von
Projektergebnissen.

Daran anschließend wird die Art der Umsetzung präsentiert
(\textbf{Kapitel~\ref{chp:execution}}). Hier wird als erstes ein Überblick über
die Gesamtarchitektur gegeben. Danach folgt die Vorstellung der
Benutzeroberfläche, dem \textit{\f Command and Control Server} in
Abschnitt~\ref{sec:candc}. Das Herzstück des \f-Systems bildet der
\textit{\nameref{sec:core}}, dessen Komponenten in
Abschnitt~\ref{sec:core} beschrieben werden. In
Abschnitt~\ref{sec:cve-db} wird thematisiert, wie sich das \f-System
bei einer Datenbank mit bekannten Schwachstellen bedient, in der
sogenannte \glsdisp{acr:cve}{CVE}-Einträge zu Schwachstellen abgefragt werden
können. Anschließend wird die Verwendung des \acr{msf}, sowie die
während des Projekts entstandenen Erweiterungen dafür beschrieben. In
Abschnitt~\ref{sec:evasion-db} wird die \textit{EvasionDB}-Komponente
beschrieben. Sie liest die durch das verwendete \acr{ids} erzeugten
\glsdisp{glos:idmef-event}{IDMEF-Events} aus der Datenbank des \acr{ids} aus und
stellt diese für das \f-System dar. Als weitere Aufgabe wird die
Konfiguration des \glsdisp{glos:ids}{IDSs} mit Hilfe der \textit{EvasionDB} in
Abschnitt~\ref{snortor} thematisiert. Zuletzt enthält das Kapitel eine
Beschreibung der verwendeten Methoden der \acr{ki} und in welchen
Szenarien diese das \f-System unterstützen (Abschnitt~\ref{sec:ki}). 

Im~\textbf{\ref{chp:conclusion}. Kapitel} werden Ergebnisse präsentiert, die
beim Testen der \glspl{glos:ids} erzielt wurden.

In \textbf{Kapitel~\ref{chp:test-infrastructure}} wird die Testumgebung erläutert,
die während der Entwicklung zum Einsatz kam. Dabei werden die Ziele
der Tests und die unterstützenden Software-Tools und
Netzinfrastrukturen präsentiert.

Im Anschluss folgt in \textbf{Kapitel~\ref{chp:results}} ein Fazit, das noch
einmal zusammenfassend auf die Ergebnisse des Projekts und die
Projektarbeit innerhalb von \f eingeht.

Das letzte Kapitel gibt einen Ausblick über noch ausstehende
Veröffentlichungen und die Entwicklungen, die im Anschluss an das
Projekt stattfinden können. (\textbf{Kapitel~\ref{chp:outlook}})

Im Anhang \textit{Referatsausarbeitungen}
(\textbf{Anhang~\ref{chp:compositions}}) sind die Ausarbeitungen der Beiträge
der einzelnen Projektmitglieder zum zweiten Projektwochenende enthalten.

Abschließend folgen mit dem Anhang \textit{Verzeichnisse}
(\textbf{Anhang~\ref{chp:lists}}) die Tabellen- und Abbildungsverzeichnisse,
das Glossar, die Akronyme und die Literaturliste.
