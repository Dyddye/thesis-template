\section{Rückblick 1. Projektjahr}
\label{sec:intro:retro}

\authors{\JF \and \DH}{\LM \and \MW \and \BK}

In diesem Abschnitt soll kurz dargestellt werden, welche Ergebnisse im
ersten Jahr des Projekts \f erzielt wurden. Für eine detaillierte
Darstellung der Arbeiten und Ergebnisse empfiehlt sich ein Blick in
den entstandenen Projektzwischenbericht~\cite{fidius}.

Im ersten Jahr war das vorrangige Ziel, die Anzahl der
\gls{glos:false-positive}-Meldungen, die ein \gls{glos:ids} einem
Administrator eines Rechnernetzes liefert, zu reduzieren. Dazu stellte
sich zunächst die Frage, wie eine Unterscheidung zwischen echten
Angriffen und harmlosen Meldungen (\glspl{glos:false-positive})
geschehen kann. Ein Versuch bestand darin, Methoden der \acr{ki}
einzusetzen, um für den Administrator unnötige Meldungen vorab zu
kennzeichnen oder auszusortieren. Um zu bestimmen, welche Meldungen
wichtig sind bzw. zu einem Angriff gehören oder nicht, wurde das
Annotationstool (\textit{AnnoGuru}) entwickelt, mit dem die erzeugten
Meldungen anhand ihrer Wichtigkeit vom Administrator klassifiziert
werden können. Mit der parallel entwickelten Bibliothek von
\acr{ki}-Algorithmen sollte dann untersucht werden, ob sich nach
Anwendung der Algorithmen mit dem Wissen der annotierten Daten die
Anzahl der \glspl{glos:false-positive} verringern lässt.

Es stellte sich allerdings heraus, dass zum einen kein
Projektteilnehmer in der Lage war, aus aufgezeichneten Meldungen
präzise auf einzelne Schritte eines Angriffs zu schließen und zum
anderen fehlten re\-a\-lis\-tische Daten, auf denen die
\acr{ki}-Algorithmen arbeiten konnten. Ein Versuch, die Daten des
FB3-Netzes zu verwenden, scheiterte am Datenschutz, weshalb nach
Möglichkeiten der Generierung von realistischem Netzverkehr gesucht
werden musste.

Dazu wurde eine Testlandschaft bestehend aus virtuellen Maschinen
aufgebaut, um ein Netz mit einigen Rechnern und einem platzierten
\acr{ids} zu simulieren, in dem gezielt Meldungen des \acr{ids}
produziert werden können. Um die Bedingungen des Datenschutzes zu
erfüllen, wurde eine Anonymisierung der IP-Adressen implementiert,
damit der Einsatz des \f-Systems auch außerhalb des simulierten
Netzes in realen Umgebungen möglich ist. In der Testlandschaft konnten
zwar Daten generiert werden, aber es war weiterhin nicht möglich, die
Daten so zu annotieren, dass die \acr{ki}-Algorithmen damit arbeiten
können.  Somit schloss das erste Projektjahr mit dem Problem des
fehlenden annotierten Netzverkehrs ab, gleichwohl waren
einzelne Komponenten wie die \acr{ki}-Bibliothek oder der
\textit{AnnoGuru} entstanden, die weiter verwendet werden können.

Schlussendlich änderte sich etwas Organisatorisches. Nach der
Beendigung des Projektberichts vom ersten Projektjahr verließen acht
Studenten das Projekt, da sie den für den Bachelor-Abschluss
benötigten Pro\-jekt\-an\-teil absolviert hatten. Ein Großteil von
ihnen bearbeitete den \acr{ki}-Bereich. Somit war eine Neuausrichtung
und Anpassung der Projektziele nötig.
