
%!TEX TS-program = xelatex
%!TEX encoding = UTF-8 Unicode
\documentclass{thesis}

\usepackage{lipsum}

\setStyleFile{glossary}
\makeglossaries

% \newglossaryentry{label}{
%   name={name},
%   description={long description}
% }


% \newacronym{label}{kurz}{lang}
% Zugriff via \gls{label} und Co.
% 
% Beispiel:
% 
%   \newacronym{acr:da}{DA}{Diplomarbeit}
% 
% wird im Text
% 
%   Heute schreibe ich meine \gls{arc:da}. Diese \gls{arc:da}.
% 
% zu
% 
%   Heute schreibe ich meine Diplomarbeit (DA). Diese DA.
% 
% Nur bei der ersten Verwendung der Abkürzung wird diese in langer Form
% dargestellt. Weitere Vorkommen nutzen die Kurz-Version.
% 
% Neben \gls{} gibt es auch noch \Gls{} und \GLS{}, welche die Groß- und
% Klein-Schreibung beeinflussen. Beispiel:
% 
%   \newacronym{acr:http}{http}{Hypertext Transfer Protocol}
%   \gls{acr:http} --> http
%   \Gls{acr:http} --> Http
%   \GLS{acr:http} --> HTTP
% 
% Zur Vereinfachung gibt es auch die Befehle \acr{}, \Acr{} und \ACR{}. Diese
% Befehle sind äquivalent:
% 
%   \gls{acr:foo} <=> \acr{foo}
%   \Gls{acr:foo} <=> \Acr{foo}
%   \GLS{acr:foo} <=> \ACR{foo}
%



\bibliography{bibliography}

\makeatletter



% basically: "\maybespaceafter{foo}{bar}" defines "\foo" which prints "bar"
\def\maybespaceafter#1#2{%
  \expandafter\def\csname #1\endcsname{#2\xspace}%
}

\addto\captionsngerman{
  \def\diplomaname{Diplomarbeit}
\def\examinername{1. Prüfer}
\def\supervisorname{2. Prüfer}
\def\contentsname{Inhalt}

\maybespaceafter{zB}{z.\,B.}%
\maybespaceafter{dh}{d.\,h.}%
\maybespaceafter{uU}{u.\,U.}%
\maybespaceafter{uA}{u.\,A.}%
\maybespaceafter{su}{s.\,u.}%
\maybespaceafter{idR}{i.\,d.\,R.}%

\def\monthyear{%
  \ifcase\month\or Januar\or Februar\or März\or%
    April\or Mai\or Juni\or Juli\or August\or%
    September\or Oktober\or November\or%
    Dezember\fi~\the\year\xspace%
}
}
\addto\captionenglisch{
  \def\diplomaname{Diploma Thesis}
\def\abstractname{Abstract}
\def\examinername{Examiner}
\def\supervisorname{Supervisor}
\def\advisorname{Advisor}
\def\matriculationname{Matriculation No.}
\def\contentsname{Contents}
\def\abbreviationsname{List of Abbreviations}
\def\glossaryname{Glossary}

\maybespaceafter{eg}{e.\,g.}%

\def\monthyear{%
  \ifcase\month\or January\or February\or March\or%
    April\or May\or June\or July\or August\or%
    September\or October\or November\or%
    December\fi~\the\year\xspace%
}




}

\newcommand*{\acr}[1]{\gls{acr:#1}}
\newcommand*{\Acr}[1]{\Gls{acr:#1}}
\newcommand*{\ACR}[1]{\GLS{acr:#1}}
\newcommand*{\acrpl}[1]{\glspl{acr:#1}}
\newcommand*{\Acrpl}[1]{\Glspl{acr:#1}}
\newcommand*{\ACRpl}[1]{\GLSpl{acr:#1}}

\newcommand*{\glos}[1]{\gls{glos:#1}}
\newcommand*{\Glos}[1]{\Gls{glos:#1}}
\newcommand*{\GLOS}[1]{\GLS{glos:#1}}
\newcommand*{\glospl}[1]{\glspl{glos:#1}}
\newcommand*{\Glospl}[1]{\Glspl{glos:#1}}
\newcommand*{\GLOSpl}[1]{\GLSpl{glos:#1}}

%% math tools

\def\N{\ensuremath{\mathbb{N}}}
\def\Z{\ensuremath{\mathbb{Z}}}
\def\I{\ensuremath{\mathbb{I}}}
\def\Q{\ensuremath{\mathbb{Q}}}
\def\R{\ensuremath{\mathbb{R}}}
\def\C{\ensuremath{\mathbb{C}}}
\DeclareMathOperator{\sgn}{\text{sgn}}

% acronyms style
\newglossarystyle{thesisacr}{%
  \renewenvironment{theglossary}%
    {\begin{description}[style=sameline]}
    {\end{description}}
  \renewcommand*{\glossaryheader}{}%
  \renewcommand*{\glsgroupheading}[1]{}%
  \renewcommand*{\glossaryentryfield}[5]{%
      \item[\glstarget{##1}{##2}] ##3, S.~##5
  }%
  \renewcommand*{\glsgroupskip}{\indexspace}%
}

% glossary style
\newglossarystyle{thesis}{%
  \renewenvironment{theglossary}%
    {\begin{description}}
    {\end{description}}
  \renewcommand*{\glossaryheader}{}%
  \renewcommand*{\glsgroupheading}[1]{}%
  \renewcommand*{\glossaryentryfield}[5]{%
      \item[\glstarget{##1}{##2}] ##3 \\
      S.~##5
  }%
  \renewcommand*{\glsgroupskip}{\indexspace}%
}

%% some tolerance parameters
\tolerance=400
\hfuzz=0.2pt

%% do not change these lines
\makeatother
\endinput
%% anything below this line is ignored


\author{asdsad}
\title{asdadasd}
\subtitle{Diplomarbeit}

\begin{document}

% Titel
\maketitle

% Impressumsseite
\clearpage
\thispagestyle{empty}\small
\textit{Herausgegeben von}

\textit{Betreut durch}


\vfill

\textit{\textcopyright\ FIDIUS, 2011.}

\url{http://www.fidius.me/}

Dieses Dokument unterliegt dem Versionskontrollsystem des Projektes. Als
Satzprogramm kam \XeTeX\ in Version \the\XeTeXversion\XeTeXrevision\ zum
Einsatz. Die Brotschrift ist Adobe Minion Pro Regular, als Akzidenzschriften
werden die Helvetica Neue Thin und Condensed sowie die DejaVu Sans Mono Book
verwendet.

\normalsize
\frontmatter
\shorttoc

\cleardoublepage
\phantomsection
\addcontentsline{toc}{section}{Inhaltsverzeichnis}
\tableofcontents

%\clearpage
%\listoftodos

\cleardoublepage

%\section*{Danksagung}


%\input{chapter/preface.tex}

\mainmatter



\lipsum

%

\lipsum

%

\lipsum

%\input{chapters/1-introduction/index.tex}
%\input{chapters/2-motivation/index.tex}
%\input{chapters/3-related-work/index.tex}
% ...


%\input{chapters/1-introduction/index.tex}
%\input{chapters/2-motivation/index.tex}
%\input{chapters/3-related-work/index.tex}
% ...


% ...


%

\lipsum

%\input{chapters/1-introduction/index.tex}
%\input{chapters/2-motivation/index.tex}
%\input{chapters/3-related-work/index.tex}
% ...


%\input{chapters/1-introduction/index.tex}
%\input{chapters/2-motivation/index.tex}
%\input{chapters/3-related-work/index.tex}
% ...


% ...


% ...


%

\lipsum

%

\lipsum

%\input{chapters/1-introduction/index.tex}
%\input{chapters/2-motivation/index.tex}
%\input{chapters/3-related-work/index.tex}
% ...


%\input{chapters/1-introduction/index.tex}
%\input{chapters/2-motivation/index.tex}
%\input{chapters/3-related-work/index.tex}
% ...


% ...


%

\lipsum

%\input{chapters/1-introduction/index.tex}
%\input{chapters/2-motivation/index.tex}
%\input{chapters/3-related-work/index.tex}
% ...


%\input{chapters/1-introduction/index.tex}
%\input{chapters/2-motivation/index.tex}
%\input{chapters/3-related-work/index.tex}
% ...


% ...


% ...


% ...

\appendix
\chapter{Appendix}
\listoffigures
\listoftables
\printbibliography
\printglossary[style=thesisacr,type=acronym,title=\abbreviationsname]
\printglossary[style=thesis,type=main,title=\glossaryname]


\end{document}
