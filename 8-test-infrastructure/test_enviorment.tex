\section{Testlandschaft}
\label{sec:test_environment}
\authors{\CA}{\LM \and \MW \and \JF \and \DH}

Das Projekt \f hat auch im zweiten Jahr einen Bedarf an 
einer kontrollierbaren Testumgebung, in der verschiedene 
Hosts zu jeder Zeit verfügbar sind, um Schwachstellen
ausnutzen zu können. Obgleich der neuen Ausrichtung
(Abschnitt~\ref{sec:motivation:change}), konnte die während des ersten
Projektjahres entstandene Testlandschaft, weiter genutzt werden.

Bereits aufgesetzte und fertig konfigurierte virtuelle Rechner 
innerhalb der Testlandschaft, wie das \glos{snort}-\acr{ids}
oder der \acr{vpn}-Rechner finden ebenso ihre
Notwendigkeit, wie die bereits installierten Betriebssysteme,
welche weiterhin als Angriffsziele dienen.

Insgesamt befinden sich folgende Rechner in der Testlandschaft:

\begin{center}
  \begin{longtable}{ll}
    \rowcolor{Beige}
      Host mit Betriebssystem & Einsatzzweck
    \endhead
      \caption[Übersicht Betriebssysteme]{\\\tabelletbcname}
    \endfoot
      \caption{Übersicht Betriebssysteme\label{tab:test_environment:bs}}
    \endlastfoot
    Debian Linux 1                 & Angriffsziel \\ 
    Debian Linux 2                 & Angriffsziel \\
    Debian Linux 3                 & Angriffsziel \\
    Debian Linux 4                 & Angriffsziel \\
    Windows XP SP2                 & Angriffsziel \\
    Ubuntu Linux                   & Angriffsziel \\
    Metasploitable (Ubuntu)        & Angriffsziel \\
    VPN (Debian Linux)             & Administration \\
    Annoguru  (Debian Linux)       & Administration \\
    Prelude-Manager (Debian Linux) & Administration \\
    Snort (Debian Linux)           & Administration \\ 
  \end{longtable}
\end{center}

Im Vergleich zum Vorjahr diese kontrollierbare Umgebung, welche im
ersten Projektjahr durch Unterstützung der Virtualisierungssoftware
\glos{virtualbox} umgesetzt und erzeugt wurde, um zusätzliche
Angriffsziele erweitert. Da der Fokus der Ausnutzung von
Schwachstellen bisher eher auf Windows-basierten Hosts lag, wurde
dieser im zweiten Projektjahr auf unixoide Betriebssysteme ausgedehnt.

Insgesamt sollten möglichst viele und heterogene Betriebsysteme für
das \f-System bereitgestellt werden, so dass ebenso viele und
unterschiedliche Angriffsmuster erzeugt werden können. Vor dem
Hintergrund, dass das zwischengeschaltete \glos{ids} in Verbindung mit
der EvasionDB (\ref{sec:idsevasion:evasiondb}) und dem \glos{rubygem}
Snortor (\ref{snortor}) auf des somit erzeugten Netzverkehrs arbeiten
kann.

In diesem Zusammenhang ist es von Vorteil, eine möglichst breite
Palette an unterschiedlichen Dienstprotokollen (\acr{http}, \acr{ftp},
\acr{smb}, \acr{imap}, \acr{pop3}, \acr{smtp} etc.) zur Verfügung zu
stellen, um so mehrere Angriffsszenarien durchspielen zu
können. Dieses Vorhaben ist immer vor dem Hintergrund der Untersuchung
der erzeugten Events durch das \acr{ids}-\glos{snort} pro
\glos{exploit} oder Ausnutzung einer Schwachstelle, zu
sehen.

Zusätzlich zur bestehenden Testlandschaft, konnte durch die engere
Zusammenarbeit mit dem Projekt \acr{fides} eine weitere Tetslandschaft
hinzugewonnen werden. Das \acr{fides}-Projekt verwaltet eine eigene,
mit der Unterstützung der T-Systems betriebene Testumgebung, welche
auf \glos{qemu} aufgebaut wurde.

Die Abbildung~\ref{sec:test_environment:fig:fides} zeigt dabei die
zwei möglichen Zugangspunkte für das \f-Projekt in Form des
\enquote{Angreifers} über die jeweils anders gerouteten
\acr{vpn}-Verbindungen. Dabei wird zwischen dem Weg über das Netz der
T-Systems und dem des \acr{tzi}, welches von dem \f-Projekt genutzt
wurde, unterschieden. Somit ist je nach Einwahl bestimmbar, ob die bei
einer Ausnutzung einer Schwachstelle erzeugten Datenpakete nur von dem
\acr{ids} des \acr{tzi} oder zusätzlich durch das der
T-Systems erfasst werden. Die rot markierte Kante verdeutlicht dabei den
ausschliesslich vom \f-Projekt genutzten Angriffsweg.

\begin{figure}
	\centering
  \pgfimage[interpolate=true,width=.8\linewidth]{images/sandbox}
	\caption{FIDeS Testlandschaft}
  \label{sec:test_environment:fig:fides}
\end{figure}

Wie auch die eigene Testlandschaft, beinhaltet die mittels \glos{qemu}
realisierte Testlandschaft ein \glos{ids} auf der Basis von
\glos{snort}, sodass Angriffe entsprechend erkannt und protokolliert
werden können. Als sehr positiv hat sich die Verwendung der Software
\glos{ovs} herausgestellt. Die eigene Testlandschaft hatte immer
wieder Probleme damit, Angriffe innerhalb eines Subnetzes zu erkennen,
wenn diese nicht über das \acr{ids} geroutet wurden. Dieses Szenario
entstand genau dann, wenn ein Angriff von einem eigentlichen
Angriffsziel zu einem anderen Angriffsziel durchgeführt wurde, ohne
den Weg über den \glos{snort}-Rechner bei gesetztem \glos{promisc} zu
gehen. Angebunden an den virtuellen Switch waren wiederum Gastsysteme
mit ausnutzbaren Schwachstellen. So wurden die Angriffsziele um diese
Hosts erweitert:

\begin{enumerate}
  \item Windows XP SP 0
  \item Windows XP SP 1
  \item Windows 2003 Server SP 0
  \item Windows 2003 Server SP 1
  \item Ubuntu Metasploitable
\end{enumerate}

Die Hosts wurden mit mehreren Diensten ausgestattet, welche explizite
Verwundbarkeiten aufwiesen. Zudem wurden innerhalb der Testlandschaft
mehrere Subnetze eingerichtet, um so Pivoting-Mechanismen
(Abschnit~\ref{sec:pivoting}) ausüben zu können.

Die Administration der Gastsysteme ist dabei jedoch nicht vergleichbar
mit der durch \glos{virtualbox} betriebenen Hosts. Die komplexeren
Anweisungen (Abschnitt~\ref{sec:test_environment:lst:qemu}) allein
zum Starten und Stoppen von Betriebssystemen innerhalb der
Testlandschaft führten dazu, dass zunächst einfache Scripte
angefertigt wurden, die dies erledigten, bis schliesslich zur
automatischen Kontrolle durch erweiterte Plugins~\ref{lab} (siehe auch
\ref{sec:test_vm_control}).

\begin{lstlisting}[caption={Qemu Anweisung zum Starten eines Images},label=sec:test_environment:lst:qemu]
  /usr/bin/sudo /usr/libexec/qemu-kvm -net
  nic,model=ne2k_pci,macaddr=52:54:00:01:05:04 -net
  socket,mcast=230.0.0.1:1234 -net
  nic,model=ne2k_pci,macaddr=52:54:00:01:15:04 -net
  tap,script=/opt/virr/nic_scripts/ifup.sh,downscript=/opt/virr/nic_scripts/ifdown.sh
  -vnc 127.0.0.1:0 -daemonize /home/machines/imgs/winXP_sp2/winxp2.img
\end{lstlisting}

Für jedes Gastsystem wurde neben dem laufenden Image ein
\glos{overlay} erzeugt, welches das unversehrte, originale Image des
Betriebsystems enthält. Somit konnte auch bei totalem Ausfall eines
Hosts immer ein Wiederherstellungspunkt reaktiviert werden.

Insgesamt diente die sehr robust laufende Testlandschaft des
\acr{fides}-Projekts als ein weiterer Angriffspunkt für das
\f-System. Hervorzuheben sind der Einsatz des virtuellen Switches,
welcher Probleme, wie sie in der \f-Testlandschaft entstanden
waren, gar nicht erst entstehen ließ - So die Angriffe zwischen
Gastsystemen, welche nicht bemerkt wurden, oder die Möglickeit zum
Pivoting~\ref{sec:pivoting}. Leider sollte es sich jedoch als
schwierig herausstellen, auf die erkannten Events und damit erzeugten
Protokollierungen des \glos{snort}-Rechners Zugriff zu erhalten. Da
dieser Zugriff jedoch als essentiell in Hinblick auf die
Auswertungskomponenten des \f-Systems, wie Snortor und EvasionDB,
zu betrachten ist, war die Verwendung der \acr{fides}-Testlandschaft eher
für die Erprobung neuer \glospl{exploit} interessant.

