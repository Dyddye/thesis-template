\section{Mock-Objekte}
\label{sec:test_mock}
\authors{\HM}{\LM \and \MW \and \JF}

Mock-Objekte werden beim Testen in der Softwareentwicklung verwendet, wenn nur
Teile einer Software getestete werden sollen, die Tests durch die Größe des Systems
sehr kompliziert geworden sind oder das Ausführen der Tests sehr zeitaufwändig \zB durch Anfragen übers Netz ist.

Oft gibt es Probleme, wenn eine Netzverbindung oder eine Datenbank zum
Betrieb der Software benötigt wird.
Die Abhängigkeit von Netzanbindungen oder Datenbanken kann \zB reduziert
werden, indem die Klassen die für die Netzzugriff bzw. Datenbankanfragen
zuständig sind, durch Mock-Objekte ersetzt werden. Die Mock-Objekte
liefern immer vorhersagbare Daten auf eine bestimmte Anfrage zurück und
vereinfacht und beschleunigen so das entwickeln und ausführen der Tests.

Außerdem will man oft auch nur einen Teil der Software testen und nicht die
gesamte Software, weil \zB bestimmte Teile noch nicht entwickelt sind oder
sehr instabil sind. In diesen Fällen kann zum Testen anstatt der richtigen
Bibliothek oder Teilkomponenten ein Mock verwendet werden.

Ein andere Anwendungsbereich ist, wenn verwendetet Bibliotheken unvorhersagbare Daten zurück liefern wie
\zB ein Zufallsgenerator. Dieser Zufallsgenerator kann zum
Testen durch einen Mock ersetzt werden, der immer einen konstanten Wert
zurück liefert und damit vorhersagbare Ergebnisse bei den zu testenden Methoden
erzeugt, damit die Rückgabe der zu testenden Methoden auch vorhersagbar ist
und im Tests überprüft werden kann.

Das \f-System verwendet \acr{msf} zum Ausführen von Angriffen.
Damit ein Angriff ausgeführt werden kann muss unter einer bekannten
\acr{ip}-Adresse ein verwundbares System erreichbar sein.
Dann benötigt so ein Angriff auf aktuellen System mindestens
10 Sekunden.
Damit auch komplexere Tests ohne das Vorhandensein eines angreifbaren
Systems schnell ausgeführt werden können haben wir einen Mock von
\acr{msf} erstellt.
Dieser Mock von \acr{msf} umfasst nur die Methoden die von unserer
Software aufgerufen werden und diese geben größtenteils konstante
und vorhersagbare Werte zurück.
Zusätzlich zum Mock von \acr{msf} haben wir auch Teile des \acr{nmap}-Scans
gemockt, damit dieser auch getestet werden kann.
Damit nur ein Programm gestartet werden muss wird die \glos{drb}-Verbindung
auch durch ein Mock ersetzt.

Alle Funktionstests des \f-Cores benutzen diesen Mock.
Wir haben unter anderem einen Test der einen kompletten Angriff unter Verwendung des Mocks
durchführt.
Bei diesem Test wird zuerst ein Scan durchgeführt.
Mit diesem Test werden die meisten Teile des Codes zum Scannen von Hosts
getestete, da nur der Aufruf von \acr{nmap} durch den Mock ersetzt wurde.
Nach dem Scan wird der Angriff auf einen Host durchgeführt.
Hierbei werden große Teile des Angriffscodes der Action Komponente getestet,
unter anderem kann die richtige Auswahl des Exploits für dieses Ziel und das
eigentliche ausführen des Exploits getestete werden.
Dann wird überprüft, ob die Session die nach einem erfolgreichen Angriff
normalerweise von \acr{msf} erstellt wird auch richtig vom \f-System
 verarbeitete wird.
Am Ende haben wir genau überprüft, ob sich alle erwarteten Daten genau
so in der Datenbank des \f-Systems befinden.

Dieses geht bei der Verwendung des Mocks besser als beim Test gegen ein 
richtiges Systeme, da die Rückgaben vom \acr{msf}-Mock immer genau vorhersagbar
sind im Gegensatz zu einem richtigen Angriff bei dem auch etwas falsch laufen kann.
Einen sehr ähnlichen Test führen wir auch als Integrationstest mit einem
echten \acr{msf} gegen echt virtuelle Maschinen aus, dieser benötigt aber
bei den meisten Entwicklern 10 mal so lange zum ausführen.
