\section{Testszenarien}
\label{sec:test_scenario}
\authors{\HM, \LM}{\MW}

Im folgenden wird nun die geplante Test-Umgebung und die Testfälle die abgedeckt werden sollen genauer beschrieben.

\subsection{Umgebung}
Die Tests des \f-Systems sollen einfach durchführbar sein und gewährleisten
das die Software komplett funktioniert. Im Idealfall sollen solche Tests
regelmäßig und automatisch durchgeführt werden. Deswegen wird auf das \acr{ci}
Jenkins\footnote{\url{http://jenkins-ci.org/}} zurückgegriffen. Dieses
System kann automatisch nach Änderungen Tests durchlaufen lassen und die
Benutzer benachrichtigen, falls etwas nicht funktioniert.

Um Jenkins einsetzen zu können, wurden zum einen \glos{rake}-Tasks für das
automatische durchführen der Tests erstellt. Hier wurde sich an die
\glos{rails} Aufteilung der Tests in Unit-, Functional- und Integrationstest
orientiert:
\begin{itemize}
  \item Unit: Tests die ein Model/Knowledge auf ihre Funktion überprüfen
  \item Functional: Testet einzelne Komponenten auf ihre Funktionsfähigkeit
  \item Integration: Überprüfen ob das Zusammenspiel zwischen Komponenten
funktioniert, hier werden auch \acrpl{vm} angegriffen, wodurch diese Tests
aufwendig und langsam sind.
\end{itemize}

Zum anderen wurde eine Testlandschaft erstellt, in der Systeme zum Angreifen
für die Jenkins-Tests bereit gestellt werden. In der Testlandschaft befinden
sich die benötigten \acrpl{vm} für die verschiedenen Tests. Die Tests können
entsprechend konfiguriert werden, dass sie auf den bereitgestellten Systemen
durchgeführt werden.

Natürlich kann die Testlandschaft auch von allen Entwicklern verwendet werden,
damit diese keine eigene Testumgebung auf ihren Rechnern konfigurieren müssen.

\subsection{Szenarien}
Typische Tests für unser System beinhalten den Angriff von einen oder mehreren
Hosts. Wenn das \f-System vollständig gestartet ist, soll die Konfiguration
und Steuerung der Hosts automatisch geschehen, deswegen wurde die Steuerung
der \acrpl{vm} in die Tests integriert.

Vor dem Test, \zB wenn das Angreifen eines Hosts getestet wird, werden die
benötigten \acrpl{vm} hochgefahren. Anschießend wird ein \glos{exploit}
ausgeführt und bei erfolgreicher Ausführung soll das \f-System eine
\acr{msf}-Session erhalten. Hier würde \zB überprüft werden, ob die erwartete
Session auch wirklich erstellt wurde. Die\acr{vm}-Steuerung für Tests
ermöglicht es auch, vor einem Test mehrere \acrpl{vm} zu starten, wenn
\zB ein Pivoting-Test durchgeführt werden soll.
