\section{Motivation}
\label{sec:test_goals}
\authors{\HM, \LM}{\MW \and \JF}
Das wesentliche Ziel des \f-Systems, die
Evaluation von \acr{ids}, wird durch Ausführen von \glospl{exploit}
und Auswerten der \acr{ids}-Events erreicht. Ein wesentlicher
Bestandteil, der beim \f-System getestet werden muss, ist also das
Ausführen von \glospl{exploit} auf Hosts und anschließender Kontrolle,
ob der Host durch den \glos{exploit} erfolgreich übernommen
wurde. 

Die Tests sollen gewährleisten, das Änderungen am System direkt überprüft
werden können und so sichergestellt ist, dass das System noch wie erwartete funktioniert.
Außerdem sollen die Tests dazu verwendet werden, den aktuelle Stand
in den Repositories regelmäßig automatisch auf ihre korrekte Funktionalität zu
überprüfen.
Bevor wir diese Testinfrastruktur aufgebaut hatten, kam es öfter vor,
dass das System für eine Woche nicht funktionierte, weil durch Änderung
ein Fehler eingebaut wurde und dieser nicht bemerkt wurde.
Diese Fehler ließen sich oft erst durch längeres Debuggen finden und es
ließ sich nicht sofort erkennen, wo die Fehlerquelle liegt.

Zum übernehmen von Hosts während der Tests, werden Angriffsziele benötigt,
auf denen die \glospl{exploit} ausgeführt werden können. Da das \f-System
auf den Laptops der Entwickler implementiert wird, müssten die Angriffsziele
allerdings auf jedem Laptop bereitgestellt werden, um das \f-System
vollständig testen zu können.

Die Bereitstellung der Angriffsziele wird noch erschwert, da die
\glospl{exploit} Schwachstellen von unterschiedlichen Services und
Betriebssystemen ausnutzen und demzufolge das komplette Repertoire an
Systemkonfigurationen abgedeckt werden muss. Zusätzlich müssen für
bestimmte Funktionen des \f-Systems die Angriffsziele auch noch nach
einem bestimmten Schema aufgebaut sein, \zB muss es Angriffsziele
geben, die erst nach erfolgreicher Übernahme eines bestimmten Hosts
für den Angreifer sicht- und angreifbar werden.

Wenn jeder Entwickler die verschiedenen Hosts und Betriebssysteme auf dem
eigenen System während der Tests zur Verfügung haben muss, wäre dies sehr
aufwändig. Deswegen soll eine zentrale Test-Infrastruktur geschaffen
werden, welche die benötigten Hosts und Betriebssysteme bereitstellt.

Während der Test-Durchführung muss sichergestellt werden, dass die
anzugreifenden Hosts auch aktiv sind. Deswegen sollen die Hosts über
virtuelle Maschinen zur Verfügung gestellt werden und über unser System
gesteuert werden.
