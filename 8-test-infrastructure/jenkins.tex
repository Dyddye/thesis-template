\section{Jenkins}
\label{sec:test_jenkins}
\authors{\HM}{\BK \and \LM \and \MW \and \JF}
Jenkins (ehemals Hudson) ist ein \glos{cis} und unterstützt Entwickler beim
Testen ihrer Anwendungen.
Entwickler führen nicht immer alle automatischen Tests aus, nachdem sie Änderungen
an der Software vorgenommen haben, da das Ausführen oft einige Zeit benötigt und
eine speziell konfigurierte Testumgebung voraussetzt, wie die Erreichbarkeit von
bestimmten \acr{vm}s oder andere laufende Programme oder spezielle Einstellungen
im Betriebssystem, oder es wird einfach vergessen die Tests auszuführen.
Damit fehlschlagende Tests und damit Fehler in der Software, von den Entwicklern
schnell erkannt und behoben werden können, ist es gut, die Tests regelmäßig
automatisch ausführen zu lassen, z.B.
in bestimmten Zeitintervallen oder wenn Änderungen an der Software
vorgenommen wurden.
Diese Aufgabe kann Jenkins erfüllen.

\subsection{Jenkins Einsatz bei FIDIUS}

Da wir zu Anfang oft das Problem hatten, dass sich Fehler in der Software befunden
haben, die ein Angreifen und übernehmen eines anderen Hosts unmöglich gemacht haben 
und es oft lange gedauert hatte, bis die Ursache des Fehlers gefunden
wurde, haben wir uns entschieden ein \glos{cis} zu benutzen.
Da schon eine Jenkins Installation vorhanden war und wir damit zufrieden waren,
haben wir diese Installation so konfiguriert, dass sie unsere Tests regelmäßig automatisch ausführt.

Für die Unit- und Funktionstests und für die Integrationstests jedes \f-Teilprojekts
wurde ein Job in Jenkins erstellt, welcher regelmäßig das Repository auf Änderungen überprüft
und dann die Tests ausführt.
Ein Job bezeichnet in Jenkins eine Aufgabe wie das regelmäßige überprüfen eines
bestimmten Repositories auf Änderungen und das Ausführen von dessen Tests.

Da wir eine Testlandschaft mit mehren verwundbaren \acr{vm}s haben und diese
für unsere Tests verwenden möchten wurde eine VPN-Verbindung zu dieser
Testlandschaft eingerichtet, damit die Integrationstests diese
Hosts angreifen können.
Damit können wir automatisch überprüfen, ob das System sich noch wie erwartet verhält
und durch die Regelmäßigkeit des Ausführens schnell herausfinden, welche Änderungen ein
Problem verursacht haben.
