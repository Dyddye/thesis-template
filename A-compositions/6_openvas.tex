\section{OpenVAS}
\authors{\JF}{}
\label{compositions:openvas}

Das Referat beschäftigt sich mit Nessus und OpenVAS (Open Source
Vulnerability Assessment Scanner), zwei Sicherheitstools aus der
Kategorie der Vulnerability Scanner. Diese bieten die Möglichkeit,
Systeme auf bekannte Schwachstellen zu untersuchen. Da Nessus in der
aktuellen Version nur unter einer kommerziellen Lizenz für den
professionellen Gebrauch zu erhalten ist, konzentriert sich dieses
Referat mehrheitlich auf OpenVAS, das aus der letzten freien Version
von Nessus hervorgegangen ist und seither getrennt unter den
Bedingungen der GPL weiterentwickelt wird.

Begonnen wird mit einem kurzen Überblick, was den Bereich des Vulnerability Scannings umfasst 
und wozu es überhaupt nötig ist, Computersysteme auf ihre Schwachstellen zu testen. Im 
Anschluss folgt eine Vorstellung der Komponenten von OpenVAS und deren Funktionsweise und 
Bedeutung innerhalb des Gesamtsystems. Dabei wird näher darauf eingegangen, aus welchen 
Quellen fertige Tests für neu bekannt gewordene Schwachstellen bezogen und wie diese in 
die Überprüfung integriert werden können. Im Anschluss wird die Erstellung eigener Network 
Vulnerability Tests (NVT) mit Hilfe der Nessus Attack Scripting Language (NASL) vorgestellt.

Abschließend wird auf den möglichen Nutzen spezieller Komponenten für FIDIUS näher eingegangen. 

\subsection{Überblick}

Das Ausnutzen von Schwachstellen in Rechnernetzen und Computersystemen durch Angreifer 
stellt heutzutage eine beträchtliche Bedrohung für den Betriebsablauf in Unternehmen dar. 
\cite{Staniford:2002}
Um dies zu verhindern, also mögliche Schwachstellen VOR dem Angreifer zu erkennen, müssen 
die Systeme proaktiv auf bekannte Sicherheitslücken untersucht werden. Hier kommt das 
Vulnerability Scanning ins Spiel. Mit Hilfe von speziellen Tools werden Systeme automatisiert 
auf bekannte Schwachstellen untersucht und die Ergebnisse dieser Suchen können in mehr oder 
weniger aussagekräftigen Berichten analysiert werden.   

Die bekanntesten kommerziellen Vertreter aus der Kategorie Vulnerability Scanner sind 
sicherlich SAINT\footnote{SAINT® product suite: \url{http://www.saintcorporation.com/}} 
und Nessus\footnote{Network Vulnerability Scanner Nessus: \url{http://www.nessus.org/nessus/}}. 
Diese kommen aus Kostengründen allerdings nicht für ein studentisches Projekt in Frage. 
Somit werden diese beiden Scanner in diesem Referat auch nicht näher behandelt.

Stattdessen beschäftigt sich dieses Referat mit dem Open Vulnerability Assessment Scanner 
(OpenVAS), der aus der letzten freien Nessus-Version (Nessus stand bis Version 2.2 unter 
der GPL) hervorgegangen ist und seitdem als freier Vulnerability Scanner angeboten 
und weiterentwickelt wird.

\subsection{OpenVAS Allgemein}

Wie im vorherigen Abschnitt erwähnt, ist OpenVAS aus der letzten freien Version von 
Nessus hervorgegangen und teilt somit noch einige Funktionen mit Nessus. Dazu gehört 
\uA die Architektur des Gesamtsystems (Abbildung~\ref{fig:openvas-structure}) sowie die 
Nessus Attack Scripting Language (NASL), mit der die Tests auf Schwachstellen implementiert 
werden. Für dieses Referat wurde die Version 3.1 näher beleuchtet.


\begin{figure}
\begin{center}
\includegraphics[scale=1.2]{images/openvas-structure}
\caption{OpenVAS Komponenten, (\url{http://www.openvas.org}, \protect\glos{cc3})}
\label{fig:openvas-structure}
\end{center}
\end{figure}

\subsubsection{Komponenten}

\textbf{Scanner:} \texttt{openvassd}

Der OpenVAS Scanner ist die Serverkomponente, welche das tatsächliche Testen der 
Schwachstellen übernimmt. OpenVAS erlaubt die Verwendung von mehreren Instanzen des 
Scanners, damit Scanaufgaben parallel durchgeführt und die Verarbeitung beschleunigt 
werden kann. Der Scanner ist in der Abbildung~\ref{fig:openvas-structure} genau wie 
der gleich vorgestellte Manager im Server beheimatet.  

\textbf{Manager:} \texttt{openvasmd}
\label{openvas:openvas-manager}

Der OpenVAS Manager ist eine neue Komponente, die erst in 2010 veröffentlicht wurde. 
Er stellt eine Abstraktionsschicht zwischen dem Scanner und den unterschiedlichen Clients 
dar. Er bietet die Möglichkeit, dass die konfigurierten Scanaufträge in einer SQL-Datenbank 
zentral auf dem Server gespeichert werden. Steuern lässt sich der Manager über das OpenVAS 
Management Protocol (OMP), mit dem Clients XML-basierte Nachrichten zur Steuerung benutzen 
können. Dies umfasst beispielsweise das Starten/Stoppen eines Scanauftrags oder Abrufen von 
Berichten zu vergangenen Scans. Zudem bietet der Manager die Erstellung von Notizen und 
Annotationen zu Scans an. \cite{openvas-openvasmd} \cite{openvas-omp}

\textbf{Verwaltung:} \texttt{openvas-administrator}

Diese Komponente dient dazu, Benutzer hinzuzufügen sowie Rechte zu vergeben, welche die 
Erlaubnis zum Scannen von bestimmten Hosts oder Subnetzen erteilen. Zudem unterstützt 
der Administrator das Anlegen von Zertifikaten für die verschlüsselte Kommunikation der 
Komponenten untereinander. \cite[S.19]{openvas-kompendium}

\textbf{Clients (diverse)}

OpenVAS bietet verschiedene Benutzeroberflächen an, unter anderem ein Commandline Interface 
(omp-cli) sowie verschiedene GUI-Clients. Unter anderem hat die Firma 
Greenbone\footnote{Greenbone: \url{http://greenbone.net/index.de.html}}, die große Teile 
des Quellcodes zum OpenVAS-Projekt beigesteuert hat, eine alternative GUI entwickelt.
Mit den Clients lassen sich Konfigurationen für Scans anlegen, die beispielsweise bestimmte 
zu testende Dienste oder Hosts definieren. 
\cite[S.35-49]{openvas-kompendium} 

\textbf{Library:} \texttt{libopenvas}

Die libopenvas wird von allen weiteren Komponenten als Basis benötigt. Sie stellt 
die Funktionalität von OpenVAS zur Verfügung. Die einizge Dokumentation für die 
Benutzung der Library scheint allerdings das Commandline Interface zu sein, sodass 
dessen Funktionsumfang für eine mögliche Verwendung der libopenvas zunächst einmal 
in einer an das Projektwochenende anschließenden Arbeitsgruppe näher evaluiert werden muss. 

\subsubsection{Plugins/Feeds}

Um regelmäßig Schwachstellentests durchzuführen bedarf es einer regelmäßigen 
Aktualisierung der zur Verfügung stehenden Tests für aktuell bekannt 
gewordene Schwachstellen. Dazu stellen die OpenVAS Entwickler einen Feed zur Verfügung, 
mit dem das Programm mit neuen Tests versorgt wird. Pro Plugin ist ein Test enthalten. 
\cite{openvas-feed}

Mit dem mitgelieferten Skript \texttt{openvas-nvt-sync} lassen sich die Plugins 
aktualisieren. Bisher ist mir nur dieser Feed bekannt, es ist aber auch möglich, 
selbst Tests zu schreiben und diese mit Hilfe eines eigenen Feeds anzubieten. 

\subsubsection{Grundsätzlicher Ablauf eines Vulnerability Scans mit OpenVAS}

In diesem Abschnitt soll beispielhaft beschrieben werden, wie ein Test auf 
Schwachstellen mit OpenVAS durchgeführt wird. Zunächst muss man sich Gedanken 
darüber machen, auf welchem Rechner im Netz der OpenVAS Scanner laufen soll. 
Da alle Scans von dem Rechner aus durchgeführt werden auf dem der Scanner läuft, 
muss sichergestellt werden, dass alle zu testenden Zielsysteme von diesem 
erreichbar sind. \cite[S. 13]{openvas-kompendium}

Wenn der Scanner läuft gibt es die optionale Möglichkeit, den neuen OpenVAS Manager 
als Schicht zwischen Scanner und Client zu verwenden 
(siehe Abschnitt~\ref{openvas:openvas-manager}). Alternativ verbindet man sich direkt 
mit einem Client mit dem Scanner.

Besteht eine Verbindung mit dem Scanner, kann man über die GUI die Konfiguration für 
den Scanvorgang vornehmen. Die Einstellungsmöglichkeiten reichen von der Anzahl der 
auszuführenden Tests bis zu den zu überprüfenden Hosts.

Im Anschluss an jeden Scanvorgang erhält man einen Bericht über die gefundenen 
Schwachstellen und kann anhand dessen die Anfälligkeit des Systems bewerten.    

\subsection{Network Vulnerability Tests}

Die Network Vulnerability Tests (NVT), die von OpenVAS verwendet werden, sind in der Nessus 
Attack Scripting Language (NASL) implementiert. Wie dem Namen zu entnehmen ist, stammt 
die Entwicklung dieser Skriptsprache noch aus der Zeit, als Nessus unter einer Open-Source 
Lizenz verfügbar war. NASL ist eine einfache Skriptsprache, die es ermöglicht, mit Hilfe 
von bereitgestellten Funktionen in relativ schneller Zeit einen Test für eine bekannte 
Schwachstelle in einem System zu schreiben. Betrachtet wird die Version 2 von NASL, da 
diese im Vergleich zur ersten Version einen größere Anzahl eingebauter Funktionen besitzt.
\cite[S. 3]{openvas-nasl-reference} 

\subsubsection{NASL}

Die Syntax von NASL ähnelt der von C. Die Sprache bietet die Möglichkeit, 
Bedingungen, Schleifen und Funktionen zu nutzen. Der Aufbau der Skripte ist 
vorgegeben. Sie setzen sich zusammen aus einer Beschreibung des jeweiligen 
Tests und dem eigentlichen Testcode. \cite[S.26]{openvas-nasl-guide}

\textbf{Funktionsumfang}

NASL ist einzig und allein für den Zweck entwickelt worden, NVTs zu schreiben. 
So bietet die Sprache keine Funktionen um externe Programme zu starten, wie 
dies in anderen Skriptsprachen wie \zB Ruby möglich ist.

Die integrierte Funktionalität beschränkt sich \uA auf die folgenden Bereiche:

\begin{itemize}
  \item Manipulation von IP-Paketen
  \item Abfragen des Scannerstatus
  \item Abfrage von Hosts, Ports, ...
  \item HTTP-Funktionen
  \item Kryptografische Funktionen
\end{itemize}  

Die vollständige NASL2-Referenzdokumentation ist in \cite{openvas-nasl-reference} zu finden.

\subsubsection{NASL Beispiel} 

Im folgenden Listing ist beispielhaft ein NVT in NASL
dargestellt\footnote{Dieser Test befindet sich im
  Installationsverzeichnis von OpenVAS in der Datei
  \texttt{plugins/secpod\_ms10-046.nasl }.}. Dieser Test prüft, ob der
getestete Host anfällig für die Windows LNK-Schwachstelle
(CVE-2010-2568\footnote{CVE-2010-2568:
  \url{http://web.nvd.nist.gov/view/vuln/detail?vulnId=CVE-2010-2568}})
ist. Zunächst folgt der allgemeine Teil mit der Beschreibung des Tests
und etwaigen Referenzen auf bestehende CVE-Nummern. Diese
Informationen werden dann im jeweiligen Client bei Auswahl des Tests
angezeigt.

\begin{verbatim}
if(description)
{
  script_id(902226);
  script_version("$Revision$:1.0");
  script_cve_id("CVE-2010-2568");
  script_bugtraq_id(41732);
  script_tag(name:"cvss_base", value:"9.3");
  script_tag(name:"risk_factor", value:"Critical");
  script_name("Microsoft Windows Shell Remote...");
  desc = "
  Overview: This host has critical security update
  missing according to Microsoft Bulletin MS10-046.
  ...
  script_description(desc);
  script_summary("Check for the vulnerable Shell32.dll file version");
  exit(0);
}

...
\end{verbatim}

Im Anschluss daran wird die eigentliche Prüfung auf die Schwachstelle durchgeführt. 
Dies erfolgt in diesem Fall durch die Prüfung, ob ein bestimmtes Patchlevel vorhanden 
ist. Ist dies nicht der Fall, dann ist der Host anfällig für diese Schwachstelle. Sind 
alle geforderten Patches dagegen vorhanden, wird die Version der \verb|Shell32.dll| geprüft 
und wenn diese unterhalb der geforderten Versionsnummer liegt, wird eine Meldung über eine 
kritische Schwachstelle (\verb|security_hole|) an den Client zurückgeliefert. 

\begin{verbatim}

# Windows XP
if(hotfix_check_sp(xp:4) > 0)
{
  SP = get_kb_item("SMB/WinXP/ServicePack");
  if("Service Pack 3" >< SP)
  {
    # Grep for Shell32.dll version < 6.0.2900.6018
    if(version_is_less(version:sysVer, test_version:"6.0.2900.6018"))
    {
      security_hole(0);
    }
    exit(0);
  }
  security_hole(0);
}

...

\end{verbatim}

Viele NVTs sind so wie im Beispiel aufgebaut. Das bedeutet, um die Versionssnummern der 
DLLs abfragen zu können, benötigt man einen Benutzernamen und Passwort auf dem zu 
überprüfenden Rechner. Daher muss noch weiter evaluiert werden, ob OpenVAS auch Tests bietet, 
die Schwachstellen eines Systems auch von außerhalb ermitteln können.


\subsection{Möglicher Nutzen für FIDIUS}

Aufgrund der Erkenntnisse über OpenVAS lassen sich drei mögliche Anwendungen für 
FIDIUS ableiten. Zunächst wäre es möglich, dass sich das entwickelte FIDIUS-Angriffstool 
der NVTs aus dem OpenVAS-Feed bedient. Damit könnten eventuell Schwachstellen getestet 
werden, für die Metasploit keine Exploits bietet. Auch der Weg zurück, also 
gewissermaßen die Autokonfiguration von OpenVAS durch das FIDIUS-Tool ist denkbar.

Die zweite Einsatzmöglichkeit bezieht sich auf NASL, da mit Hilfe dieser Skriptsprache 
eigene Tests geschrieben werden können, die weitere Schwachstellen auf bereits 
kompromittierten Hosts aufdecken können.

Der letzte Ansatz zur Einbindung von OpenVAS betrifft die CVE-Nummern. Die generierten 
Berichte enthalten die CVE-Nummern zu den gefundenen Schwachstellen, die geparst werden 
können und als Eingabe für Metasploit verwendet werden könnten.

Die genannten Ansätze setzen allerdings voraus, dass OpenVAS aus dem FIDIUS-Tool heraus 
angesprochen werden kann und Scanvorgänge daraus gesteuert werden können. Dies bedarf 
noch weiterer Evaluation, vielversprechend klingt hier die libopenvas.
