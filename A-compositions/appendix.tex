\chapter[Referatsausarbeitungen]
  {Referatsaus-\\arbeitungen}
\label{chp:compositions}

\section{Motivation}

Wie auch im ersten Projektjahr fand zu Beginn des Wintersemesters ein Projektwochenende
statt. Darin wurde über den bisherigen Weg reflektiert und der Zukünftige angerissen.
Für die gemeinsamen Treffen wurde dieses Mal allerdings keine Jugendherberge organisiert.
Externe Termine auf Seiten der Betreuer und eine mangelnde Bereitschaft auf studentischer
Seite, Geld und Zeit in weite Anreisen zu investieren führten dazu, dass wir auf dem
Campus der Universität Bremen verblieben. Das Projektwochenende fand am 17.--19.~Oktober
2010 statt.

Im Folgenden seien kurz die Referats-Themen aufgelistet, die von der Projektbetreuung im
Voraus an die Teilnehmer verteilt worden sind. Die Reihenfolge orientiert sich dabei
an der Vortragsreihenfolge vom Wochenende.

\section{Ablauf}

\paragraph*{Freitag, 17.10.2010}

\begin{tabbing}
  ================== \= ====================================== \= \kill
  \LM:  \> \ref{compositions:cert-cve} CERT \& CVE                              \> S.~\pageref{compositions:cert-cve}\\
  \AB:  \> \ref{sec:compositions:netflow-basics} NetFlow \& IPFIX -- Grundlagen \> S.~\pageref{sec:compositions:netflow-basics} \\
  \KUH: \> \ref{compositions:netflow-security} NetFlow \& IPFIX -- Sicherheit   \> S.~\pageref{compositions:netflow-security} \\
\end{tabbing}

\paragraph*{Samstag, 18.10.2010}

\begin{tabbing}
  ================== \= ====================================== \= \kill
  \DH: \> \ref{compositions:botnetze} Botnetze                                \> S.~\pageref{compositions:botnetze} \\
  \MW: \> \ref{compositions:honeynets} Honeypots                              \> S.~\pageref{compositions:honeynets} \\
  \JF: \> \ref{compositions:openvas} OpenVAS                                  \> S.~\pageref{compositions:openvas} \\
  \BK: \> \ref{sec:comp:nse} NMap Scripting Engine                            \> S.~\pageref{sec:comp:nse} \\
  \DE: \> \ref{compositions:msf} Wie und wofür können wir Metasploit nutzen?  \> S.~\pageref{compositions:msf} \\
  \HM: \> \ref{compositions:meterpreter} Metasploits Meterpreter              \> S.~\pageref{compositions:meterpreter} \\
\end{tabbing}

\paragraph*{Sonntag, 19.10.2010}

\begin{tabbing}
  ================== \= ====================================== \= \kill
  \WF: \> \ref{compositions:wuermer} Würmer                   \> S.~\pageref{compositions:wuermer} \\
  \DM: \> \ref{compositions:stuxnet} Stuxnet                  \> S.~\pageref{compositions:stuxnet} \\
  \CA: \> \ref{composition:tunnel} Tunnel -- Hole-Punching    \> S.~\pageref{composition:tunnel} \\
  \DK: \> \ref{sec:compositions:timeanalysis} Zeitreihen      \> S.~\pageref{sec:compositions:timeanalysis} \\
\end{tabbing}



% 
% datei anlegen und verlinken
% 
% bitte keine nummerierung im dateinamen verwenden. die sind redundant, und (wie
% an den hier gelisteten beispielen erkennbar) fehleranfällig.
% 

% Freitag
\section{CERT und CVE}
  \label{compositions:cert-cve}
  \authors{\LM}{}
\subsection{CERT}
\subsubsection{Einführung von CERT} Das erste Computer Emergency
Response Team (CERT) wurde im Anschluss an den Morris-Wurm 1988
gegründet. Der Morris-Wurm hatte damals ca. 10\% der Internet fähigen
Rechner infiziert und lahm gelegt. Das erste CERT wurde an der
Carnegie Mellon University, Pittsburgh gegründet und von der Defense
Advanced Research Projects Agency (DARPA/U.S.-Regierung)
unterstützt. Das Team sollte als Ansprechpartner bei zukünftigen
Security-Emergencies dienen und zusätzlich auftreten zukünftiger
Incidents verhindern.

Im Laufe der Zeit wurden immer mehr CERT-Organisationen, vor allem
nationale von Ländern unterstütze Organisationen gründet, um mit den
Anwachsenden des Internets und der Anzahl der Auftretenden Incidents
mit zuhalten.

Seit 1988 bzw. 1998 hat CERT damit begonnen sowohl Advisorys über
Schwachstellen, Beschreibung der Schwachstelle und mögliche Lösungen,
als auch Informationen über Security-Incidents, dies sind Viren,
Würmer oder allgm. Schädliche Programme, die Systeme Angreifen, zu
veröffentlichen. Seit 2004 werden diese Informationen von US-CERT im
Rahmen von Technical Cyber Security Alerts veröffentlicht.

\subsubsection{Advisories}

Ein Advisory beginnt mit der Beschreibung der betrachteten
Schwachstelle und auf welchen Systemen sie ausgenutzt werden
kann. Zusätzlich wird darauf eingegangen welche Auswirkungen das
Ausnutzen dieser Schwachstelle auf ein System haben kann.

Hauptsächlich behandelt ein Advisory die Möglichkeiten, die man zum
lösen der Schwachstellen besitzt, damit das System wieder sicher
ist. Hier bei Kann sich um Updates oder Upgrades von Anwendungen oder
Diensten oder bestimmten Konfigurationseinstellungen von System oder
Firewall handeln.

Zusätzlich wird noch auf Anmerkungen von Herstellern verwiesen, die
von diesem Problem betroffen sind.

\subsubsection{Incident Notes} Eine Incident Note bezieht sich auf
einzelne Security Incidents, meistens entweder ein Wurm oder Virus,
teilweise aber auch auf Code mit dem eine bestimmte Schwachstelle
ausgenutzt werden kann.

Eine Incident Note beginnt mit einer kurz Beschreibung des Incidents
gefolgt mit einer genaueren Beschreibung. Hierbei wird zum einen der
Verbreitungsweg beschrieben als auch die ausgenutzten
Schwachstellen. Außerdem wird beschrieben welche Schritte durchgeführt
werden, damit sich der Angreifer auf den System festsetzen kann.

Wenn es noch weiterführende Informationen von
Security-Information-Providern gibt wird auf diese anschließend
verlinkt.

Anschließend wird auf Möglichkeiten eingegangen die Infizierung zu
verhindern.  Dies umfasst zum einen aufspielen von vorhanden Patches,
aktuelle Versionen von Anti-Viren Programmen oder Verhaltensweisen
beim Erhalt noch Daten aus dem Internet.

Zusätzlich wird noch Hilfestellung gegeben, wie mit einem
kompromittierten System umgegangen werden kann.
\subsubsection{Aktuelle Arbeit} Mittlerweile veröffentlicht das erste
CERT keine Informationen mehr zu Vulnerabilities und Incidents sondern
beschäftigt sich hauptsächlich mit den folgenden Themen Bereichen:
\begin{itemize}
 \item Software Assurance
 \item Secure Systems
 \item Organizational Security
 \item Coordinating Response
 \item Training
\end{itemize}

Im Rahmen von Software Assurance befassen sie sich mit Möglichkeiten,
wie bei der Entwicklung von Anwendungen Schwachstellen vermieden
werden können. Cert bieten auch die Möglichkeit an, entwickelte oder
bereits verwendete Anwendungen auf Schwachstellen zu untersuchen um
Schwachstellen zu entdecken und beheben. Außerdem untersuchen sie auch
bösartigen Code um so einen besseren Einblick in die Funktionsweise zu
erhalten.

\subsection{US-CERT} Das US-CERT wurde 2003 als Teil des Department of
Homeland Security gegründet und dient als Ansprechpartner für
Security-Incidents für alle US-Behörden oder mit der US-Regierung
verbundenen Organisationen. Und dient auch als Ansprechpartner zum
Thema IT-Security und der US-Regierung.

US-CERT ist zwar unabhängig von anderen CERT-Organisationen, steht
aber sehr wohl in Kooperation mit diesen Organisationen um
beispielsweise Security Incidents zu beheben.

\subsubsection{Technical Cyber Security Alerts} Innerhalb der
Technical Cyber Security Alerts werden zeitnah Informationen über
Schwachstellen, Security Incidents und Exploits veröffentlicht. Dabei
handelt es sich um eine Mischung der Advisories und Incident Notes die
bis 2004 von CERT veröffentlicht wurden.

Sie bestehen aus folgenden Bestandteilen:\\
\begin{itemize}
 \item Befallene Systeme
 \item Beschreibung
 \item Auswirkung
 \item Gegenmaßnahmen
 \item Quellen
\end{itemize}

Genau wie bei den Advisories und Incident Notes wird die behandelte
Schwachstelle beschrieben genauso wie die Möglichkeiten diese
Auszunutzen und welche Systeme betroffen sein können.  Des weiteren
wird auch wieder die Auswirkungen und die Möglichen Gegenmaßnahmen
eingegangenen.

Für weiterführende Informationen wird auf
Security-Information-Provider und die Hersteller verwiesen.

\subsubsection{Cyber Security Bulletins} Bei den Cyber Security
Bulletins handelt es sich um eine wöchentliche Zusammenfassung
aktueller Schwachstellen die in der Woche registriert sind. Dabei
werden die Informationen aus der National Vulnerability Database (NVD)
zusammengefasst.

Die Schwachstellen werden anhand von CVE-Nummern beschrieben, und auch
auf ihren Risikofaktor anhand von CVSS eingestuft.

Für jede Schwachstelle wird aufgelistet welche Produkte und Hersteller
davon betroffen sind und dazu gibt es eine kurze Beschreibung. Neben
der Risikoeinschätzung anhand von CVSS wird noch auf den entsprechende
CVE oder Einträge von Security-Information-Providern verwiesen für
zusätzliche Informationen verwiesen.

\subsection{Vulnerability Notes Database}

Bisher haben wir uns mit verschiedenen Veröffentlichungen befasst, die
mit dem Thema Schwachstellen zu tun hatten. Diese bezogen sich alle
entweder nur auf bestimmte Security Incidents oder Vulnerabilities,
oder auf Vulnerabilities die in einem bestimmte Zeitraum auftraten.

Allerdings bergen nicht alle Vulnerabilities ein so hohen
Risikofaktor, dass eine Veröffentlichungen in einzelnen Berichten wie
Technical Cyber Security Alerts rechtfertigbar sind.  Risiko setzt
sich hierbei aus der Auswirkung der Schwachstelle zusammen,
z.B. Root-Rechte auf dem Host und der Häufigkeit von anfälligen
Systemen. Eine Schwachstelle die alle Windows-Versionen betrifft, hat
einen höheren Risikofaktor als eine Schwachstelle in einen veralteten
Ubuntu-Kernel.

Deswegen gibt es gesammelte Datenbanken die alle veröffentlichten
Informationen über Vulnerabilities sammeln und der Öffentlichkeit
Bereitstellen. Über diese können nun auch Vulnerabilities mit einem
geringeren Risikofaktor veröffentlicht werden.  Die gesammelten Daten
erleichtern es zusätzlich z.B. auch Systemadministratoren den
Überblick über Schwachstellen in eingesetzten Systemkonfigurationen zu
wahren.

Solche Datenbanken werden von verschiedenen
Security-Information-Providern, wie z.B US-CERT, wie z.B National
Vulnerability Database (NVD), bereitgestellt. Da leider die meisten
Security-Information-Provider auf eine eigene Namensgebung
zurückgreifen, gibt es leider teilweise nur geringe Vergleichbarkeit
zwischen den verschiedenen Datenbanken.

\subsection{CVE}
\subsubsection{Motivation} Um Einträge aus verschiedenen Datenbank
miteinander verknüpfen zu können, wurde 1999 Common Vulnerabilities
and Exposures (CVE) gestartet. Das Ziel war hierbei nicht, ein
Nachschlagewerk für alle vorhandenen Schwachstellen zu erstellen,
sondern eine Art Wörterbuch mit denen die verschiedenen Informationen
über Vulnerabilities nachgeschlagen werden können und nicht erst für
jeden Security-Information-Provider den Namen einer Vulnerability zu
suchen, da sie alle eigene Namenskonventionen verwenden.

Dies ermöglicht nun den Datenaustausch zwischen verschiedenen
Datenbanken und auch den Vergleich der Datenbanken. So kann
z.B. festgestellt werden welche Datenbanken Einträge zu bestimmten
Schwachstellen besitzen. Dadurch ergibt sich natürlich auch die
Möglichkeit Informationen über einen Vulnerability aus verschiedenen
Datenbanken zu gelangen.
\subsubsection{CVE-Nummer} Eine einzelne CVE-Nummer besteht aus den
Präfix CVE- gefolgt von der Jahreszahl und am Schluss einen
Identifier. Für den Identifier wird einfach die Anzahl der gefunden
Vulnerabilities in diesem Jahr genommen. Eine typische CVE-Nummer
sieht also folgendermaßen aus:\\
\begin{center} \textit{CVE-2008-4250}
\end{center} Mit dieser CVE-Nummer wird also die Vulnerability 4250
aus dem Jahr 2008 identifiziert.  Die CVE-Nummer wird beim einreichen
der Vulnerability bei einer \textit{CVE Candidate Numbering Authority}
vergeben. Gleichzeitig erhält der CVE auch den
\enquote{candidate}-Status und wird anschließend dem \textit{CVE
Editorial Board} vorgelegt. Wird der CVE akzeptiert, erhält er den
\enquote{entry}-Status, die Vergabe einer CVE-Nummer ist keine
Garantie dafür, dass die Vulnerability auch wirklich ein CVE-Entry
wird.
\subsection{Security Ecosystem}

Bisher haben wir nur die Möglichkeiten betrachtet, von denen
Informationen über Vulnerabilites erhalten werden können
betrachtet. Bisher wurde aber noch nicht der Zeitpunkt der
Veröffentlichung betrachtet. Der Veröffentlichungszeitpunkt ist
allerdings ein wichtiger Faktor der betrachtet werden muss.

Vorweg sollten nun einmal geklärt werden welche Personen in Rahmen von
Vulnerabilities-Informationen betrachtet werden:\\
\begin{itemize}
 \item Entdecker: Entdecker der Vulnerability, kann eine einzelne
Person oder Organisation sein
 \item Kriminelle: Versucht die Vulnerability auszunutzen um einen
eigenen Vorteil zu erhalten
 \item Hersteller: Versucht die Vulnerability zu beheben, da mit seine
hergestellte Software sicherer ist
 \item SIP: Stellt Informationen bereit, welche Schwachstellen es
gibt, welche Systeme betroffen sind wie die Schwachstelle behoben
werden kann
 \item Öffentlichkeit: Möchte aktuelle Informationen über
Schwachstellen erhalten, um das eigene Risiko abschätzen zu können
\end{itemize}

Wenn ein Entdecker seine Informationen an Kriminelle weiterverkauft,
werden diese versuchen die Schwachstelle immer noch geheim zuhalten,
damit diese weiter ausgenutzt werden kann und nicht vom Hersteller
behoben wird. Auch für die Hersteller ist es Interessant, die
Schwachstellen geheim zuhalten. Am liebsten würden diese die
Schwachstellen nicht beheben, da dieses für sie am billigsten
wäre. Andererseits müssen sie veröffentlichte Schwachstellen beheben,
da die Benutzer sonst Aufgrund von massiven Sicherheitsrisiken
aufhören würden ihre Software zu benutzen.

SIP kümmern sich darum, dass Vulnerabilitys veröffentlicht werden und
somit die Öffentlichkeit zu informieren. Der Zeitpunkt um eine
Vulnerability zu veröffentlichen muss gut gewählt werden. Denn zum
einen sollten die Hersteller die Möglichkeit erhalten, die
Schwachstellen rechtzeitig zu beheben, andererseits darf eine
Schwachstelle nicht zu lange verschwiegen werden.

Mit der Veröffentlichung einer Schwachstelle wird die Gefahr dieser
Schwachstelle massiv erhöht. Viele Kriminelle besitzen zwar nicht das
Know-How um Schwachstellen als erstes zu finden, sind aber sehr wohl
in der Lage aus den veröffentlichten Informationen der SIPs die
Schwachstelle herzuleiten und anschließend auszunutzen.


\subsection{Zusammenhang mit FIDIUS} Dadurch das wir nun die
Ausrichtung von \f geändert haben, ist das endgültige Ziel noch
nicht genau definiert. Im ersten Schritt wollen wir versuchen in Hosts
in einen Rechnernetz zu übernehmen. Hierzu müssen wir also die
verschiedenen Hosts analysieren und Schwachstellen zum eindringen
ausnutzen. Im zweiten Schritt sollen nun Informationen von IDS
verglichen werden, um die Trefferquoten vergleichen zu können.

Im ersten Schritt kann nun angesetzt werden. Die Vulnerabilities
Databases enthalten viele Schwachstellen und Informationen, welche
Systeme für diese anfällig sind.  Es gilt also, anhand von
Informationen die über ein Host gesammelt werden können, geeignete
Schwachstellen zu finden und diese dann anzuwenden.

Beim Anwenden von Schwachstellen kann zum Beispiel auf das
Metasploit-Framework zurückgegriffen werden. Dieses bietet die
Möglichkeit Exploits anhand von CVE-Nummern zu finden und anschließend
auszunutzen. CVE-Nummern sind hierbei die Ideale Verbindung zwischen
Vulnerability Notes Databases und Metasploit, da es ein Name für die
selbe Schwachstelle ist.

Ein geeigneter Anwendungsfall wäre als das finden von Schwachstellen
mittels einer Vulnerabilites Notes Database, z.B. NVD, und der
anschließenden Anwendung der Schwachstellen mit Metasploit.

 


 % LM: CVE-Nummern
\clearpage\section[Netflow \& IPFIX -- Grundlagen]
  {Netflow \& IPFIX\\Grundlagen}
\label{sec:compositions:netflow-basics}
\authors{\AB}{}

\subsection{Einleitung}

Im Rahmen der neuen Zielfindung des Projektes \f kam die Idee auf, im
Bereich Intrusion Detection einen neuen Weg einzuschlagen. Während in den ersten
beiden Semestern auf den von Snort erzeugten Events gearbeitet werden sollte,
wurde für die zweite Projekt-Hälfte der Ansatz vorgeschlagen Netflow Daten
auszuwerten und anhand dieser Anomalien in einem Rechnernetz zu erkennen.

Dieses Referat soll erklären was sich hinter Netflow und IPFIX verbirgt, welche
Daten diese Protokolle liefern können und wie die Informationsgewinnung
funktioniert. Als Beispiel dafür dient die von Ethereal entwickelte 
Netzverkehr-Analyse-Software \texttt{nTop}, die unter anderem Netflow Daten sammeln und
in verschiedenen Sichtweisen einem Benutzer zur Verfügung stellen kann.

In einem weiteren Referat -- \enquote{Netflow / IPFIX Security und Diagnose} -- wird näher
darauf eingegangen, wie Netflow Daten zur Erkennung von Anomalien verwendet
werden können.

\subsection{Überblick}
\label{sec:compositions:netflow-basics:overview}

Bei Netflow handelt es sich um ein von Cisco entwickeltes Protokoll, das in
seiner ersten Version 1996 erschienen ist (vgl.~\cite[2]{netflow-oslebo}).
Es dient der Überwachung des IP-Verkehrs in Netzen, in denen Sensoren - sog.
Probes - installiert werden. Diese senden Informationen über den Verkehr an
Daten-Sammelstellen - sogenannte Collectoren, welche die Daten speichern,
weiterverarbeiten oder den Benutzern zur Verfügung stellen.

Insgesamt wurden 10 Netflow Versionen entwickelt, von denen jedoch nur 7
veröffentlicht wurden. Tabelle~\ref{tab:compositions:netflow-versions} zeigt
eine Übersicht der Eigenschaften. (vgl.~\cite{netflow-caligare})

\begin{longtable}{lp{.7\linewidth}}
  \rowcolor{Beige}
    Version & Eigenschaften \\
  \endhead
    \caption[]{Netflow Versionen\\\tabelletbcname}
  \endfoot
    \caption{Netflow Versionen\label{tab:compositions:netflow-versions}}
  \endlastfoot  
  V1 & Beschränkung auf IPv4\\
  V5 & IPv4, Border Gateway Protocol (BGP) und Autonomous
System (AS) Informationen, Sequenz-Nummern.\\
  V6 & Entwicklung für speziellen Kunden, wird nicht verwendet \\
  V7 & Verwendung ausschließlich auf Cisco Catalyst 5000 Switches\\
  V8 & Reduzierung des Export Volumens durch Router-Based Netflow Aggregation 
\cite{netflow-cisco-flow-aggregation}\\
  V9 & Einführung von Templates, IPv6, Multicasts, IPSec. MPLS \\
  V10 (IPFIX) & Erweiterung von V9, IETF Standardisiert, standard-export über
SCTP (statt UDP)
\end{longtable} 

Die durch Netflow-Probes gesammelten Daten sind vielseitig verwendbar. Anhand
der Informationen über den stattgefundenen Netz-Verkehr wäre beispielsweise die
Kosten Abrechnung eines Providers möglich, der anhand des aufgekommenen
Datenvolumens Rechnungen an seine Klienten ausstellt. Es lassen sich ebenso
Netztopologien erstellen anhand derer ersichtlich wird, wo im Netz wichtige
Knotenpunkte vorhanden sind die besonders belastet werden oder selber viel
Verkehr erzeugen.

Im Kontext von \f sind hierbei Sicherheitsaspekte
(siehe~\cite[6]{netflow-nvisionip-security-visualizations}) besonders
interessant, denkbar wäre unter anderem die Erkennung von:

\begin{itemize}
  \item Wrum-infizierten Rechnern, die atypischen Verkehr erzeugen.
  \begin{itemize}
    \item z.B. ein vom Slammer Wurm infizierter Mail-Server, der ständig
UDP-Pakete an Port 1434 schickt.
  \end{itemize}
  \item Übernommene Systeme, die Teil eines Botnetzes sind.
  \begin{itemize}
    \item Erkennung von erhöhtem Traffic auf Port 6667 \& 6668
  \end{itemize}
  \item Port Scans, die möglicherweise Vorbereitung eines Angriffs sind.
  \begin{itemize}
    \item Flows die eine vielzahl an Ports eines oder mehrerer Zielsysteme
erreichen
  \end{itemize}
  \item (Distributed) Denial of Service Angriffe.
  \begin{itemize}
    \item Ausgehende Angriffe: gleichzeitiges Ansteigen des Traffics auf vielen
Hosts. Eingehende Angriffe: Hohes eingehendes Datenvolumen bei dem angegriffenen
Host.
  \end{itemize}
\end{itemize}

\subsection{Funktionsweise}

Im Folgenden wird die Funktionsweise von NetFlow erläutert. Dazu gehört die Art,
in der Informationen gesammelt werden können und die Infrastruktur die dazu
benötigt wird.

\subsubsection{Flows}

Das Grundprinzip, das NetFlow zugrunde liegt, ist die Zusammenlegung des
IP-Vekehrs in sogenannte Flows. Die Einteilung wird anhand verschiedener
Parameter vorgenommen, die als Flow-Keys bezeichnet werden. Während die
Flow-Keys in Netflow V1 bis V8 eine fixe Menge bilden, aus der eine Teilnemge
ausgewählt werden kann, bieten V9 und IPFIX die Verwendung von Templates, mit
deren Hilfe eigene Flow-Keys definiert werden können.

Die Einteilung eines Flows geschieht in NetFlow V5 üblicherweise anhand eines
7-Tupels (vgl. \cite{netflow-cisco-flow-information}) bestehend aus:

\begin{itemize}
  \item IP-Adresse (Quelle \& Ziel)
  \item Port-Nummer (Quelle \& Ziel)
  \item IP Protokoll
  \item Type of Service
  \item Input logical Interface
\end{itemize}

Eine NetFlow-Probe fasst Pakete die in ihren Flow-Keys übereinstimmen zu einem
Flow zusammen und sendet die gesammelten Informationen an einen oder mehrere
Collectoren. Dabei gibt es unterschiedliche Ursachen wann ein Flow beendet wird.
Bei verbindungsorientierten Protokollen kann dies die Beendigung der Verbindung
sein, was in der Probe bei TCP anhand des FIN-Pakets erkannt werden kann.
Weiterhin kann ein Flow beendet werden, wenn nach einem definierten Timeout
keine neuen Pakete eingetroffen sind, die diesem Flow zugeordnet werden können.
Bei lang anhaltenden Flows können diese nach einem Zeitintervall beendet werden.
Anschließend entsteht ein neuer Flow mit denselben Flow-Keys. Dadurch werden
lange Flows in feste Intervalle aufgeteilt damit die gesammelten Informationen
bereits an den Collector übertragen werden können, obwohl der Flow eigentlich
noch anhält. Sowohl der Timeout für Flows ohne neue Pakete als auch der
Intervall bei lang anhaltenden Flows sind variabel und in der NetFlow-Probe
konfigurierbar.

Neben der normalen Erfassung aller Pakete besteht die Möglichkeit des Paket
Samplings. Dabei werden nurnoch Stichproben der Pakete genommen, wobei die Art
und Granularität vom Benutzer bestimmt werden.
Tabelle~\ref{tab:compositions:netflow-sampling} zeigt 3 mögliche Varianten des
Samplings, wobei \textbf{N} vom Benutzer gewählt werden kann. (siehe
\cite{netflow-cisco-flow-sampling})

\begin{longtable}{lp{.7\linewidth}}
  \rowcolor{Beige}
    Methode & Beschreibung \\
  \endhead
    \caption[]{Paket-Sampling Varianten\\\tabelletbcname}
  \endfoot
    \caption{Paket-Sampling Varianten\label{tab:compositions:netflow-sampling}}
  \endlastfoot  
  Deterministisch & Jedes \textbf{N}-te Paket wird verwendet \\
  Zufällig & Alle \textbf{N}-Pakete wird eins ausgewählt und der Rest nicht
betrachtet\\
  Zeit basiert & Jede \textbf{N}-te Millisekunde wird das aktuellste Paket
ausgewählt
\end{longtable} 

Da ein Router (oder eine andere Probe) nicht mehr alle Pakete in Flows einteilen
muss wird dessen CPU entlastet und das Export Volumen zum Collector sinkt. Daher
eignet sich Paket Sampling besonders für Netze mit hohem Datenaufkommen, in
denen nicht alle Ströme erfasst werden müssen, sondern ein grober Überblick über
die Auslastung ausreicht.

\subsubsection{Templates}

Mit Version 9 von NetFlow wurden Templates eingeführt, durch die Flows flexibler
eingeteilt werden können als anhand der festen Menge an Flow-Keys in den vorigen
Versionen. IPFIX als Erweiterung von NetFlow V9 ermöglicht ebenfalls die
Verwendung von Templates. Dazu werden 2 unterschiedliche Arten von Datensätzen
von der Probe an den Collector geschickt: Template Records und Flow Data
Records.

Template Records definieren \texttt{\{Typ, Länge\}}-Paare, die angeben
welche Informationen in den Flow Data Records enthalten sind. Die Zuordnung der
Templates zu Flow Daten geschieht über eine Template ID, die im Template Record
enthalten ist und im Collector gespeichert wird.

Abbildung~\ref{fig:compositions:netflow-template} zeigt ein Export Paket, das
Template und Data Records enthält. Als \texttt{\{Typ, Länge\}}-Paar ist im
Template beispielsweise \texttt{\{IPv4\_SRCADDR (0x0008), Length = 4\}} zu
sehen, im Data Record erhält es dann einen konkreten Wert:
\texttt{192.168.1.12}. Über die ID -- in diesem Fall 256 -- werden die Daten dem
Template zugeordnet.

\begin{figure}
  \begin{center}
    \includegraphics[scale=0.7]{images/netflow-v9-template.png}
    \caption[Netflow FlowSet]{Netflow FlowSet, Quelle: Cisco NetFlow Services Solutions Guide,
    \url{http://www.cisco.com/en/US/docs/ios/solutions_docs/netflow/nfwhite.html\#wp1052564}, abgerufen am 14.11.2010}
    \label{fig:compositions:netflow-template}
  \end{center}
\end{figure}

Export Pakete die von der Probe an den Collector geschickt werden können
beliebig gemischte Template und Data Records enthalten
(siehe \cite[7, 8]{rfc-3954}):

\begin{itemize}
  \item Nur Template Records
  \begin{itemize}
    \item z.B. wenn die Probe gestartet wird und alle Templates and den
Collector schickt
  \end{itemize}
  \item Sowohl Template als auch Data Records
  \begin{itemize}
    \item z.B. wenn während dem verschicken von Data Records ein neues Template
an den Collector gesendet werden soll
  \end{itemize}
  \item Nur Data Records
  \begin{itemize}
    \item z.B. der Collector hat alle nötigen Templates und bekommt daher nur
Data Records geschickt
  \end{itemize}
\end{itemize}

\subsubsection{Probes}

Wie bereits erwähnt werden bei NetFlow Informationen über den IP-Verkehr durch
Probes gesammelt und an einen oder mehrere Collectoren gesendet. Probes können
Router sein die Hardware- oder Softwareseitig NetFlow unterstützen, aber auch
jeder beliebige andere Rechner in einem Netz, auf dem eine NetFlow fähige
Software (z.B. ntop) installiert ist. Dabei spielt es eine große Rolle an
welcher Position sich die Probe im Netz befindet.

Fungiert ein Router als Probe, an den mehrere Rechner als Stern-Topologie
angeschlossen sind und über den der Internet Zugang realisiert wird, wird dieser
Flow-Informationen über den gesamten Netzverkehr im LAN und \enquote{nach draußen}
erhalten. Ist die Probe hingegen auf einem Rechner installiert, der an einem
Router oder Switch angeschlossen ist, wird die Probe nur Informationen über den
Verkehr des Rechners selber erhalten. (Abgesehen von Broadcasts oder wenn der
IP-Verkehr des Switches/Routers über einen Mirror Port an den als Probe
fungierenden Rechner weitergeleitet wird.)

Der Transport der gesammelten Flow-Informationen geschieht bis NetFlow V8 über
UDP. V9 und IPFIX können auch andere Transportprotokolle verwenden, im RFC 3954
wird dazu SCTP genannt (\cite[6]{rfc-3954}), das im Gegensatz zu UDP
Überlastungskontrolle besitzt. Auch wenn es bei UDP keine Garantie dafür gibt,
dass Export Pakete ihr Ziel erreichen, enthält der NetFlow Header ab V5 eine
Squenz-Nummer, anhand der erkannt werden kann ob auf dem Weg von Probe zum
Collector Pakete verloren gegangen sind~\cite{netflow-cisco-sequence-number}.

\subsubsection{Collectoren}

Collectoren sind Software-Lösungen, die ihre Daten von NetFlow-Probes empfangen,
ggf. speichern und dem Benutzer in verschiedenen Arten präsentieren können. Im
Falle von \texttt{ntop} kann die Software  auch mittels NetFlow Plugin als Probe und
Collector in einem dienen. Die Aufbereitung hängt dabei davon ab, welche
Erkentnisse aus den Flow-Informationen gezogen werden sollen.

Abbildung~\ref{fig:compositions:netflow-live-action} zeigt eine Ansicht der
Cisco \enquote{LiveAction} Oberfläche. Diese stellt die Kommunikations-Beziehungen
zwischen mehreren Hosts eines LAN und der Außenwelt dar. Weiterhin sind die
Interfaces dargestellt (kleinere, grüne Kreise)  über die der Verkehr
stattfindet. Pfeile die von der unteren Kreis-Hälfte ausgehen bedeuten dabei
ausgehender Verkehr, Pfeile die zur oberen Hälfte führen Eingehender.

\begin{figure}
  \begin{center}
    \includegraphics[scale=0.3]{images/netflow-live-action-screen.jpg}
    \caption{LiveAction Screenshot, Quelle: CiscoGuard
    \label{fig:compositions:netflow-live-action}
(\url{http://www.ciscoguard.com/LiveAction.asp}, Abgerufen am 14.11.2010)}
  \end{center}
\end{figure}

\subsection{ntop \& nProbe}

Mithilfe von \texttt{ntop} und \texttt{nProbe} ist es Möglich NetFlow Daten zu sammeln und
darzustellen. Abbildung~\ref{fig:compositions:netflow-net} zeigt eine mögliche
Netztopologie mit folgenden Komponenten:

\begin{itemize}
  \item (1) Router, der Zugang zum Internet hat, sendet NetFlow Daten über einen
Mirror Port an (4)
  \item (2) Switch an dem 2 weitere Rechner und ein Server angeschlossen
sind
  \item (3) File-Server auf dem nProbe läuft, sendet NetFlow Daten an (4)
  \item (4) Workstation auf der ntop als Collector arbeitet
\end{itemize}

Während der Router die Verkehrsdaten ins Internet und zwischen dem linken und
rechtem Teilnetz auswerten kann, wäre Verkehr innerhalb des rechten Teilnetzes -
zum Beispiel wenn Workstation (4) Dateien vom File-Server (3) abruft für den
Router nicht sichtbar. Dies kann durch eine Probe auf dem File-Server (3)
umgangen werden, die ihrerseits NetFlow Daten sammelt und ebenfalls an den
Collector (4) schickt. Dadurch kann der gesamte IP-Verkehr des Netzes
ausgewertet werden.

\begin{figure}
  \begin{center}
    \includegraphics[scale=0.4]{images/netflow-net.png}
    \caption{Netzbeispiel}
    \label{fig:compositions:netflow-net}
  \end{center}
\end{figure}

\subsection{Verwendung in \f}

Wie im Überblick bereits angesprochen 
(siehe~\ref{sec:compositions:netflow-basics:overview}) könnte man NetFlow
im Kontext von \f dazu verwenden um mithilfe von Zeitreihenanalysen
(siehe Abschnitt~\ref{sec:compositions:timeanalysis}) Anomalien zu erkennen,
die auf Angriffe, Würmer, etc. hindeuten können. Neben der Ausarbeitung zu
Zeitreihenanalysen behandelt dies das \enquote{Netflow / IPFIX Security und Diagnose}
Referat. (Abschnitt~\ref{compositions:netflow-security})

Um NetFlow Daten zu erhalten könnte dazu ntop sowie nProbe verwendet werden.
Während ntop frei verfügbar ist, ist nProbe in der standard Version ab 100€
erhältlich (siehe \url{http://www.nmon.net/shop/cart.php}). Es besteht jedoch
die Möglichkeit für Universitäten eine freie Lizenz zu erhalten, solange diese
nicht für kommerzielle Zwecke genutzt wird, wodurch auch nProbe für bei \f
verwendet werden kann.

\subsection{Weiterführende Links \& Literatur}

\url{http://netflow.caligare.com/applications.htm} : Übersicht Software die
Netflow unterstützt. Abgerufen am 05.11.2010
 % AB
\clearpage\section[Netflow \& IPFIX -- Security und Diagnose]
  {Netflow \& IPFIX\\Security und Diagnose}
\label{compositions:netflow-security}
\authors{\KUH}{}

\subsection{Einleitung}
Dieses Referat soll einen Überblick darüber geben, inwiefern sich
Netflow/IPFIX zur Diagnose und zur Verbesserung der Sicherheit eines
Netzes verwenden lassen. 

Durch die Verwendung von Netflow/IPFIX soll auch die Möglichkeit gegeben
werden, ein Netz mit hoher/sehr hoher Auslastung (GB-Bereich) zu
überwachen. Im Rahmen des Vortrages möchte ich vor allem darauf
eingehen, wie man mit Netflow/IPFIX es dem Benutzer (Administrator)
eines Netzes vereinfachen kann, Angriffe auf das Netz oder Anomalien im
Netz zu finden und zu beheben. Dabei gibt es verschiedene Verfahren, um
Schlüsse über mögliche Angriffe oder Anomalien zu ziehen.

Zum einen gibt es statische Verfahren, die ähnlich wie Snort auf
festgelegten Regeln basieren, zum anderen gibt es Verfahren, die
lernfähig sind und sich am verhalten des Traffics im Netz orientieren.

Als Beispiel für bisherige Implementationen dient die Programmsammlung
NFDUMP mit der Benutzeroberfläche NfSen.

\subsection{Netflow vs. libpcap}

In diesem Teil schaue ich mir Netflow und die libpcap ein wenig näher an.
Ich werde auf Tests der jeweiligen Implementierung eingehen und inwiefern
dies für den FIDIUS-Kontext von Bedeutung ist.

\subsubsection{libpcap}

Die libpcap wird bei vielen gängigen passiven Paketerfassungs-Programmen
verwendet. Die Bibliothek selber ist relativ einfach zu bedienen und es
gibt schon sehr gute Tutorials, die die Verwendung genauer erklären%
\footnote{\url{http://yuba.stanford.edu/~casado/pcap/section1.html},
abgerufen am 13.11.2010}.

Die libpcap wird bisher den Programmen: Tcpdump%
\footnote{\url{http://www.tcpdump.org/}, abgerufen am 14.11.2010}, 
Ethereal\footnote{\url{http://www.ethereal.com/}, abgerufen am 14.11.2010}
sowie das von uns bisher verwendete Snort verwendet.

Libpcap hat jedoch das Problem, dass das heutige Datenvolumen nicht mehr
vollständig erfasst werden kann. Wobei hier jedoch die Unterscheidung
zwischen großen und kleinen Paketen gemacht werden muss. Dies wird in
den Tests noch näher erläutert. Das eben genannte Problem wird auch
\textsl{livelock} genannt.

\begin{figure}
  \centering
  \includegraphics[width=\linewidth]{images/pcap_abb1}
  \caption{Darstellung des livelock Problems}
  \label{fig:netflow-security:livelock}
\end{figure}
 
\subsubsection{Netflow}

Netflow wurde von Cisco entwickelt um auch Netze mit höherem
Datenaufkommen, im GBit-Bereich, überwachen zu können. Näheres dazu im
Referat von Andreas Bender. Die Performance der Netflow Kollektoren skaliert stark mit der
eingesetzten Hardware. Dies wird in den Tests auch nochmal verdeutlicht.

\subsubsection{Tests}

Die Tests wurden einem Paper von Luca Deri entnommen\footnote{%
\url{http://citeseerx.ist.psu.edu/viewdoc/download?doi=10.1.1.58.3128&rep=rep1&type=pdf},
abgerufen am 14.11.2010}.

\paragraph{libpcap}

Der Test der libpcap entstand mit folgendem Sender und Empfänger:

\begin{itemize}
 \item Sender: Dual Core 1.8 GHz Athlon, Intel Gbit ethernet card
 \item Empfänger: Pentium III 550 MHz, Intel Gbit ethernet card
\end{itemize}

\begin{figure}
  \centering
  \includegraphics[width=.8\linewidth]{images/pcap_abb_2}
  \caption{Test der libpcap}
  \label{fig:netflow-security:test-libpcap}
\end{figure}

Wie man anhand der Tabelle erkennen kann, sind die prozentualen Werte
der Paket-Erfassung nicht wirklich gut. FreeBSD mit Polling bietet im
großen und ganzen die besten Werte im Gegensatz zu den beiden anderen
Betriebssystemen. Es fallen jedoch bei größeren Paketen noch viele
Pakete durchs Raster durch. Dies ist nicht sonderlich gut, wenn wir das
Netz überwachen wollen.

\paragraph{Netflow}

\subparagraph{Test 1}

Der erste Test mit Netflow entstand mit folgendem Sender und Empfänger:

\begin{itemize}
 \item Sender: Dual Core 1.8 GHz Athlon, Intel Gbit ethernet card
 \item Empfänger: Pentium III 550 MHz, Intel Gbit ethernet card
\end{itemize}

\begin{figure}
  \centering
  \includegraphics[width=.8\linewidth]{images/netflow_abb_1}
  \caption{Test von Netflow 1}
  \label{fig:netflow-security:test-netflow-1}
\end{figure}

In der Tabelle erkennt man, wieviele Pakete pro Sekunde verschickt
wurden. Dies kann man auch auf den Test der libpcap beziehen. Man erkennt
in der Tabelle deutlich, dass ohne Polling ein Großteil der kleinen
Pakete erfasst werden, jedoch die Rate der erfassten Pakete kleiner wird
bei größeren Paketen. Dies passiert mit Polling ebenso, nur werden
weniger kleine Pakete erfasst.

\subparagraph{Test 2}

Der zweite Test mit Netflow entstand mit folgendem Sender und Empfänger:

\begin{itemize}
  \item Sender: Dual Core 1.8 GHz Athlon, Intel Gbit ethernet card
  \item Empfänger: Pentium 4 1.7 GHz, Intel GE 32-bit
\end{itemize}

\begin{figure}
  \centering
  \includegraphics[width=.8\linewidth]{images/netflow_abb_2}
  \caption{Test von Netflow 2}
  \label{fig:netflow-security:test-netflow-2}
\end{figure}

Bei diesem Test wurde der Empfänger von der Hardware her um einiges
verbessert. Stellt jedoch noch keinen Highend Computer dar. Man erkennt
deutlich die Verbesserung im Gegensatz zum Ersten Test. Es werden immer
noch nicht alle kleinen Pakete erfasst, jedoch werden sämtliche größeren
Pakete erfasst. Somit erkennt man die Skalierung von Netflow mit der
Hardware sehr deutlich.

\paragraph{Ergebnisse der Tests}

Netflow liefert weitaus bessere Ergebnisse bei größeren Paketen ab.
Allerdings haben libpcap sowie auch netflow Probleme mit den kleinen
Paketen. Dies ist wohl daraus begründed, dass einfach zuviele Pakete
ankommen, die in einer Sekunde verarbeitet werden sollen. Wie man im
zweiten Test von Netflow gesehen hat, waren es über eine halbe Million
Pakete in der Sekunde. Da kam die Hardware an ihre Grenzen. 

Da die Tests jedoch auf relativ alten Computern ausgeführt wurden und
auch schon älteren Datums sind, stellen die Tests nicht den heutigen
Stand der Dinge dar. Aufgrund dessen, dass die meisten Angriffe mit
kleinen Paketen gemacht werden, bietet es sich weiterhin an mit den
aktuellen Sensoren weiter zu arbeiten.

\subsection{Probleme von Netflow}

Ebenso wie auch schon bei der libpcap, kann es bei Netflow zu einem
\textsl{livelock}~\ref{fig:netflow-security:livelock} kommen. Dazu hat
man bei Netflow jedoch mehrere Möglichkeiten diesen zu umgehen.

\begin{itemize}
  \item Verringern der Anzahl der Flows
  \item Flows  zusammenfassen
  \item Starten des filtern, bevor die Daten in den Flow gepackt werden
  \item Ändern der Filter-Parameter
  \item Ändern der Prozess Priorität
  \item Im schlechtesten Fall: Stoppen der Messung
\end{itemize}

Ein weiteres Problem bei Netflow stellt der Verlust von Flow-Daten dar.
Auch hierfür gibt es verschiedene Lösungsansätze um das Problem zu umgehen.

\begin{itemize}
  \item Erneutes senden der Flow-Daten
  \item Verbindungsabbruch und Fehlersuche
  \item Bestätigung der Flow-Daten vom Kollektor
\end{itemize}

Weiterhin können die Flow-Daten natürlich auch von Angreifern abgefangen
werden, sofern die Übertragung nicht anständig gesichert ist. Um so etwas
zu umgehen, sollten zum einen die Daten nicht übers Internet versendet
werden. Die Daten sollten Anonymisiert werden, sowie für die Übertragung
verschlüsselt werden.

\subsection{Analyse Verfahren}

Zum Analysieren von Flow-Daten gibt es zwei gängige Verfahren. Zum einen
Top-N-Analyse und zum anderen die Analyse über Zeitreihen. 
Näheres über Zeitreihen ist dem Referat von Daniel Kohlsdorf zu entnehmen.

\subsubsection{Top-N-Analyse}

Bei der Top-N-Analyse werden die Flow-Daten anhand fester Regeln, ähnlich
wie bei Snort, durchgesehen. Dabei hat der Benutzer mehrere Möglichkeiten
diese Regeln selber zu definieren. Es sind bei weitem nicht so viele
Regeln wie z.B. in Snort. Die meisten Kriterien für die Regeln sind
gespeicherte Daten im flow. Ausserdem werden viele Regeln schon von
Visualisierungsprogrammen zusammengefasst. Somit brauch der Benutzer
z.B. nur nach den Elementen suchen, wo der meiste Traffic anfällt. 

\subsection{NFDUMP}

Für das Referat habe ich mir ein Programm-Paket angeschaut, dass schon
die Verarbeitung von Netflow-Daten anbietet. NFDUMP\footnote{%
\url{http://nfdump.sourceforge.net/}, abgerufen am 14.11.2010} besteht
aus folgenden einzelnen Programmen.

\begin{itemize}
  \item nfcapd: nfcapd ist ein Programm um die Netflow-Daten aus dem Netz
  zu bekommen.
  \item nfdump: Dient der Analyse der Netflow-Daten.
  \item nfprofile: Aufteilen der Netflow-Daten anhand von vorher
  definierten Profilen.
  \item nfclean.pl: Löscht alte, nicht mehr benötigte Netflow-Daten.
  \item ft2nfdump: Ein Tool, dass die Daten von dem Programm flow-tools
  konvertiert.
\end{itemize}

Ein weiterer Vorteil den die Ansammlung bietet, ist das die Netflow-Daten
schon über ein Flag direkt anonymisiert werden können. Somit erspart man
sich ein eigenes Anonymisierungstool.

\subsection{NfSen}

NfSen\footnote{\url{http://nfsen.sourceforge.net/}, abgerufen am 14.11.2010}
ist die Benutzeroberfläche für das Programm-Paket NFDUMP. Mit diesem
Frontend-Programm werden die Netflow-Daten grafisch für den Benutzer 
aufbereitet. Dabei werden verschiedene Möglichkeiten für die Ansicht
angeboten. 

\begin{figure}
  \centering
  \includegraphics[width=\linewidth]{images/nfsen_abb_1}
  \caption{Allgemeine Ansicht}
  \label{fig:netflow-security:test-nfsen-1}
\end{figure}

\begin{figure}
  \centering
  \includegraphics[width=\linewidth]{images/nfsen_abb_2}
  \caption{Genauere Ansicht eines Flows}
  \label{fig:netflow-security:test-nfsen-2}
\end{figure}

\begin{figure}
  \centering
  \includegraphics[width=\linewidth]{images/nfsen_abb_3}
  \caption{Noch genauere Ansicht}
  \label{fig:netflow-security:test-nfsen-3}
\end{figure}

Ausserdem besteht die Möglichkeit eigene Plugins zu entwickeln und zu
integrieren. 

\subsection{Verwendung in FIDIUS}

Der Nutzen für das Projekt wäre wohl, dass man sich bei Angriffen
anschaut, welche vom jeweiligen Sensor erkannt werden. Sei es nun Snort
oder ein Netflow Kollektor. Meiner Meinung nach sollten wir wohl beide
Sensoren benutzen, um den Großteil der Angriffe mitbekommen zu können.

\subsection{Weiterführende Links \& Literatur}

\begin{itemize}
  \item \url{http://www.splintered.net/sw/flow-tools/}
  \item \url{http://eprints.eemcs.utwente.nl/16911/01/JNSM-Netflow.pdf}
  \item \url{http://citeseerx.ist.psu.edu/viewdoc/download?doi=10.1.1.60.6198&rep=rep1&type=pdf}
  \item \url{http://citeseerx.ist.psu.edu/viewdoc/download?doi=10.1.1.74.6338&rep=rep1&type=pdf}
\end{itemize}
 % KUH: Netflow Security

% Samstag
\clearpage\section{Botnetze}
\label{compositions:botnetze}

\subsection{Abstract}
Netzwerke, die aus gekaperten Rechnern unter der gemeinsamen Kontrolle eines 
Angreifers bestehen, gehören zu den größten Gefahren des Internets. Diese Netze 
werden für verteilte Denial-of-Service-Angriffe und zum Versenden von Spam 
verwendet.

Zuerst wird die Frage geklärt, was ein Bot ist und woher der Name stammt. In dem
Zusammenhang wird der Begriff Botnet erläutert. Anschließend folgt eine 
Übersicht aktiver Botnetze, sowie deren ungefähre Größe. Danach wird der 
Einsatzzweck von solchen Botnetzen beschrieben und wie mit Hilfe dieser 
Netzwerke viel Geld verdient werden kann. Der Hauptteil dieser Ausarbeitung 
beschäftigt sich mit den verwendeten Technologien in Botnetzen. Es werden unter 
anderem die Kommunikationsformen, die Verbreitungstrategien und die 
Selbstverteidigung der Bots näher betrachtet. Daraufhin wird aufgezeigt wie 
diese Botnetze erkannt und entfernt werden können. 

Am Ende des Referats wird der mögliche Nutzen bestimmter Bot-Technologien für 
FIDIUS aufgezeigt. 

\subsection{Definition Bot/Botnet}
Bei einem Bot handelt es sich um ein Computerprogramm, das auf einen
kompromittierten System installiert wird. Dieser Bot stellt dem Angreifer einen
Kommunikationskanal zur Verfügung um das System zu kontrollieren. Der Begriff Bot
ist eine verkürzte Form des englischen Wortes Robot. Ein Bot besitzt 
Eigenschaften eines Wurms um sich autonom zu verbreiten, sowie Eigenschaften 
eines Trojaners, um Befehle des Angreifers entgegenzunehmen. 

Werden mehrere Bots zu einem Netzwerk zusammengeschlossen, spricht man von einem
Botnetz.

\subsection{Übersicht aktiver Botnetze}
Die nachfolgende Tabelle~\ref{fig:botnetz_uebersicht} zeigt eine Übersicht der 
bisher bekannten aktiven  Botnetze bis zum Jahr 2009, sowie deren geschätzte 
Größe. Dabei kann ein Computer mit verschiedenen Bots infiziert sein \cite{de-wiki:botnetze}.

\begin{figure}
  \centering
  \includegraphics[width=\linewidth]{images/botnetz_uebersicht}
  \caption{Übersicht von aktiven Botnetzen}
  \label{fig:botnetz_uebersicht}
\end{figure} 

\subsection{Einsatzszenarien}
Um mit infizierten Computern Geld zu verdienen, gibt es mehrere Möglichkeiten.
DDoS-Attacken, Diebstahl vertraulicher Informationen, Spamversand, Phishing, 
betrügerische Generierung von Klicks sowie der Download von Adware
und Schadprogrammen. Mit einem Botnetz können alle oben genannten Angriffe 
gleichzeitig durchgeführt werden. Die Abbildung~\ref{fig:botnetz_einsatz}
veranschaulicht nochmal die möglichen Einsatzszenarien eines Botnetzes \cite{namestnikov}.

\begin{figure}
  \centering
  \includegraphics[width=\linewidth]{images/botnetz_einsatz}
  \caption{Geschäfte mit Botnetzen}
  \label{fig:botnetz_einsatz}
\end{figure} 

In den folgenden Abschnitten werden die drei populärsten Angriffarten beschrieben.

\subsubsection{DDoS-Attacken}
Eine Distributed Denial-of-Service-Attacke (DDoS-Attacke) ist ein Angriff auf ein
Computersystem mit dem Ziel, dieses zu überlasten, damit es nicht mehr in der
Lage ist, Anfragen zu bearbeiten. DDoS-Attacken sind eine effektive Waffe um
Konkurrenten auszuschalten.

\subsubsection{Diebstahl vertraulicher Informationen}
Vertraulichen Informationen wie Kreditkartennummern, Bankinformationen und 
Passwörter zu verschiedenen Diensten (E-Mail, FTP, Instant-Messenger), die auf 
Anwendercomputern gespeichert sind, können von dem Angreifer ausgespäht werden.
Die gestohlenen Informationen werden dann weiter verkauft oder von dem Angreifer
selbst missbraucht.

\subsubsection{Spamversand}
Die gekaperten Computer werden zum Versenden von unerwünschten Mitteilungen
(Spam-Mails) missbraucht. Der Versand von unerwünschten Mitteilungen ist eine 
der wichtigsten Funktionen von Botnetzen.


\subsection{Verwendete Technologien in Botnetzen}
In den folgenden Abschnitten werden Technologien beschrieben, die in Botnetzen
eingesetzt werden.
\subsubsection{Topologien}
Die Grafik~\ref{fig:botnetz_topologie} veranschaulicht die zwei 
möglichen Netzwerk-Topologien eines Botnetzes \cite{dahl}.
\begin{figure}
  \centering
  \includegraphics[width=\linewidth]{images/botnetz_topologie}
  \caption{Botnetz-Topologien}
  \label{fig:botnetz_topologie}
\end{figure}
Bei der zentralisierten Topologie verbinden sich alle
infizierten Computer mit einer Steuerungszentrale ($Command \& Control Centre$) oder
auch $C\&C$. Die Steuerungszentrale registriert neu hinzugekommene Bots in seiner
Datenbank, überwacht ihren Zustand und schickt ihnen Befehle.

Der Aufbau eines dezentralisierten Botnetzes ist in der Praxis aufwändig, da 
jeder neu infizierte Computer über eine Liste der Bots verfügen muss, mit denen 
er sich innerhalb des Botnetzes verbinden soll.

\subsubsection{Kommunikation}
Damit die Befehle des Botnetz-Betreibers alle infizierten Rechner erreichen, 
muss eine Verbindung zwischen den Bots und der Steuerungszentrale bestehen. 
Eine mögliche Kommunikationsart zwischen Bot und Steuerungszentrale ist der
\gls{acr:irc}-orientierte Datenaustausch. Dabei verbindet sich jeder 
infizierte Computer mit einem vordefinierten \gls{acr:irc}-Server. 
Danach empfängt der Bot über einen bestimmten \gls{acr:irc}-Kanal die Befehle des Botnetz-Betreibers.

Eine weitere Kommunikationsart zwischen Bot und Steuerungszentrale ist die 
\gls{acr:im}-orientierte. Diese Kommunikationsart ist nicht sehr populär, weil 
der Datentransfer über \gls{acr:im}-Dienste wie AOL, MSN oder ICQ abläuft
und jeder infizierte Computer dazu einen eigenen \gls{acr:im}-Account benötigt, weil
es nicht möglich ist, sich mit dem gleichen Account von unterschiedlichen 
Computern aus einzuloggen. Zudem verhindern die \gls{acr:im}-Anbieter die automatische
Generierung von Accounts.

Die Web-orientiert Kommunikationsart wird immer beliebter bei Botnetz-Betreibern,
weil solche Botnetze sich einfach einrichten lassen und auf viele Webserver
zurückgegriffen werden kann. Ebenso lässt sich das gesamte Botnetz komfortabel über
ein Web-Interface steuern. Der Bot verbindet sich mit dem Webserver, welcher
als Stererungszentrale dient, und nimmt Befehle entgegen \cite{kamluk}. 

\subsubsection{Scanning-Mechanismen}
Damit sich ein Bot möglichst schnell unentdeckt verbreiten kann, benötigt der
Bot eine Scanning-Strategie. Jeder Bot könnte zur Verbreitung genutzt werden.
Tatsächlich werden nur wenige Bots zum Scannen und Verbreiten genutzt, damit nur
einige Bots entdeckt werden und nicht das gesamte Botnetz. Eine mögliche
Strategie sich zu verbreiten ist das \emph{Hit"=list"=Scanning}. Der Bot 
sucht nicht selbst nach verwundbaren Systeme, sondern nutzt eine vordefinierte
Liste mit verwundbaren Zielsystemen. Diese Liste wird dann abgearbeitet und 
jedes enthaltene System wird angegriffen. Die Liste kann im Bot vorkompiliert
sein oder kann von einem Webserver nachgeladen werden.

Bei der \emph{Topological"=Scanning"=Strategie} verbreitet sich der Bot mittels
eines Peer"=to"=Peer"=Netzes. Dazu wird die Liste der verwundbaren Zielsystemem
von bekannten Peers benutzt.

Die \emph{Flash"=Scanning"=Strategie} bietet eine weitere Möglichkeit um
verwundbare Zielsysteme zu finden. Bei dieser Strategie besteht bereits
eine Liste mit verwundbaren Computersystemen oder sie wird dynamisch
erstellt in dem \glos{googledorks} dazu missbraucht werden. Diese Art der Verbreitung
ist am schnellsten, weil kein Scann mehr durchgeführt werden muss.

Nachdem ein System kompromittiert wurde, benötigen die Bots eine Reinfektions-Strategie, 
um bereits infizierte Systeme nicht nochmal zu scannen bzw. zu infizieren. Dies
erfordert allerdings eine Übersicht aller Bots im Bonetz. Diese 
Reinfektions-Strategie wird auch als \emph{Permutation Scanning} bezeichnet
\cite{baier}.

\subsubsection{Infektion-Mechanismen}
Bei den Infektions-Mechanismen wird zwischen interaktiver Infektion und
automatisierter Infektion unterschieden.

Bei interaktiven Infektionen findet eine direkte Interaktion mit dem
Benutzer des zu infizierenden Computers statt. Der Nutzer des angegriffenen Systems 
wird dazu verleitet infizierte Programme herunterzuladen und auszuführen. Wenn
der Benutzer dazu aufgefordert wird bestimmte Browser-Plugins zu installieren,
so bezeichnet man diese Art von Angriff \emph{Drive"=by"=download}. 

Bei automatisierten Infektionen werden Sicherheitslücken ausgenutzt. Man
unterscheidet zwischen Lokalen-Exploits und Remote-Exploits. Die Lokalen-Exploits
dienen in erster Linie dazu, die vorhanden Rechte auf dem infizierten System
auszuweiten. Die Remote-Exploits dienen zum Infizieren von Computersystemen
über das Internet oder Netzwerk. Bei den automatisierten Infektionen findet
keine Interaktion mit dem Benutzer statt. Der Benutzer merkt in der Regel nicht,
wenn der Computer infiziert wurde \cite{baier}.

\subsubsection{Update-Mechanismen}
Jeder neue Bot bringt einen Update-Mechanismus mit. Mit Hilfe dieser Update-Funktion 
kann der Botnetz-Betreiber Fehler in den Bots beseitigen oder den Bot um weitere 
Funktionalität erweitern. Ebenso wichtig wie die Fehlerbehebung und Erweiterung
des Bots ist die Aktualiserung der Signatur des Bots. Anhand von Signaturen
werden infizierte Programme von Anti-Viren Herstellern erkannt. Und wenn sich die 
Signatur ändert, so wird die Chance erhöht vor Anti-Viren Programmen unentdeckt
zu bleiben.

\subsubsection{Selbstverteitigung-Mechanismen}
Jeder Botnetz-Betreiber hat zwei Sicherheitsziele. Das erste Sicherheitsziel
ist die Stuerungszentrale ($Command \& Control Centre$). Dieser Server ist die 
zentrale Komponente im Botnetz. Diese muss besonders gegen Übernahme geschützt werden.
Eine Schutzmaßnahme vor Abschaltung oder Übernahme der Stuerungszentrale wäre
das verlagern des Serverstandorts ins Ausland wie z.B. Russland oder Panama.
Diese Länder bieten \emph{Bullet"=Proof"=Hosting} an. Das bedeutet, dass Anfragen
und Beschwerden direkt entsorgt werden und denen nicht nachgegangen wird.
Falls die IP-Adresse doch gesperrt werden sollte, so greift man alternativ auf DynDNS
zurück. Aktualisiert sich die IP-Adresse, so wird auch der DNS-Eintrag aktualisiert.
Die neueste Form dabei ist \emph{Fast-Flux}. Bei Fast-Flux werden einer Domain 
viele verschiedene IP-Adressen zugeordnet (Round-Robbin-DNS). Der Bot erhält zufällig 
eine Adresse. Allerdings führt diese Adresse nicht zur richtigen Stuerungszentrale,
sondern zu einem infiziertem Bot, der die Anfrage an den richtige Stuerungszentrale
weiterleitet. Mit dieser Technik bleibt die Stuerungszentrale unentdeckt, da
nur wenige Bots mit dieser kommunizieren.

Damit das Botnetz wachsen kann, müssen sich die Bots gegen Entfernung
vom System schützen. Das ist das zweite Sicherheitsziel eines Botnetz-Betreibers.
Die Bots müssen vor Anti-Viren Programmen geschützt werden. Dazu kann sich
der Bot als Root-Kit im System installieren und ist somit fast unsichtbar im System agieren.
Ebenso sollte der Bot eine Ausführung verhindern, wenn ein Honeypot, VMWare 
oder ein Debugger erkannt wurde. Falls Anti-Viren Programme auf dem infizierten
System entdeckt werden, so sollten diese bei Bedarf deaktiviert werden. Wenn sich
der Bot im System installiert hat, so sollte er sich unauffällig verhalten
(wenig CPU, Speicher und Netzverkehr verbrauchen), damit er weiter unentdeckt
bleibt.

\subsection{Erkennung von Botnetzen}
Botznetze können auf mehrere Arten erkannt werden. Eine Möglichkeit das Verhalten
von Bots zu untersuchen, bieten Honeynets. Honeynets bestehen aus mehreren Honeypots,
die mit miteinander vernetzt sind. Eine weitere Möglichkeit Botnetze zu entdecken
ist die Analyse des Netzwerkverkehrs. Es gibt verschiedene Verfahren die sich zur 
Analyse des Netzwerkverkehrs eignen. 

Die \emph{signaturbasierten Verfahren} untersuchen den Netzwerkverkehr auf eine 
Bot-Infektion. Snort setzt dieses Verfahren ein, um Angriffe zu erkennen.
Der Nachteil dieses Verfahren besteht darin, dass nur bekannte Bedrohungen erkannt
werden können. 

Die \emph{anomaliebasierten Verfahren} beobachten den Netzwerkverkehr auf
Unregelmäßigkeiten. Ein Beispiel hierfür wäre die Übertragung von großen
Mengen an Daten oder die Kommunikation auf unüblichen Ports.

Die \emph{DNS-basierten Verfahren} untersuchen die DNS-Anfragen zur 
Steuerungszentrale. Ein Botnetz kann durch hohen DNS-Verkehr aufgespürt werden.
Der Ansatz von Schonewille und Van Helmond beruht darauf, dass Bots Kontakt zu
deaktivierten $C \& C$-Server herstellen wollen. Dies führt zu einer großen Anzahl
wiederauftretender Namensfehler, welche auffällig werden.

\subsection{Möglicher Nutzen für FIDIUS}
Nach Analyse mehrerer Bots, stellte sich für mich herraus, dass Bots vor allem
auf Windows-Systemen verbreitet sind. Dies liegt nicht nur an der Architektur
von Windows, sondern auch daran, dass die Allgemeinheit fast ausschließlich
Windows nutzt. Für Bot-Entwickler lohnt sich der Aufwand nicht für andere
Betriebssysteme zu entwickeln. Außerdem müssen Bots klein und möglichst ohne
größere Abhängigkeiten auskommen, damit der Bot auf möglichst vielen Computern
ausgeführt werden kann. Dieses ist bei interpretierten Sprachen nicht gegeben,
weil nicht jeder Anwender diese installiert hat. Als Topologie kommt nur der
zentralisierte Ansatz in Frage. Alles andere ist nicht nicht kontrollierbar.
Möglich wäre es die Steuerungszentrale in Ruby zu schreiben, die Bots in C.
Die Bots bekämen keine eigene Intelligenz, sondern würden benötigte Teile von
der Steuerungszentrale nachladen.

 % DH: Botnetze
\clearpage\section{Honeynets}
\label{compositions:honeynets}
Ziel dieser Arbeit ist es kurz zu erläutern was
ein Honeynet ist, wie es funktioniert, wozu es dient und ob der
Einsatz innerhalb des Projektes FIDIUS lohnenswert ist.

\subsection{Honeypot} Ein Honeypot ist ein für einen Angreifer
interessant wirkendes System in einem Produktivnetz. Dabei ist der
Honeypot stark überwacht und meldet jegliche Interaktion, die sich mit
oder von ihm ergibt. In einem Produktivnetz dient ein Honeypot als
Frühwarnsystem, da er sich, im Vergleich zu den restlichen Komponenten
im Produktivnetz, eher als leichtes aber dennoch interessantes Ziel
darstellt. Ein möglicher Angreifer soll durch einen Honeypot vom
Restnetz abgelenkt werden, und seine Angriffe auf den Honeypot
richten.

Ein Honeypot zeichnet sich dadurch aus, dass er so gut wie keine
\textit{False-Positives} erzeugt, da es im Netz ohne Angriff keine
Interaktion mit dem Honeypot gibt. Allerdings bringt ein Honeypot auch
Gefahren mit sich, denn ein Angreifer könnte im Stande sein ihn als
Angriffsplatform zu nutzen.  Deswegen ist es besonders wichtig einen
Honeypot dauerhaft zu überwachen und ihn so einzurichten, dass er im
Zweifelsfall den Dienst verweigert.

\subsection{Honeynet} Ein Honeynet ist ein Netz von verschiedenen
Honeypots, erstellt um sie durch die \textit{black-hat community}
angreifen zu lassen. Im Gegensatz zu einzelnen Honeypots im Netz die
die Aufmerksamkeit von Hackern auf sich ziehen sollen, dient ein
Honeynet dazu Taktiken und die Vorgehensweise von Hackern zu erlernen
und zu analysieren.  Ein Honeynet ist so konstruiert, dass es zwar
übernommen werden kann und man so die Schritte eines Angreifers
nachvollziehen kann, aber das Honeynet selbst nicht weiter benutzt
werden kann um Angriffe außerhalb dieses Netzes durchzuführen.  Um
sicherstellen zu können, dass auftretende Logereignisse nicht vom
Angreifer zerstört werden, ist es sinnvoll für diesen Zweck einen
\textit{Remote-Log-Server} einzurichten. Ein weiterer Vorteil eines
\textit{Remote-Log-Servers} ist es, dass es einem die Möglichkeit
bietet alle Logereignisse an einer zentralen Stelle zu prüfen.

Die Einsatzmöglichkeiten von Honeynetzen variieren von einem reinen
Forschungsnetz bis hin zum Einsatz in einem Produktivnetz. Dabei muss
man sich bewusst sein, dass ein Honeynet durchaus im Stande ist die
Sicherheit eines Netzes, ähnlich wie ein Honeypot, zu erhöhen, aber es
auch einige Risiken mit sich bringt.

Vor dem Aufsetzen eines Honeynet ist es wichtig die folgenden Punkte
zu beachten:
\begin{itemize}
  \item Die Komplexität des Netzes erhöht sich.
  \item Der Zeitaufwand / Kosten für die Wartung und Analyse des
Honeynetzes ist erheblich.
  \item Es ergeben sich rechtliche Verpflichtungen bezüglich des
Verfahren und der Haftbarkeit bei der Übernahme des Honeynetzes.
  \item Die Sicherheit des Produktivnetzes kann sich bei einer
Übernahme verringern.
\end{itemize} 

Grundsätzlich muss man vor der Einrichtung eines
Honeynetzes den Kosten / Nutzen Faktor abschätzen.

\subsection{Funktionsweise und Aufbau} Dieser Abschnitt befasst sich
mit den nötigen Schritten ein Honeynet einzurichten und wie ein
Honeynet im Allgemeinen aufgebaut ist. Die hier aufgeführten Aspekte
beruhen auf den vom \textit{Honeynet-Project} empfohlenen Schritten
ein Honeynet mit Hilfe der \textit{Honeywall} einzurichten.

Die \textit{Honeywall} wurde vom \textit{Honeynet-Project} entwickelt
und ist ein installierbares \textit{Layer-3-Gateway} basierend auf
einem minimal gehaltenen Fedora OS. Der Funktionsumfang der
\textit{Honeywall} wird in dem jeweiligen Abschnitt näher erläutert.

\subsubsection{Data Control} Sobald ein System im Honeynet übernommen
wurde, ist es wichtig die Möglichkeiten des Angreifers unbemerkt so zu
Kontrollieren, dass das übernommene System keinen Schaden an
Fremdsystemen anrichten kann. Diese Kontrollmechanismen fasst man in
diesem Zusammenhang unter dem Begriff \textit{Data Control}
zusammen. Bei Einrichtung eines Honeynetzes, ist dies der wichtigste
zu beachtende Schritt.

Die \textit{Honeywall} bietet für diesen Teil mehrere redundante
Funktion zu Realisierung des \textit{Data Control}. Zum einen kommt
die \textit{Honeywall} mit einem vorkonfigurierten Network Intrusion
Prevention System (NIPS), dass den ausgehenden Datenverkehr überwacht
und bekannte Angriffe nach außen verhindert. Zum anderen ist die
\textit{Honeywall} im Stande ausgehende Verbindungen zu Zählen und zu
limitieren. Der gesammte ausgehende Verkehr wird mittels
\textit{Ip-Tables} zum NIPS geleitet und sorgt dafür, dass wenn das
NIPS ausfällt garkein Verkehr mehr nach außen gelangt.


\subsubsection{Data Capture} Ein Honeynet ergibt erst Sinn, wenn es
möglich ist die Anfallenden Aktivitäten im Honeynet zu analysieren. Zu
diesem Zweck ist es Notwendig Verfahren zu Implementieren die auf so
vielen Ebenen wie möglich das Verhalten eines Angreifers
aufzeichnen. An dieser Stelle ist nochmals zu erwähnen, dass die
anfallenden Logereignisse in einem \textit{Remote-Log-Server}
zusammenlaufen sollten, da es die Analyse vereinfacht und
sichergestellt wird, dass der Angreifer nicht in der Lage ist
Logereignisse zu verändern.  Es ist empfehlenswert Logs zur Auswertung
in folgenden Bereichen bereitzustellen:
\begin{itemize}
  \item Firewall Logs
  \item Network Traffic
  \item System Activity
\end{itemize}
\paragraph*{Firewall Logs} Dieser Teil wird bereits durch die
\textit{Ip-Tables} realisiert, da es alle ein und ausgehenden
Verbindungen logt.
\paragraph*{Network Traffic} Für diesen Zweck lauscht ein Snort
Intrusion Detection System auf dem inneren Netzinterface der
\textit{Honeywall}. Diese Aufgabe könnte zwar auch das bereits
erwähnte NIPS übernehmen, doch ist es sicherer redundant zu arbeiten.
\paragraph*{System Activity} Das loggen der \textit{System Activity}
kann durch \textit{Debugger} oder auch durch das Programm
\textit{Sebek} übernommen werden. Hierzu sendet zum Beispiel
\textit{Sebek} alle auf einem System auftretenden Ereignisse per UDP
an die \textit{Honeywall} oder einen \textit{Remote-Log-Server}.

\subsubsection{Alerting} Auch wenn die Gefahr, dass das Honeynet
missbraucht wird, durch \textit{Data Control} vermindert wird, ist es
weiterhin nötig bei einem Angriff Zeitnah zu reagieren. Aus diesem
Grund muss vor der Endgültigen Inbetriebnahme des Honeynetzes
sichergestellt sein, dass Angriffe gemeldet werden. Wie dies Umgesetzt
wird, ist allerdings von Fall zu Fall unterschiedlich und soll hier
nicht näher erläutert werden.  \clearpage

\subsubsection{Architektur} Der typische Aufbau eines Honeynetzes
(siehe Abb.~\ref{fig:honeynet}) besteht aus der Trennung vom
Produktivnetz (ähnlich einer DMZ) und dem Honeynet mit Hilfe der
\textit{Honeywall}
\begin{figure}[htbp] 
    \center 
    \pgfimage[interpolate=true,width=1.0
    \linewidth]{images/honeynet.png}
    \caption{Honeynet Architecture [KYE: Honeynets 2006]}
    \label{fig:honeynet}
\end{figure}

\subsection{Einsatz in FIDIUS} Der Einsatz eines Honeynetzes im
Projektkontext kann in zwei Arten umgesetzt werden.
\paragraph*{Als typisches Honeynet} Unter typisches Honeynet wird in
diesem Kontext davon ausgegangen, dass das Honeynet einen Zugang zum
Internet bekommt und somit von außen erreichbar ist.  Es würden also
nicht nur Projektmitglieder Zugriff auf das System erhalten, sondern
auch Angreifer aus dem Internet. Dieses Vorgehen hätte den Vorteil,
dass eine große Anzahl von Angriffen auf das System erfolgt und sehr
viele Daten über Angriffe gesammelt werden könnten. Weiterhin wären
die Angriffsdaten sehr realistisch und es bestünde die Möglichkeit
neue Angriffe zu identifizieren. Allerdings würde dieses Vorgehen
einen stark erhöhten Aufwand mit sich bringen, da die aktuellen
Fähigkeiten des Projektes noch nicht ausreichen, um die anfallenden
Daten zu analysieren. Der Aufwand beschränkt sich zu dem nicht nur auf
das erlernen der benötigten Fähigkeiten, sondern auch auf die
Umsetzung der in dieser Arbeit beschrieben Aspekte bezüglich der
Einrichtung des Honeynet. Damit müsste FIDIUS die rechtliche Situation
klären und ein Konzept erarbeiten, dass es erlaubt im Falle einer
Übernahme angemessen zu reagieren.

\paragraph*{Als Sandbox} Der Einsatz eines Honeynetzes als Sandbox
könnte auch von Interesse im Projekt sein, denn ein Honeynet ist von
sich aus sehr realistisch Aufgebaut und deswegen ein ideales
Testnetz. Zudem würde es die Möglichkeit bieten, die ausgeführten
Angriffe näher zu betrachten und zu analysieren. Es könnte also nicht
nur beobachtet werden wie ein Intrusion Detection System auf Angriffe
reagiert, sondern auch wie der Angriff sich auf ein System
auswirkt. Die durch die Honeynet Architektur zusätzlich gewonnen Daten
könnten so verwendet werden, dass die im Projektkontext benutzen
System verbessert werden.  Im Vergleich zum typischen Honeynet, hätte
dieses Vorgehen den Vorteil, dass die ausgeführten Angriffe bekannt
sind. Es würde also entfallen, dass man die Angriffe erst einmal
identifizieren müsste. Leider ist dieser Vorteil auch gleichzeitig ein
Nachteil, denn FIDIUS würde so nur Angriffe analysieren können die dem
Projekt bereits bekannt sind.

\subsection{Fazit} An sich ist der Einsatz eines Honeynetzes im
Projektkontext wünschenswert, da es viel Potential bietet
Angriffsmethoden zu erlernen und zu studieren. Ebenso kann ein
Honeynet als Sandbox die Gefahr verringern, dass der noch zu
entwickelnde FIDIUS-Wurm aus dem Testnetz entwischt.

Für ein typisches Honeynet besitzt das Projekt allerdings nicht die
zeitlichen Ressourcen, da man davon ausgehen muss, dass für jede
Stunde Aktivität im Honeynet ca. 40 Personenstunden benötigt werden
sie zu analysieren.

Für den Einsatz als Sandbox ist es auch zweifelhaft in wiefern sich
der Kosten / Zeitaufwand mit dem Nutzen deckt.
 % MW: Honeypots
\clearpage\section{OpenVAS}
\authors{\JF}{}
\label{compositions:openvas}

Das Referat beschäftigt sich mit Nessus und OpenVAS (Open Source
Vulnerability Assessment Scanner), zwei Sicherheitstools aus der
Kategorie der Vulnerability Scanner. Diese bieten die Möglichkeit,
Systeme auf bekannte Schwachstellen zu untersuchen. Da Nessus in der
aktuellen Version nur unter einer kommerziellen Lizenz für den
professionellen Gebrauch zu erhalten ist, konzentriert sich dieses
Referat mehrheitlich auf OpenVAS, das aus der letzten freien Version
von Nessus hervorgegangen ist und seither getrennt unter den
Bedingungen der GPL weiterentwickelt wird.

Begonnen wird mit einem kurzen Überblick, was den Bereich des Vulnerability Scannings umfasst 
und wozu es überhaupt nötig ist, Computersysteme auf ihre Schwachstellen zu testen. Im 
Anschluss folgt eine Vorstellung der Komponenten von OpenVAS und deren Funktionsweise und 
Bedeutung innerhalb des Gesamtsystems. Dabei wird näher darauf eingegangen, aus welchen 
Quellen fertige Tests für neu bekannt gewordene Schwachstellen bezogen und wie diese in 
die Überprüfung integriert werden können. Im Anschluss wird die Erstellung eigener Network 
Vulnerability Tests (NVT) mit Hilfe der Nessus Attack Scripting Language (NASL) vorgestellt.

Abschließend wird auf den möglichen Nutzen spezieller Komponenten für FIDIUS näher eingegangen. 

\subsection{Überblick}

Das Ausnutzen von Schwachstellen in Rechnernetzen und Computersystemen durch Angreifer 
stellt heutzutage eine beträchtliche Bedrohung für den Betriebsablauf in Unternehmen dar. 
\cite{Staniford:2002}
Um dies zu verhindern, also mögliche Schwachstellen VOR dem Angreifer zu erkennen, müssen 
die Systeme proaktiv auf bekannte Sicherheitslücken untersucht werden. Hier kommt das 
Vulnerability Scanning ins Spiel. Mit Hilfe von speziellen Tools werden Systeme automatisiert 
auf bekannte Schwachstellen untersucht und die Ergebnisse dieser Suchen können in mehr oder 
weniger aussagekräftigen Berichten analysiert werden.   

Die bekanntesten kommerziellen Vertreter aus der Kategorie Vulnerability Scanner sind 
sicherlich SAINT\footnote{SAINT® product suite: \url{http://www.saintcorporation.com/}} 
und Nessus\footnote{Network Vulnerability Scanner Nessus: \url{http://www.nessus.org/nessus/}}. 
Diese kommen aus Kostengründen allerdings nicht für ein studentisches Projekt in Frage. 
Somit werden diese beiden Scanner in diesem Referat auch nicht näher behandelt.

Stattdessen beschäftigt sich dieses Referat mit dem Open Vulnerability Assessment Scanner 
(OpenVAS), der aus der letzten freien Nessus-Version (Nessus stand bis Version 2.2 unter 
der GPL) hervorgegangen ist und seitdem als freier Vulnerability Scanner angeboten 
und weiterentwickelt wird.

\subsection{OpenVAS Allgemein}

Wie im vorherigen Abschnitt erwähnt, ist OpenVAS aus der letzten freien Version von 
Nessus hervorgegangen und teilt somit noch einige Funktionen mit Nessus. Dazu gehört 
\uA die Architektur des Gesamtsystems (Abbildung~\ref{fig:openvas-structure}) sowie die 
Nessus Attack Scripting Language (NASL), mit der die Tests auf Schwachstellen implementiert 
werden. Für dieses Referat wurde die Version 3.1 näher beleuchtet.


\begin{figure}
\begin{center}
\includegraphics[scale=1.2]{images/openvas-structure}
\caption{OpenVAS Komponenten, (\url{http://www.openvas.org}, \protect\glos{cc3})}
\label{fig:openvas-structure}
\end{center}
\end{figure}

\subsubsection{Komponenten}

\textbf{Scanner:} \texttt{openvassd}

Der OpenVAS Scanner ist die Serverkomponente, welche das tatsächliche Testen der 
Schwachstellen übernimmt. OpenVAS erlaubt die Verwendung von mehreren Instanzen des 
Scanners, damit Scanaufgaben parallel durchgeführt und die Verarbeitung beschleunigt 
werden kann. Der Scanner ist in der Abbildung~\ref{fig:openvas-structure} genau wie 
der gleich vorgestellte Manager im Server beheimatet.  

\textbf{Manager:} \texttt{openvasmd}
\label{openvas:openvas-manager}

Der OpenVAS Manager ist eine neue Komponente, die erst in 2010 veröffentlicht wurde. 
Er stellt eine Abstraktionsschicht zwischen dem Scanner und den unterschiedlichen Clients 
dar. Er bietet die Möglichkeit, dass die konfigurierten Scanaufträge in einer SQL-Datenbank 
zentral auf dem Server gespeichert werden. Steuern lässt sich der Manager über das OpenVAS 
Management Protocol (OMP), mit dem Clients XML-basierte Nachrichten zur Steuerung benutzen 
können. Dies umfasst beispielsweise das Starten/Stoppen eines Scanauftrags oder Abrufen von 
Berichten zu vergangenen Scans. Zudem bietet der Manager die Erstellung von Notizen und 
Annotationen zu Scans an. \cite{openvas-openvasmd} \cite{openvas-omp}

\textbf{Verwaltung:} \texttt{openvas-administrator}

Diese Komponente dient dazu, Benutzer hinzuzufügen sowie Rechte zu vergeben, welche die 
Erlaubnis zum Scannen von bestimmten Hosts oder Subnetzen erteilen. Zudem unterstützt 
der Administrator das Anlegen von Zertifikaten für die verschlüsselte Kommunikation der 
Komponenten untereinander. \cite[S.19]{openvas-kompendium}

\textbf{Clients (diverse)}

OpenVAS bietet verschiedene Benutzeroberflächen an, unter anderem ein Commandline Interface 
(omp-cli) sowie verschiedene GUI-Clients. Unter anderem hat die Firma 
Greenbone\footnote{Greenbone: \url{http://greenbone.net/index.de.html}}, die große Teile 
des Quellcodes zum OpenVAS-Projekt beigesteuert hat, eine alternative GUI entwickelt.
Mit den Clients lassen sich Konfigurationen für Scans anlegen, die beispielsweise bestimmte 
zu testende Dienste oder Hosts definieren. 
\cite[S.35-49]{openvas-kompendium} 

\textbf{Library:} \texttt{libopenvas}

Die libopenvas wird von allen weiteren Komponenten als Basis benötigt. Sie stellt 
die Funktionalität von OpenVAS zur Verfügung. Die einizge Dokumentation für die 
Benutzung der Library scheint allerdings das Commandline Interface zu sein, sodass 
dessen Funktionsumfang für eine mögliche Verwendung der libopenvas zunächst einmal 
in einer an das Projektwochenende anschließenden Arbeitsgruppe näher evaluiert werden muss. 

\subsubsection{Plugins/Feeds}

Um regelmäßig Schwachstellentests durchzuführen bedarf es einer regelmäßigen 
Aktualisierung der zur Verfügung stehenden Tests für aktuell bekannt 
gewordene Schwachstellen. Dazu stellen die OpenVAS Entwickler einen Feed zur Verfügung, 
mit dem das Programm mit neuen Tests versorgt wird. Pro Plugin ist ein Test enthalten. 
\cite{openvas-feed}

Mit dem mitgelieferten Skript \texttt{openvas-nvt-sync} lassen sich die Plugins 
aktualisieren. Bisher ist mir nur dieser Feed bekannt, es ist aber auch möglich, 
selbst Tests zu schreiben und diese mit Hilfe eines eigenen Feeds anzubieten. 

\subsubsection{Grundsätzlicher Ablauf eines Vulnerability Scans mit OpenVAS}

In diesem Abschnitt soll beispielhaft beschrieben werden, wie ein Test auf 
Schwachstellen mit OpenVAS durchgeführt wird. Zunächst muss man sich Gedanken 
darüber machen, auf welchem Rechner im Netz der OpenVAS Scanner laufen soll. 
Da alle Scans von dem Rechner aus durchgeführt werden auf dem der Scanner läuft, 
muss sichergestellt werden, dass alle zu testenden Zielsysteme von diesem 
erreichbar sind. \cite[S. 13]{openvas-kompendium}

Wenn der Scanner läuft gibt es die optionale Möglichkeit, den neuen OpenVAS Manager 
als Schicht zwischen Scanner und Client zu verwenden 
(siehe Abschnitt~\ref{openvas:openvas-manager}). Alternativ verbindet man sich direkt 
mit einem Client mit dem Scanner.

Besteht eine Verbindung mit dem Scanner, kann man über die GUI die Konfiguration für 
den Scanvorgang vornehmen. Die Einstellungsmöglichkeiten reichen von der Anzahl der 
auszuführenden Tests bis zu den zu überprüfenden Hosts.

Im Anschluss an jeden Scanvorgang erhält man einen Bericht über die gefundenen 
Schwachstellen und kann anhand dessen die Anfälligkeit des Systems bewerten.    

\subsection{Network Vulnerability Tests}

Die Network Vulnerability Tests (NVT), die von OpenVAS verwendet werden, sind in der Nessus 
Attack Scripting Language (NASL) implementiert. Wie dem Namen zu entnehmen ist, stammt 
die Entwicklung dieser Skriptsprache noch aus der Zeit, als Nessus unter einer Open-Source 
Lizenz verfügbar war. NASL ist eine einfache Skriptsprache, die es ermöglicht, mit Hilfe 
von bereitgestellten Funktionen in relativ schneller Zeit einen Test für eine bekannte 
Schwachstelle in einem System zu schreiben. Betrachtet wird die Version 2 von NASL, da 
diese im Vergleich zur ersten Version einen größere Anzahl eingebauter Funktionen besitzt.
\cite[S. 3]{openvas-nasl-reference} 

\subsubsection{NASL}

Die Syntax von NASL ähnelt der von C. Die Sprache bietet die Möglichkeit, 
Bedingungen, Schleifen und Funktionen zu nutzen. Der Aufbau der Skripte ist 
vorgegeben. Sie setzen sich zusammen aus einer Beschreibung des jeweiligen 
Tests und dem eigentlichen Testcode. \cite[S.26]{openvas-nasl-guide}

\textbf{Funktionsumfang}

NASL ist einzig und allein für den Zweck entwickelt worden, NVTs zu schreiben. 
So bietet die Sprache keine Funktionen um externe Programme zu starten, wie 
dies in anderen Skriptsprachen wie \zB Ruby möglich ist.

Die integrierte Funktionalität beschränkt sich \uA auf die folgenden Bereiche:

\begin{itemize}
  \item Manipulation von IP-Paketen
  \item Abfragen des Scannerstatus
  \item Abfrage von Hosts, Ports, ...
  \item HTTP-Funktionen
  \item Kryptografische Funktionen
\end{itemize}  

Die vollständige NASL2-Referenzdokumentation ist in \cite{openvas-nasl-reference} zu finden.

\subsubsection{NASL Beispiel} 

Im folgenden Listing ist beispielhaft ein NVT in NASL
dargestellt\footnote{Dieser Test befindet sich im
  Installationsverzeichnis von OpenVAS in der Datei
  \texttt{plugins/secpod\_ms10-046.nasl }.}. Dieser Test prüft, ob der
getestete Host anfällig für die Windows LNK-Schwachstelle
(CVE-2010-2568\footnote{CVE-2010-2568:
  \url{http://web.nvd.nist.gov/view/vuln/detail?vulnId=CVE-2010-2568}})
ist. Zunächst folgt der allgemeine Teil mit der Beschreibung des Tests
und etwaigen Referenzen auf bestehende CVE-Nummern. Diese
Informationen werden dann im jeweiligen Client bei Auswahl des Tests
angezeigt.

\begin{verbatim}
if(description)
{
  script_id(902226);
  script_version("$Revision$:1.0");
  script_cve_id("CVE-2010-2568");
  script_bugtraq_id(41732);
  script_tag(name:"cvss_base", value:"9.3");
  script_tag(name:"risk_factor", value:"Critical");
  script_name("Microsoft Windows Shell Remote...");
  desc = "
  Overview: This host has critical security update
  missing according to Microsoft Bulletin MS10-046.
  ...
  script_description(desc);
  script_summary("Check for the vulnerable Shell32.dll file version");
  exit(0);
}

...
\end{verbatim}

Im Anschluss daran wird die eigentliche Prüfung auf die Schwachstelle durchgeführt. 
Dies erfolgt in diesem Fall durch die Prüfung, ob ein bestimmtes Patchlevel vorhanden 
ist. Ist dies nicht der Fall, dann ist der Host anfällig für diese Schwachstelle. Sind 
alle geforderten Patches dagegen vorhanden, wird die Version der \verb|Shell32.dll| geprüft 
und wenn diese unterhalb der geforderten Versionsnummer liegt, wird eine Meldung über eine 
kritische Schwachstelle (\verb|security_hole|) an den Client zurückgeliefert. 

\begin{verbatim}

# Windows XP
if(hotfix_check_sp(xp:4) > 0)
{
  SP = get_kb_item("SMB/WinXP/ServicePack");
  if("Service Pack 3" >< SP)
  {
    # Grep for Shell32.dll version < 6.0.2900.6018
    if(version_is_less(version:sysVer, test_version:"6.0.2900.6018"))
    {
      security_hole(0);
    }
    exit(0);
  }
  security_hole(0);
}

...

\end{verbatim}

Viele NVTs sind so wie im Beispiel aufgebaut. Das bedeutet, um die Versionssnummern der 
DLLs abfragen zu können, benötigt man einen Benutzernamen und Passwort auf dem zu 
überprüfenden Rechner. Daher muss noch weiter evaluiert werden, ob OpenVAS auch Tests bietet, 
die Schwachstellen eines Systems auch von außerhalb ermitteln können.


\subsection{Möglicher Nutzen für FIDIUS}

Aufgrund der Erkenntnisse über OpenVAS lassen sich drei mögliche Anwendungen für 
FIDIUS ableiten. Zunächst wäre es möglich, dass sich das entwickelte FIDIUS-Angriffstool 
der NVTs aus dem OpenVAS-Feed bedient. Damit könnten eventuell Schwachstellen getestet 
werden, für die Metasploit keine Exploits bietet. Auch der Weg zurück, also 
gewissermaßen die Autokonfiguration von OpenVAS durch das FIDIUS-Tool ist denkbar.

Die zweite Einsatzmöglichkeit bezieht sich auf NASL, da mit Hilfe dieser Skriptsprache 
eigene Tests geschrieben werden können, die weitere Schwachstellen auf bereits 
kompromittierten Hosts aufdecken können.

Der letzte Ansatz zur Einbindung von OpenVAS betrifft die CVE-Nummern. Die generierten 
Berichte enthalten die CVE-Nummern zu den gefundenen Schwachstellen, die geparst werden 
können und als Eingabe für Metasploit verwendet werden könnten.

Die genannten Ansätze setzen allerdings voraus, dass OpenVAS aus dem FIDIUS-Tool heraus 
angesprochen werden kann und Scanvorgänge daraus gesteuert werden können. Dies bedarf 
noch weiterer Evaluation, vielversprechend klingt hier die libopenvas.
 % JF
\clearpage\section{NMap Scripting Engine}
\label{sec:comp:nse}

Ausarbeitung des Referats vom Projektwochenende. Vortragende: Bernhard
Katzmarski

\subsection{Benutzung} Die Nmap-Scripting-Engine (NSE), die seit
Version 4.21 in Nmap integriert ist [1], sorgt dafür, dass sich die
Funktionalität von Nmap durch Scripts erweitern lässt. Zum Zeitpunkt
der Erstellung dieses Textes kommt Nmap mit ca. 80 vorinstallierten
Scripts und insgesamt 153 stehen auf der Entwicklerhomepage zum
Herunterladen bereit[2].

Nmap-Scripts sind in der Skriptsprache LUA geschrieben. Bei LUA
handelt es sich um eine objektorientierte, Perl-ähnliche Sprache.  Das
bemerkenswerte an LUA ist die Größe des Interpreters, der nur 120 KB
groß ist.

Gängige Nmap-Scripts werden über das Argument -sC gestartet oder
gezielt mit dem Argument --script ausgeführt, das sich auf mehrere
Arten nutzen lässt. Man kann Dateinamen, Verzeichnis oder
Script-Kategorie angeben und die Scripts wählen die ausgeführt werden
sollen. Manche Scripts erwarten Parameter. Diese lassen sich mit dem
Argument --script-args setzen. Mehrere Parameter werden durch Komma
getrennt.

\subsection{Kategorien}

Scripts werden in Kategorien eingeteilt.

\begin{tabelle}{.5,.5}{Hochverfügbarkeitsklassen}{tab:comp:nse}
       \hline
         auth & Ermitteln von Authentisierungs-Credentials (z.B. Bruteforce) \\
       \hline
         default & Standard-Skripte, die beim Aufruf von -sC ausgeführt werden \\
       \hline
         discovery & Auswertung zugänglicher Dienste (z.B. Titel eines HTML-Dokuments oder SNMP-Einträge) \\
       \hline
         external & Skripte, die zwecks Weiterverarbeitung Daten an externe Dienste schicken (z.B. whois) \\
       \hline
         intrusive & Intrusive Skripte, die das Zielsystem (negativ) beeinträchtigen könnten (z.B. hohe CPU-Auslastung) \\
        \hline
          malware & Überprüfung der Infektion von Malware (Viren und Würmer) \\
        \hline
          safe & Defensive Skripte, die keine intrusiven und destriktiven Zugriffe durchführen \\
        \hline
          version & Erweiterung zum Fingerprinting mit dem Schalter -sV \\
        \hline
          vuln & Identifikation spezifischer Verwundbarkeiten (ähnlich einem Vulnerability Scanner) \\
        \hline
\end{tabelle}
Quelle: http://www.scip.ch/?labs.20100507

\subsection{Aufbau von NSE-Scripts}

 NSE-Scripts können in derivative und nicht-derivative Scripts geteilt
werden. Derivative Scripts benutzen nur bereits bestehende Information
von Nmap und bereiten diese auf.  Nicht-derivative führen zusätzliche
Netzzugriffe durch, um bestehende Informationen zu verifizieren, oder
neue zu gewinnen.

\subsubsection{Head} Im Head werden allgemeine Informationen über das
Script abgelegt.

\begin{lstlisting}[language={}]
  description = [[ Dieses minimale Skript identifiziert Webdienste ]]
  author = "Marc Ruef"
  license = "(c) 2010 by scip AG"
  categories = {"default", "safe"}
\end{lstlisting}

\subsubsection{Rules} Die Rules des Scripts entscheiden darüber, ob
ein Script ausgeführt wird oder nicht. Selbst, wenn das Script
explizit über --script ausgewählt wurde, muss auch die
Ausführungsregel gelten. Es muss mindestens eine der folgenden Regeln
vom Script überschrieben werden:

\begin{lstlisting}[language={}]
  prerule() 
  hostrule(host)
  portrule(host, port)
  postrule()
\end{lstlisting}
  
Am gängisten sind die Portrules. Die meisten Scripts entscheiden
anhand des Ports ob sie ausgeführt werden.


\subsubsection{Action} Die Action ist der Einstiegspunkt in das
Script. Von dort aus lässt sich das Script beliebig programmieren.

\subsection{Beispiele}

\begin{lstlisting}[language={}]
nmap -p80 -script=http-enum localhost
  PORT   STATE SERVICE
  80/tcp open  http
  | http-enum:  
  |   /admin/: Admin directory
  |   /icons/: Icons and images
  |   /phpmyadmin/: phpMyAdmin
  |_  /tmp/: tmp
\end{lstlisting}

\begin{lstlisting}[language={}]
nmap -p80 -script=http-php-version localhost
  PORT   STATE SERVICE
  80/tcp open  http
  | http-php-version: Versions from logo query 
  | (less accurate): 5.3.1RC3, 5.3.1
  | Versions from credits query (more accurate): 5.3.1RC3, 5.3.1
  |_Version from header x-powered-by: PHP/5.3.1
\end{lstlisting}

\begin{lstlisting}[language={}]
nmap -p 80 --script=http-form-brute 
--script-args 'http-form-brute.path=/dvwa/login.php' 127.0.0.1

  PORT   STATE SERVICE REASON  VERSION
  80/tcp open  http    syn-ack Apache httpd 2.2.14 ((Unix) 
  DAV/2 mod_ssl/2.2.14 OpenSSL/0.9.8l PHP/5.3.1 mod_apreq2-20090110/2.7.1 
  mod_perl/2.0.4 Perl/v5.10.1)
  | http-form-brute:  
  |   Accounts
  |     admin:password => Login correct
  |   Statistics
  |_    Perfomed 2011 guesses in 9 seconds, average tps: 223
  Final times for host: srtt: 333 rttvar: 2912  to: 100000
\end{lstlisting}


\begin{lstlisting}[language={}]
nmap --script smb-check-vulns.nse -p 445 192.168.178.40
Host is up (0.00017s latency).
PORT    STATE SERVICE
445/tcp open  microsoft-ds
MAC Address: 00:16:41:A7:39:D6 (USI)

Host script results:
| smb-check-vulns:  
|   MS08-067: LIKELY VULNERABLE (host stopped responding)
|   Conficker: UNKNOWN; not Windows, or Windows with disabled browser service 
|   (CLEAN); or Windows with crashed browser service (possibly INFECTED).
| |  If you know the remote system is Windows, try rebooting it and scanning
| |_ again. (Error NT_STATUS_OBJECT_NAME_NOT_FOUND)
|   regsvc DoS: CHECK DISABLED (add '--script-args=unsafe=1' to run)
|_  SMBv2 DoS (CVE-2009-3103): CHECK DISABLED 
|    (add '--script-args=unsafe=1' to run)
\end{lstlisting}

\subsection{Fazit}

NMap ist ein bewährtes Tool, um Informationen über ein Netz zu
gewinnen. Die Funktionalität von NMap lässt sich mit Scripts
erweitern, allerdings gibt es bis auf die Scripts auf der offiziellen
Entwicklerhomepage kaum fertige Scripts. Es würde einiges an Aufwand
kosten neue Scripts zu erstellen oder die vorhandenen anzupassen.

Es existieren auch Scripts, die Bruteforce-Angriffe, oder andere
Penetrationtests durchführen, was die Tendenz erahnen lässt NMap zu
einem Vulnerability-Scanner auszubauen. Allerdings kann der derzeitige
Script-Umfang mit anderen Tools, wie beispielsweise OpenVas nicht
mithalten.

Scripts könnten allerdings gut verwendet werden, um die gewonnen
Information aus NMap automatisiert weiterverarbeiten zu können.

\subsection{Material}

\begin{itemize}
  \item NSE Tutorial, abgerufen am 12.10.2010,
      URL: \url{http://www.scip.ch/?labs.20100507}
  \item Offizielle Nmap Webseite, abgerufen am 12.10.2010,
      URL: \url{http://nmap.org/nsedoc/}
  \item Blogeintrag favorisierte NSE-Scripte, abgerufen am 12.10.2010,
      URL: \url{http://www.attackvector.org/favorite-nmap-nse-scripts/}
	\item Wikipedia Artikel zu \textit{LUA}, abgerufen am 12.10.2010,
	    URL: \url{http://de.wikipedia.org/wiki/Lua}
  \item NASL CVE-Support, abgerufen am 15.10.2010,
      URL: \url{http://www.virtualblueness.net/nasl.html\#tth_sEc5.2.3}
\end{itemize}
 % BK: NMap
\clearpage\section{Wie und wofür können wir Metasploit nutzen?}
\label{compositions:msf}

Das \acr{msf} stellt ein Rahmenwerk zur Verfügung, so dass
sich Entwickler von Exploits auf den Kern der Entwicklung des Exploits
konzentrieren können. Hierfür stellt es verschiedene APIs zur Verfügung,
um z.B. mit einem SMB-Server zu kommunizieren.
Die so entwickelten Exploits stehen jedem Nutzer des \acr{msf} zur
Verfügung, so dass alle Exploits auf jeden Host angewendet werden
können. 
Auch für Exploits notwendige Payloads werden in verschiedenen
Varianten zur Verfügung gestellt. Alle Payloads sind, sofern
notwendig und sinnvoll, konfigurierbar und damit auf alle Bedürfnisse
anpassbar. So kann man bei einem Payload, das eine Shell auf dem remote
Host öffnet, einstellen, auf welchem Port diese Shell gestartet werden
soll. 
Das MSF bietet für jeden Sicherheitsexperten einen guten Rahmen, der
Es einem Sicherheitsexperten ermöglich die Sicherheit eines Netztes zu
überprüfen.

\subsection{Unsere Ziele}

Das Projekt FIDIUS hat zum zweiten Jahr hin sein Ziel geändert. Es ist
nun unser Ziel \gls{glos:ids} zu testen. Hierzu möchten wir
ein Programm entwickeln, dass automatisch versucht ein Netz zu
explorieren und die Schwachstellen der gefundenen Hosts
ausnutzt. Dieses Vorgehen soll protokolliert werden. So kann
vergleichen, was in einem Netz passiert ist und was davon verschiedene
\gls{glos:ids} erkannt haben.

\subsection{Was müssen wir können}

Um das Ziel verschiedene \gls{glos:ids} zu testen müssen wir
vornehmlich 5 Dinge können.
\begin{enumerate}
  \item Netze explorieren
  \item Versionen von auf Hosts installierten Diensten erkennen
  \item Sicherheitslücken der Dienste kennen oder herausfinden
  \item Exploits zu diesen Lücken finden und ausnutzen um die Kontrolle
    über einen Host zu übernehmen
  \item einen Host als Sprungbrett in andere Netze nutzen
\end{enumerate}

Diese 5 Teil-Ziele werden zum Großteil bereits von OpenSource-Tools
erreicht.
So kann \acr{nmap} bereits Netze explorieren und Versionen
von in einem Netz verfügbaren Dienste erkennen. \gls{glos:openvas}
kann Sicherheitslücken von Diensten nennen und die dazu gehörigen
CVE-Nummern anzeigen.
Exploits anwenden, um so die Kontrolle über einen Host zu erlangen
kann das \acr{msf} bereits semiautomatisch.

Das Benutzen eines Hosts als Sprungbrett in noch unbekannte Netze ist
ebenfalls bereits umgesetzt worden. Jeder Wurm besitzt diese
Fähigkeit.

\subsection{Was ist zu tun}

Die o.g. Programme können bereits einen Großteil von dem, was wir
können möchte. Dort wurden bereits wichtige Teilschritte unseres
Vorhabens umgesetzt. Es fehlen die Verbindungen zwischen den
Komponenten. Neben den \acr{api}s für Programme wie
\gls{acr:nmap}, \gls{glos:openvas} und \acr{msf} fehlt auch eine
zentrale Einheit, die das Zusammenspiel der Komponenten steuert und
das weitere Vorgehen (intelligent) plant.
Da das \acr{msf} zum einen sehr umfangreich ist und zum anderen
sehr viel, für unser Projekt besonders wichtige, Funktionalität
mitbringt, ist es vor allem zu Beginn wichtig, dass die Teilnehmer des
Projektes ein Verständnis für die Mächtigekeit und den Funktionsumfang
des \gls{acr:msf} entwickeln. 

\subsection{Was ist das Metasploit Framework}

Das \gls{acr:msf} ist wie eingangs erwähnt ein Framework, dass einen
Entwickler bei der Entwicklung von Exploits und anderen
Sicherheitsrelevanten Programmen unterstützt. Es besteht im
Wesentlichen aus drei Teilen:

\begin{enumerate}
  \item Ruby Extension Library (REX)\\
    Die REX Library ist eine Library, die darauf ausgelegt ist mit der
    Basis-Installation von Ruby zu funktionieren. Es sollen keine
    weiteren Gems oder Libraries zur Lauffähigkeit von Nöten sein. 
    Die REX Library beinhaltet vor allem 
    \begin{itemize}
      \item Wrapper für Sockets
      \item Implementierungen von Protokollen, z.B. SMB, FTP, HTTP
        etc.
      \item Das vom \acr{msf} genutzte Logging-Framework
      \item Utility-Klassen für Exploits 
      \item Diverse Helper Klassen
    \end{itemize}
  \item MSF-Core\\
    Das MSF-Core bildet den inneren Kern des Frameworks. Es stellt
    Interface-Klassen für alle Module bereit. Auch sind hier
    Manager-Klassen für die Verwaltung von Modulen, Sessions etc. abgelegt
  \item MSF-Base\\
    MSF-Base dient der Abstraktion vom MSF-Core. Es simplifiziert das
    API von MSF-Core und stellt zusätzliche Funktionen bereit. 
\end{enumerate}

Diese drei Teile stellen das Framework dar. Sie enthalten jedoch weder
Exploits noch Payloads. Diese sind in Modulen unterschieden die
seperat abgelegt sind.

\subsubsection{Module}

Module sind Klassen, die das Framework nutzen und vom Framework
benutzt werden. Es werden fünf verschiedene Module unterschieden. 

\begin{enumerate}
  \item Exploits\\
    Um eine Sicherheitslücke auszunutzen, benötigt man ein Programm,
    dass eine Schwachstelle eines anderen Programmes ausnutzt. Diese
    Programme heißen Exploits. 
  \item Payloads\\
    Nachdem man eine Sicherheitslücke ausgenutzt hat, muss ein
    weiteres Programm gestartet werden, dass einem Zugriff auf das
    System gewährt oder weiteren Code nachlädt, um das angegriffene
    System unter seine Kontrolle zu bringen. Diese Programme heißen
    Payloads.  
  \item Encoders\\
    Programme müssen auf verschiedenen Rechner-Architekturen den
    Prozessor unterschiedlich ansprechen. Damit ein Programm auf
    nahezu jeder Architektur ausführbar ist, müssen sie enkodiert
    werden. Diese Aufgabe übernehmen Encoder.
  \item Nops\\
    Ein Prozessor beendet die Verarbeitung eines Befehles,
    sobald ein \emph{\\0} darin vorkommt. Daher muss ein
    für die Zielplattform geeigneter Ersatz genutzt werden, trotz des
    notwendigen Zeichens \emph{\\0} keinen Abbruch der Verarbeitung
    des eigenen Programmes zu erreichen. Dies sind die Nops. Sie
    führen keine Operation durch (No Operations) 
  \item Auxiliaries\\
    Es gibt Programme, die die Sicherheit anderer Programme testen,
    ohne sie direkt zu beeinflussen. Ein Programm könnte z.B. den
    Netzverkehr mithöhren und Passwörter, die unverschlüsselt über das
    Netz gesendet werden, mitschneiden. Ein solchen Programm ist kein
    Exploit im eigentlichen Sinne, wie er oben beschrieben worden
    ist. Dennoch erfüllt ein solches Programm eine ähnliche
    Funktion. Solche Programme werden Auxiliaries (Aux) genannt. 
\end{enumerate}

Metasploit unterscheidet Module nach diesen fünf Typen.

\subsubsection{Unterscheidung von Payloads}
\label{compositions:msf:payloads}
Die Payloads untereinander lassen sich in weitere Gruppen
unterteilen. 

Die erste Gruppe der Payloads sind die sog. \emph{Singles}. Sie führen
genau eine Aktion aus. So wird durch ihre Ausführung z.B. ein neuer
Benutzer dem System hinzugefügt oder eine bestimmte Datei ausführt. 
Die zweite Gruppe sind die \emph{Stagers}. Sie dienen als Wegbereiter
für größere Programme. \emph{Stagers} Laden in der Regel weiteren Code
nach, da der Code, den man einem Exploit mit auf den Weg geben kann
meist eine bestimmte Größe nicht übersteigen darf. 
\emph{Stages} bilden die dritte Gruppe der Payloads. Sie werden durch
\emph{Stagers} nachgeladen und installiert. 

\subsubsection{Ordnung im Dateisystem}

Um eine gewisse Übersichtlichkeit zu bewahren hat das \acr{msf}
eine Struktur im Datei-System, die im Folgenden kurz erläutert
wird. Die u.g. Pfade sind relativ zum Installationsverzeichnis.

\begin{itemize}
  \item lib\\
    Hier befindet sich das Framework selbst. Unter anderem die
    REX-Library und der Code des MSF-Core und MSF-Base.
  \item data\\
    In diesem Verzeichnis werden Dateien abgelegt, die der Benutzer an
    seine Bedürfnisse anpassen soll. Z.B. Wörterbüchcher zum Erraten
    von Login-Daten. 
  \item tools\\
    Im Ordner Tools sind verschiedene Hilfsprogramme zu finden. So
    z.B. eines, das exe-Dateien in VBS-Dateien umwandelt
  \item modules\\
    Dieser Ordner beinhaltet, wie der Name schon andeutet, alle
    Module. Also alle Exploits, Payloads etc. 
  \item plugins\\
    Plugins sind z.B. Klassen, die die Anbindung an eine Datenbank
    abstrahieren. Sie werden in diesem Ordner abgelegt.
  \item scripts\\
    Im Verzeichnis Scripts befindet sich unter anderem ein Teil des
    Meterpreters. 
  \item external\\
    Hier können Programme von Drittanbietern abgelegt werden.
\end{itemize}

Neben den o.g. Ordnern befinden sich im root-Verzeichnis der
\gls{acr:msf}-Installation noch ein Ordner mit Tests und ein weiterer
mit Dokumentation. Zusätzlich lassen sich noch diverse Binärdateien
finden, deren Beschreibung über den Rahmen dieser Ausarbeitung hinaus
geht. 

\subsubsection{Module im Dateisystem}

Die Module des \acr{msf} sind in einem eigenen Ordner abgelegt und
nach ihrer Funktion aufgeteilt. So befinden sich alle Payloads, alle
Exploits etc. in eigenen Ordnern. 

Innerhalb eines jeden Ordners für jeden Modul-Typ befindet sich eine
weitere Ebene, die die Exploits, Nops etc nach z.B. Betriebssystem
oder Architektur-Typ unterscheidet.

Für eigene Module ist der Ordner \emph{.msf} im Homeverzeichnis
vorgesehen. Module die hier abgelegt werden, werden beim Laden des
\acr{msf} berücksichtig. 

\subsection{APIs des MSF}

Das \acr{msf} bietet verschiedene \glspl{acr:api}. Je nach Ziel ist
aufgrund der etwas unterschiedlichen Funktionalität ein anderes
\gls{acr:api} geeignet.

Im Folgenden werde ich vier Methoden vorstellen, mit denen man das \acr{msf}
anzusprechen kann. Hierbei werde ich auf die Vor- und Nachteile
eingehen und die Verwendbarkeit des \acr{api} für FIDIUS besprechen.

\subsubsection{RPC}

Im Root-Verzeichnis der \acr{msf}-Installation befinden sich die
beiden Dateien \emph{msfrpc} und \emph{msfrpcd}. Sie starten den
RPC-Client bzw. den RPC-Daemon. 
Über RPCs lassen sich diverse Funktionen des \acr{msf} nutzen. 
Jedoch müssten wir entweder den RPC-Client erweitern oder über die
Ausführung der Binärdatei nutzen. 

Nachteile:
\begin{itemize}
  \item kein Zugriff auf volle Funktionalität des \acr{msf}. 
  \item Anfällig für MITM-Angriffe
\end{itemize}

Vorteile: 
\begin{itemize}
  \item über Rechnernetze nutzbar
  \item einfache, klare Schnittstelle
\end{itemize}

Als wesentlicher Nachteil ist hier zu nennen, dass der Zugriff auf die
Funktionalität des \acr{msf} beschränkt ist, so dass zur Beurteilung
der Verwendbarkeit dieser \acr{api} die Analyse des genauen
Funktionsumfanges notwendig wäre.

\subsubsection{msfcli}

Die ausfühbare Datei \emph{msfcli} im Root-Verzeichnis der
\acr{msf}-Installation erwartet als Parameter ein Exploit und die für
die Ausführung des Exploits notwendigen Parameter folgend.
Auch hier müsste, damit wir im Projekt FIDIUS diese API nutzen können,
die binäre Datei angesprochen werden. 

Nachteile:
\begin{itemize}
  \item keine Suche von Exploits möglich 
  \item keine Abfrage der notwendigen Parameter möglich
  \item nicht über Rechnernetze nutzbar
\end{itemize}

Vorteile: 
\begin{itemize}
  \item klare, unkomplizierte Schnittstelle
\end{itemize}

Sollten wir \emph{msfcli} als Schnittstelle zum \acr{msf} nutzen, so
würden wir Probleme erneut lösen müssen, die im \acr{msf} bereits
gelöst sind. Über den \emph{msfcli} können wir z.B. keine Suche nach
Exploits durchfürhen oder Exploits nach bestimmten Kriterien auswählen.

\subsubsection{RC-Scripte}

RC-Scripte sind Dateien in denen die Befehle, die das \acr{msf} ausführen
soll nacheinander auflistet sind. Durch Tags (<ruby> und </ruby>)
lässt sich Ruby-Code einbetten. 
Stöße man jedoch über ein RC-Script eine Suche nach einem Expoit an,
so müsste man das Ergebnis parsieren und darauf erneut ein RC-Script
erzeugen. Dies bedeutete, dass das Framework mehrfach geladen werden
müsste. Da beim Laden des Frameworks unter anderen alle Dateien im
Verzeichnis modules/ eingelesen, parsiert und in den Arbeitsspeicher
geladen werden, ist dieser Vorgang sehr aufwändig und benötigt im
Verhältnis zu anderen Operationen des Frameworks sehr große Zeit.

Nachteile:
\begin{itemize}
  \item Ergebnisse müssten parsiert werden und neue RC-Scripte 
  \item evtl. mehrfaches Laden des Frameworks notwendig. 
\end{itemize}

Vorteile: 
\begin{itemize}
  \item mögliche Einbettung von Ruby-Code
\end{itemize}

Das \acr{msf} löst viele Probleme, denen wir uns gegenüber gestellt
sehen. Wir müssten, um diese Schnittstelle zum \acr{msf} zu nutzen,
sehr komplexe RC-Files schreiben. Diese zu entwickeln und auch zu
pflegen scheint ein großer Aufwand zu sein. Daher kommt auch diese
Schnittselle tendenziell nicht in Frage.

\subsubsection{Ruby}

Das \acr{msf} erlaubt dem Benutzer, durch das Einbinden der
entsprechenden Dateien, das Framework direkt aus Ruby-Code zu laden
und zu benutzen. 
Durch den folgenden Code z.B. lassen sich alle im \acr{msf} befindlichen
Module auflisten:

\begin{verbatim}
  $:.unshift(PATH\_TO\_MSF\_LIB)
  require "msf/base"

  f = Msf::Simple::Framework.create

  f.modules.each_modules do |name, module|
    puts name
  end
\end{verbatim}

Zum Durchlaufen aller im Framework verfügbares Exploits muss man nur
f.exploits.each do |e| puts e.fullname end schreiben. 
Auch zum Ausführen von Exploits, dem Suchen nach Payloads und dem
Setzen aller notwendigen Optionen gibt es genügend Dokumentation.

Nachteile:
\begin{itemize}
  \item tiefer Einstieg in den Quellcode des \acr{msf} notwendig
  \item nicht ohne Erweiterungen über Rechnernetze nutzbar
\end{itemize}

Vorteile: 
\begin{itemize}
  \item Verfügbarkeit des gesamten Frameworks
  \item Community, die Erweiterungen evtl. übernimmt und weiter pflegt.
\end{itemize}

Die direkte Ansprche des \acr{msf} über Ruby scheint bei dem großen Umfang
der Funktionalität, die wir vom \acr{msf} nutzen möchten der
geeignetste Weg zu sein. Trotz der Notwendigkeit der Entwicklung eines
Verständnisses des Quellcodes des \acr{msf}.

\subsection{Ausblick}

Für unser Vorhaben müssen wir Zugriff auf alle Funktionen des
\acr{msf} haben. Hierzu ist der Zugriff direkt über Ruby am besten
geeignet, da wir so auch die Funktionen des \acr{msf} die wir benutzen
wollen an unsere Bedürfniss anpassen können. 
Z.B. ist das automatisierte Angreifen von allen Hosts in einem
bestimmten Netz nicht besonders ausgereift. So werden immer alle
Hosts, die in der Datenbank erfasst sind angegriffen und nicht nur ein
bestimmtes Subnet. Auch werden für jeden Hosts alle Exploits in
zufälliger Reihenfolge verwendet. Dies könnte man optimieren, in dem
man im Quellcode des Frameworks die entsprechenden Stellen im Code
optimiert und anpasst. Darüber hinaus schränkten die anderen Optionen
unsere Möglichkeiten im Arbeiten mit dem \acr{msf} ein. Nur durch die
direkte Verwendung der vom \acr{msf} bereitgestellen Libraries steht
uns der volle Funktionsumfang des Frameworks zur Verfügung.
 % DE
\clearpage\section{Metasploits Meterpreter}
\label{compositions:meterpreter}
\authors{\HM}{}

\subsection{Einleitung}

Meterpreter, Kurzform für Meta-Interpreter, ist ein hochentwickelter
Payload für Metasploit \cite{skape-meterpreter}. Metasploit ist ein
erweiterbares Framework zum Penetration Testing, das vielen Exploits
und damit kombinierbaren Payloads mitliefert (siehe~\ref{compositions:msf}).
Meterpreter ist ein Payload der von einem Exploit in Metasploit
ausgeführt werden kann. Es ist mit vielen Exploits von Metasploit
kompatibel und lässt sich anstatt einer Systemshell wie z.B.
\texttt{/bin/bash} oder \texttt{cmd.exe} wählen, die Bedienung von
Meterpreter ist für den Angreifer ähnlich zu einer Systemshell.

\subsection{Probleme von System Shells}
\label{compositions:meterpreter:systemshell}

Wenn ein System angegriffen wird soll dieses normalerweise sehr leise
geschehen, damit installierte Sicherheitssysteme wie z.B. \acr{nids},
\acr{hids} und Antivirenprogramme dieses nicht bemerken. Die Benutzung
von Systemshells als Payload von Exploits bereitet bei Angriffen auf
Systemen, wenn mehr als nur die Angreifbarkeit durch einen bestimmten
Exploit gezeigt werden soll, Probleme.

Wenn die Systemshell als Payload eines Exploits von diesem gestartet
wird, muss dazu ein neuer Prozess erzeugt werden oder der alte Prozess
komplett überschrieben werden z.B. mithilfe von \texttt{exec()}.
Dieses Verhalten ist sehr auffällig, vor allem wenn ein Programm welches
sonst nie eine Shell starten würde dieses auf einmal tut.
Viele \acr{hids} und Antivirenprogramme haben eine Heuristik, um die
Wahrscheinlichkeit für einen Angriff zu ermitteln. Ein Aufruf von
\texttt{exec()} oder ähnlichen Funktionen geht immer über einen
Systemcall in den Kernel und lässt sich dort sehr gut abfangen und trägt
dann stark dazu bei die Aktion als Angriff zu bewerten.

Vor allem in Unix Systemen werden Prozesse in \texttt{chroot}-Umgebungen
ausgeführt und auf Desktopsystemen werden immer mehr Client Programme,
wie Internet Browser, innerhalb einer Sandbox ausgeführt. In diesen
Umgebungen ist nicht immer ein Zugriff auf die Systemshell möglich,
z.B. weil der Prozess, der in der \texttt{chroot} läuft diese nicht
benötigt.

Für bestimmte Aktionen die nach dem Angriff eines System ausgeführt
werden sollen, wie z.B. das Sammeln von Informationen werden zusätzliche
Tools benötigt. Diese Tools müssen manuell nachgeladen werden, wenn sie
noch nicht auf dem angegriffenen System installiert sind, was
zusätzliche, auffällige Aktionen verursacht. Diese Programme würden z.B.
über HTTP auf die Festplatte des angegriffenen Systems geladen und dann
mithilfe der Shell ausgeführt. Daten die auf die Festplatte geschrieben
werden, werden aber normalerweise vorher vom Antivierenprogramm
analysiert und der Schreibvorgang wird unterbunden, falls er als Teil
eines Angriffs erkannt wird. Lese- und Schreiboperation die sich nur im
Arbeitsspeicher abspielen werden von Antivirenprogrammen meistens nicht
untersucht, da dieses viel schwieriger zu implementieren wäre und die
Systemperformance zu stark beeinträchtigen würde. Die Überprüfung von
Lese- und Schreiboperationen auf Festplatten stellt kein großes
Performance-Problem da, weil dieses Medium im Gegensatz zum
Arbeitsspeicher sehr langsam ist und deswegen auch nicht so viele
Operationen darauf ausgeführt werden.

Eine Systemshell wie \texttt{cmd.exe} bietet auch oft nicht den vom
Angreifer bevorzugten Bedienkomfort oder ist nicht so einfach skriptbar.
Desweiteren hat jede Systemplattform eine etwas andere Shell, wobei sich
Skripte, die für die Windows-Shell \texttt{cmd.exe} geschrieben wurde
nicht auf Unix-Systemen mit einer Bash nutzen lasen und umgekehrt. Damit
müssten Skripte, die bestimmte Aktionen nach einem Angriff ausführen
zumindest für die beiden großen Plattformen separat erstellt und gepflegt
werden.

Ein Vorteil von Systemshells ist, dass sie schon auf dem angegriffenen
System installiert sind und damit nicht extra übertragen werden müssen,
was Bandbreite spart und vielleicht auch dem \acr{nids} weniger
Möglichkeiten zum Erkennen gibt. Teilweise darf eine Payload bei einem
Exploit eine bestimmte Größe nicht überschreiten, was für den Aufruf der
Systemshell meistens kein Problem darstellt, aber bei eigener Software,
die größer ist Probleme bereiten kann.

\subsection{Meterpreter}
\label{compositions:meterpreter:meterpreter}

Meterpreter versucht fast alle die Probleme, die bei der Benutzung von
Systemshells auftreten zu beheben. Dazu ist Meterpreter so gebaut, dass
es zur Laufzeit weiteren Code nachladen und ausführen kann, je nachdem
was gerade benötigt wird. Dadurch muss zu Anfang nur ein kleines
Programm übertragen werden welches dann die restlichen Funktionen
nachlädt.

Der Meterpreter ist in zwei Teile aufgeteilt: Der Server, der in C
geschrieben ist, läuft auf dem angegriffenen Host und der Client, der in
Ruby geschrieben ist, läuft beim Angreifer. Diese beiden Komponenten
kommunizieren miteinander über ein speziell für den Meterpreter
entwickeltes Protokoll, welches über verschiedene standardisierte
Protokolle wie TCP, UDP und HTTP gelegt werden kann.

Der Angreifer kann den Meterpreter über eine Shell bedienen, die auch
\emph{tab completion} und eine \emph{command history} bietet. Meterpreter
ist mithilfe von Ruby komplett skriptbar. Es gibt viele Skripte, die
schon mit Metasploit ausgeliefert werden und nur gestartet werden müssen,
es ist aber auch relativ einfach eigene Skripte zu schreiben um noch
fehlende Funktionen nachzurüsten.

\subsection{Funktionen}
\label{compositions:meterpreter:funktionen}
Meterpreter bringt schon viele nützliche Funktionen mit, die man bei
einer Systemshell nur mithilfe von externe Programmen erreichen kann.
Es ist möglich Dateien vom angegriffenen System runter- und auf das
System hochzuladen, direkt über die Meterpreter Verbindung. So muss keine
Extra-Verbindung aufgebaut werden oder ein Extra-Programm wie wget
installiert werden.

Mithilfe von Port-Forwarding können einzelne Port vom System des
Angreifers auf den angegriffenen Host geleitet werden. Dieses bietet
ähnliche Funktionen wie das SSH-Port-forwarding. Die Daten werden über
die Meterpreter Verbindung geleitet und können wie die anderen Daten
verschlüsselt werden. Es ist auch möglich einen Socks4a Proxy ähnlich zu
\cite{rfc-1928} über diese Verbindung laufen zu lassen. Weitere Angriffe
aus Metasploit können auch über diese Verbindung geleitet werden indem
in der Metasploit Konsole mithilfe von route alle Pakete, die an Hosts
in einem bestimmten Subnetz gerichtet sind, über diese Verbindung
geroutet werden. Zum Betrachten und Verändern der Routingtabelle kann
der Befehl \texttt{route} verwendet werden.

Zum Abhören von Netzverkehr kann die Meterpreter Sniffer-Erweiterung
verwendet werden. Diese Erweiterung verwendet die libpcap\footnote{%
\url{http://www.tcpdump.org/}} und sendet alle abgefangenen Pakete an
den Angreifer, damit dieser sie auswerten kann und weitere Angriffe
planen und ausführen kann.

Die Tastatureingaben des Nutzers können mithilfe von \texttt{keyscan}
abgefangen werden und mit \texttt{webcam} ist es sogar möglich, wenn
eine Webcam an dem Rechner vorhanden ist, mit dieser Bilder zu machen.
Zum verwischen der Spuren kann \verb!clearev! verwendet werden.
Dieses Tool löscht den Windows Event Log.

Die Erweiterung Railgun eignet sich dafür, um mit Ruby-Skripten
Funktionen auf Windows System Bibliotheken aufzurufen. Es gibt schon für
folgende DLLs vorgefertigte Definitionen zum Aufrufen von Funktionen:
\textit{advapi32.dll}, \textit{iphlpapi.dll}, \textit{kernel32.dll},
\textit{ntdll.dll}, \textit{shell32.dll}, \textit{user32.dll} und
\textit{ws2\_32.dll}. Damit ist es dann z.B. möglich ein Fenster auf dem
Desktop anzuzeigen oder Komplexere Aktionen durchzuführen.

Der übertragene Programmcode von Meterpreter kann auch mithilfe von
Encodern versteckt werden, damit die Signaturerkennung bei \acr{nids},
\acr{hids} und Antivirenprogramme erschwert wird. Dazu wird der
eigentliche Programmcode des Meterpreters verschlüsselt und an den
Anfang wird ein Decoder mit den Schlüssel zum Entschlüsseln geschrieben.
Wenn Meterpreter gestartet wird, wird zuerst der Decoder gestartet der
mit dem mitgelieferten Schlüssel den Meterpreter entschlüsselt und
startet. Zur Verschlüsselung reicht oft schon eine XOR Verknüpfung, da
der Schlüssel sowieso mitgeliefert wird und der Code nur jedes mal immer
anders aussehen soll. Ein Decoder für eine XOR-Verschlüsselung ist auch
sehr einfach und klein zu implementierten.

\subsection{Meterpreter Skripte}
\label{compositions:meterpreter:meterpreter-skripte}

Die Meterpreter Skripte laufen auf dem Client, dem System des Angreifers,
und sagen dem Server welche Operationen er ausführen soll. Der Server,
der auf dem angegriffenen Host läuft, ist nicht direkt skriptbar, sondern
nur indirekt über den Client. Es muss also immer eine Verbindung zwischen
Client und Server bestehen während die Skripte ausgeführt werden.

Meterpreter bringt schon einige Skripte mit, die nützliche Aufgaben
erledigen.

\begin{longtable}{lp{.7\linewidth}}
  \rowcolor{Beige}
    Name & Beschreibung \\
  \endhead
    \caption[]{Meterpreter Skripte\\\tabelletbcname}
  \endfoot
    \caption{Meterpreter Skripte\label{tab:compositions:meterpreter-scripts}}
  \endlastfoot
  getgui & Windows Remote Desktop aktivieren und Benutzer dafür anlegen \\
  hashdump & Windows Passwort Hashes anzeigen \\
  arp\_scanner & ARP Scanner \\
  enum\_firefox & Firefox Einstellungen (History, Passwörter, ... )
    kopieren (gibt es auch für andere Browser)\\
  checkvm & Überprüft, ob es sich um eine VM handelt \\
  getcountermeasure & Überprüft welche Schutzmechanismen das System hat \\
  killav & Versucht die aktiven Antivirenprogramme auszuschalten \\
  hostsedit & Bearbeiten der /etc/hosts Datei
\end{longtable} 

\subsection{Architektur}

\subsubsection{Anforderungen}

Um die Probleme die bei der Benutzung von Systemshells auftreten
(siehe~\ref{compositions:meterpreter:systemshell}) zu umgehen, wurden
beim Entwurf bestimmte Anforderungen an die Architektur des Meterpreter
gestellt \cite{skape-meterpreter}. Es muss kein neuer Prozess gestartet
werden, um Meterpreter zu starten, sondern Meterpreter muss in dem
Prozess in dem der Exploit ausgeführt wurde weiter laufen können ohne
den kompletten Inhalt auszutauschen. Dadurch kann sich Meterpreter
besser vor \acr{hids} und Antivirenprogrammen verstecken. Zusätzlich
soll es auch in einer \texttt{chroot}-Umgebung laufen können und keine
speziellen Anforderungen an installierte Programme auf dem System stellen.
Meterpreter soll sich zusätzlich dynamisch zur Laufzeit erweitern lassen.

\subsubsection{Aufbau}

Wie schon in~\ref{compositions:meterpreter:meterpreter} beschrieben, ist
der Meterpreter eine Client-Server Anwendung. Die Kommunikation der
beiden Teile miteinander findet über den vom Stager aufgebauten Socket
statt. Ein Stager ist ein kleines Programm welches direkt an den Exploit
als Payload angehängt wird. Oft können direkt mit dem Exploit nur sehr
kleine Payloads mit übertragen werden, weil der Exploit nicht viel Platz
für den Payload bietet. Um trotzdem größere Anwendungen zu verwenden
wird direkt im Payload des Exploits nur ein kleiner Stager mitgeliefert,
der dann später den eigentlichen Payload, z.B. Meterpreter nachlädt und
startet.

Dabei baut der Stager schon eine Socket-Verbindung auf, die Meterpreter
dann später für seine Kommunikation weiterbenutzt. Es gibt Stagers für
verschiedene Protokolle, unter anderem für (TCP und UDP jeweils
\textit{reverse}\footnote{angegriffener Host baut Verbindung zu
Angreifer auf} und \textit{bind}\footnote{angegriffener Host öffnet Port
und Angreifer verbindet sich}). Zusätzlich gibt es auch einen Stager,
der eine HTTP- oder HTTPS-Verbindung aufbauen kann und somit die
Kommunikation gut tarnen kann. Eine Verbindung über TCP, UDP und HTTP
kann mithilfe von SSL verschlüsselt werden, wobei die Validität der
Zertifikate nicht überprüft wird. Die Überprüfung der Zertifikate würde
die Sicherheit nicht stark erhöhen, da der private Schlüssel des
Meterpreter-Servers und das Zertifikat des Clients mit dem Exploit
übertragen werden müssten und diese Daten abhängig vom Exploit auch
schon unverschlüsselt sind und manipuliert werden können.

\paragraph{Server}

Der Server läuft auf dem angegriffenen Host und ist in der
Programmiersprache C geschrieben, zum Kompilieren wird Microsoft Visual
Studio 2008 oder 2010 benötigt. Da der Meterpreter hauptsächlich dazu
entwickelt wird Windows-Systeme anzugreifen, stehen auch nur unter
diesem System alle Funktionen zur Verfügung. Zusätzlich wird auch
Linux unterstützt, für Mac OS X gibt es noch keinen offiziellen Server,
es gibt aber schon einige Anfänge dazu, wobei diese sich eher an das
iPhone richten, welches sehr ähnlich zu Mac OS X ist\dots

Für PHP-Webanwendungen gibt es auch einen Meterpreter, der in PHP
geschrieben ist und sich als Payload für Exploits von PHP -Anwendungen
eignet. Für Java-Applets und andere Java-Anwendungen gibt es auch noch
eine spezielle Java-Version des Meterpreters.

Diese verschiedenen Versionen von Meterpreter unter anderem für Java und
PHP werden benötigt, da Meterpreter im Kontext des angegriffenen
Prozesses laufen soll. Zum Starten von Meterpreter wird
Remote-Library-Injection verwendet, also der Meterpreter-Code wird in
den laufenden Prozess geschrieben und dann ausgeführt. Deswegen müssen
keine Daten auf die Festplatte geschrieben werden und es muss auch kein
neuer Prozess erstellt werden, was Schutzprogramme alarmieren könnte.
Nachdem der Meterpreter in den Prozess geladen wurde funktioniert der
ursprüngliche Prozess meistens  nicht mehr, dieses hängt aber vom
Exploit ab.

Der Meterpreter-Server kann auch, nachdem er in einem Prozess gestartet
wurde, in einen anderen Prozess wechseln. Das nennt sich Migration und
ist z.B. Sinnvoll, wenn der angegriffene Prozess beendet werden könnte
oder der Angreifer für bestimmte Aktionen in einem anderen Kontext
arbeiten möchte. Um die Tastatureingaben abzufangen muss sich der
Meterpreter in einem Prozess befinden, der unter dem Nutzer läuft,
dessen Eingaben abgefangen werden sollen.

\paragraph{Client}

Der Meterpreter-Client ist in Ruby geschrieben und läuft auf dem System
des Angreifers. Der Benutzer kann diesen über eine Shell steuern, der
Client sendet dann die entsprechenden Befehle an den Server. Mit Ruby
ist der komplette Client skriptbar und es steht auch eine IRb zur
Verfügung.

\subsubsection{Netzschnittelle}

Meterpreter benutzt ein komplexes System zur Kommunikation zwischen dem
Client und dem Server. Es steht normalerweise nur ein Socket für die
Kommunikation zur Verfügung und es sollen nicht beliebig viele
Systemsockets geöffnet werden, damit Meterpreter nicht so einfach
entdeckt wird. Deswegen können Kanäle definiert werden die sich ähnlich
wie unterschiedliche Ports verhalten. Über diese Kanäle werden Pakete,
die nach dem \glos{tlv} Verfahren verpackt wurden, geschickt. Es gibt
einige vordefinierte Typen für die auch Parser mitgebracht werden,
zusätzlich können Erweiterungen auch eigene Typen definieren. Die
standardmäßig mitgelieferten Parser können die Nullterminierung bei
Strings überprüfen und Integer von der Network-\glos{byteorder} in die
Host-\glos{byteorder} umwandeln. Diese Pakete werden normalerweise
verschachtelt, sodass jede Schicht die Pakete wieder nach dem
\glos{tlv}-Verfahren einpackt. Meterpreter Erweiterungen können diese
Schnittstelle nutzen, Meterpreter Skript müssen sich um diese
Schnittstelle nicht kümmern, da diese eine Schicht weiter oben laufen.

\subsection{Ablauf eines Angriffs}

Ein Angriff auf ein System, bei dem Meterpreter als Exploit verwendet
wird, läuft in mehreren Schritten ab. Wenn der Exploit erfolgreich
ausgeführt wurde, führt er den mit eingepackten Payload aus. Dieser
Payload ist meistens ein Stager, der relativ klein ist und nur den
richtigen Payload wie z.B. Meterpreter nachladen soll. Der Stager baut
dann eine Verbindung zu Metasploit auf und lädt die \textit{.dll} oder
\textit{.so} des Meterpreters über das Netz nach und legt sie im Speicher
des angegriffenen Prozesses ab und startet ihn. Der soeben gestartete
Meterpreter baut, wenn er sich \textit{reverse} verbinden soll, eine
Verbindung zum Angreifer über den Socket des Stagers auf. Nach dem
Verbindungsaufbau wird der Meterpreter-Server vom Client konfiguriert
und es wird automatisch die \texttt{stdapi}-Erweiterung nachgeladen.
Wenn Meterpreter Administratorenrechte auf dem angegriffenen System hat,
wird auch noch die \texttt{priv}-Erweiterung nachgeladen
\cite{offensive-security-meterpreter}.

\subsection{Meterpreter Skripte schreiben}

Meterpreter lässt sich mithilfe von Ruby Skripten. Metasploit liefert
schon viele nützliche Skripte mit (siehe~\ref{compositions:meterpreter:meterpreter-skripte})
die nur noch ausgeführt werden müssen, es ist aber auch relativ einfach
möglich eigene Skripte zu schreiben. Während der Entwicklung biete es
sich an den integrierten IRb zu benutzen. Dieser verhält sich wie der
normale Ruby-IRb, es gibt aber die Variable \texttt{client}, die das
Objekt dieser Meterpreter Session enthält. Mithilfe von Methoden auf
dieser Variable können nun Operationen auf dem angegriffenen Host
durchgeführt werden.

Folgender Codeschnippel gibt z.B. den Namen des Betriebssystems aus:
\begin{lstlisting}[language=bash]
  client.sys.config.sysinfo['OS']
  => "Windows XP (Build 2600, Service Pack 3)."
\end{lstlisting}

Mithilfe von Railgun lassen sich Funktionen auf der Windows-API aufrufen.
Um ein Fenster Anzuzeigen wird nur folgender Code benötigt:

\begin{lstlisting}[language=bash]
  client.railgun.user32.MessageBoxA(0,"Hello","world","MB_OK")
\end{lstlisting}

Wenn man komplexe Skripte erstellt hat, dann können diese im Verzeichnis
\path{/scripts/meterpreter/} abgelegt werden und via \texttt{run <Skriptname>
<Parameter>} gestartet werden.


\subsection{Nutzen für \f}

Mithilfe des Meterpreters können nach dem erfolgreichen Angriff eines
Host noch sehr viele Operationen ausgeführt werden. Meterpreter eignet
sich auch sehr gut dazu von einem schon übernommenen Host aus weitere
Ziel anzugreifen und ihn als Brückenkopf zu verwenden oder um
Informationen für weitere Angriffe auf solch einem Host zu sammeln.
Es lassen sich auch viele Aufgaben über Skripte automatisieren was uns
beim Ziel eines sich automatisch weiterverbreitenden Wurm weiterhelfen
kann. Nachteilig ist, dass der Meterpreter Server nicht autark läuft und
immer auf eine Verbindung zu einem Meterpreter-Client angewiesen ist, um
Operationen durchzuführen. Die Nutzung von Meterpreter setzt auch die
Nutzung von Metasploit voraus, dieses stellt aber auch kein großes
Problem dar.
 % HM

% Sonntag
\clearpage\section{Würmer}
\label{compositions:wuermer}

\subsection{Einführung}

\subsubsection {Definition}

Ein Computerwurm bezeichnet ein Programm, dass vorhandene
Schwachstellen im Betriebssystem oder in Anwendungen, aber auch oft
die Naivität und Unwissenheit der Nutzer ausnutzt, um sich
unberechtigt auf dem Zielsystem Zugang zu Ressourcen zu
verschaffen. Nach einer erfolgreichen Infiltration durchsucht der Wurm
auf dem Zielsystem bestimmte Dateien nach neuen Hosts, IP- oder
Emailadressen, also neuen Opfern, um diese anschließend mit bereits
bestehenden bzw. vom Wurm \textit{mitgebrachten} Netzwerkdienste
anzugreifen.

Im RFC 4949 wird ein Wurm wie folgt definiert:

\begin{quote} \itshape \foreignquote{english}{ computer program that
can run independently, can propagate a complete working version of
itself onto other hosts on a network, and may consume system resources
destructively.}  ~\cite{wuermer-rfc4949}
\end{quote}

Würmer werden der Kategorie der Maleware zugeordnet, da Würmer
unerwünschte Aktionen auf dem Zielsystem ausführen, verdeckt
operieren, Daten stehlen, Ressourcen verbrauchen und oftmals auch
absichtlich Schaden verursachen. Ein Wurm ist einem Virus in vielen
Belangen (Ziele, Verbreitungswege, Eigenschaften) sehr ähnlich, wobei
es auch essentielle Unterschiede gibt: Viren sind lediglich
Code-Fragmente, die sich an andere Daten anhängen und sich nur bei
deren Ausführung oder Verarbeitung vermehren
~\cite{wuermer-viren}. Würmer sind hingegen komplexer und verbreiten
sich selbst.

\subsubsection {Analogie im Tierreich}

Zwischen den Pedanten im Tierreich gibt es eine ähnliche
Relation. Viren sind kleine Krankheitserreger, bestehend aus
Erbmaterial, das von einer schützenden Hülle aus Fetten oder Eiweißen
umgeben ist. Sie besitzen keinen eigenen Stoffwechsel und können sich
nur innerhalb lebender Wirtszellen (Analogie: Wirtsdatei)
vermehren. ~\cite{wuermer-bioviren}.  Im Gegensatz dazu sind
parasitäre Würmer höherentwickelte und selbständige Lebewesen
(\textit{wobei es in der Wissenschaft unterschiedliche Definition von
Leben gibt} ~\cite{wuermer-leben}), die sich in einem Wirt einnisten,
sich von diesem ernähren und Eier legen. Die Eier werden über die
Fäkalien ausgeschieden und verbreiten sich dadurch.

Eine direkte Übereinstimmung liegt natürlich nicht vor, da biologische
Organismen ungleich komplexer sind. Doch gewisse Analogien lassen sich
ableiten: Z.B. sind aggressive Viren (z.B. der Ebolavirus) nicht sehr
verbreitet, da die Inkubationszeit gering ist und die Sterberate zu
hoch ist, wodurch potenzielle Überträger schnell dezimiert werden
~\cite{wuermer-ebola}. Die erfolgreichsten Viren, wie z.B. der
Herpesvirus 1, von dem 90\% der Menschen infiziert sind, sind meisten
harmlos oder können sogar eine symbiotische Verbindung mit dem Wirt
eingehen ~\cite{wuermer-herpes}.  Diese Beobachtungen aus dem
Tierreich können auch als Inspiration für künftige Wurmprogrammierer
dienen.

\subsubsection {Verbreitungswege}

Es gibt eine Vielzahl von Verbreitungswegen für Würmern. Grundsätzlich
können Würmer jede Art von Anwendung dafür verwenden neue Opfern zu
finden. Eine beliebte z.B. ist der E-Mail Dienst. Der Wurm wird
entweder im Anhang mitgeschickt oder ein Hyperlink in der Mail
verweist auf eine attackierende Seite, die den Angriff bzw. die
Infektion startet. Nach der Infektion wird die Wurm-Mail an alle auf
dem Zielsystem gefunden Adressaten mit Hilfe eines installierten
E-Mail Programms bzw. eines vom Wurm mitgelieferten verschickt. Analog
dazu können auch Instant-Messaging Dienste verwendet werden, die
ebenfalls entweder einen Link an alle lokal gespeicherten Adressaten
(\textit{in der Fachliteratur auch Buddies genannt}) auf eine
attackierende Seite verschicken oder den Wurm direkt allen anderen zum
Download anbieten. Weitere beliebte Anwendungen für Würmer sind IRC-
und P2P-Clients. Eine unübliche Verbreitunsgart sind
Wechseldatenträger, da bei diesem Medium die Reproduktionsrate äußert
gering ist ~\cite{wuermer-ways}.

\subsubsection {Angriffstaktiken}

Würmer können theoretisch alle vorhanden Angriffstaktiken verwenden,
um sich einen unberechtigten Zugang zu einem System zu verschaffen
(wenn man jedoch für die Verbreitung Wechselmedien verwendet, werden
menschliche Helfer benötigt). Z.B. werden sehr häufig Pufferüberläufe
erzeugt, um in das System zu gelangen. Ein weiterer Schwachpunkt ist
das Opfer selber: die Naivität, Ahnungslosigkeit, aber auch oft
Faulheit bzw. Gleichgültigkeit machen es dem Wurm leichter in das
System zu gelangen.  Z.B. konnte sich der \textit{I-love-you-Wurm} so
schnell verbreiten, weil die Opfer den Anhang öffneten.  Ein andere
Schwachpunkt sind unsichere Passwörter: Manche Würmer probieren
Wörterbuch-Passwörter aus oder kombinieren auf dem System gefundene
Nach- und Vornamen zu entsprechenden Hosts, um sich zu z.B. per SSH zu
authentifizieren.

\subsection {Morris-Wurm}
\subsubsection {Einführung} Um zu verstehen wie Würmer aufgebaut sind,
welche Motivation dahinter steckt und welche Konsequenzen die
Aussetzung eines Wurmes haben können wird im folgenden näher auf den
Morris-Wurm eingegangen.  Der erste Internet- bzw. Arpawurm mit dem
Namen Morris wurde vom gleichnamigen Autor Robert Tappan Morris
programmiert und 1988 ausgesetzt. Der Wurm infizierte ungefähr 6000
Systeme, also 10\% des gesamten Arpanets und verursachte einen Schaden
von bis zu 100 Millionen US- Dollar. Als Reaktion auf den Wurm wurde
das \textit{Computer Emergency Response Team Coordiantion Center}
(kurz CERT) gegründet ~\cite{wuermer-overview}.

\subsubsection {Angriffstaktiken}

Der Wurm konnte vor allem VAX and Sun-Maschinen infizieren, die mit
einem BSD-Unix betrieben wurden.  Die folgenden Schwächen
~\cite{wuermer-morris} bzw. Angriffsstrategien wurden vom Wurm
ausgenutzt:

\paragraph{Pufferüberlauf vom Programm fingerd}

Der Daemon \texttt{fingerd} verwendete die Funktion \texttt{gets}, um
eine Usereingabe entgegenzunehmen, ohne die Länge zu
überprüfen. Dadurch konnte ein Pufferüberlauf verursacht werden, um
die Rücksprungadresse auf den überschriebenen Puffer zeigen zu
lassen. Der eingeschleuste Code führte ein \texttt{chmk}
(\textit{change-mode-to-kernel}) aus, wodurch eine neue Shell-Session
erstellt werden konnte. Die Shell \textit{kontaktierte} ein bereits
infiziertes System und holt sich die benötigten Dateien zum
installieren des Wurmes. Bei den Sun-Rechnern führte diese Funktion
aufgrund eines Programmierfehlers zu einem Core Dump.  Obwohl die
Existenz dieses Sicherheitsloch bekannt war, wurden die verfügbaren
Fixes nicht von allen Administratoren eingespielt.

\paragraph{Bug/Backdoor im Programm sendmail}

Ein Bug bzw. eine Backdoor von sendmail ermöglichte es Remotebefehle
abzusetzen; und das ohne jegliche Authentifizierung. Im Debug-Modus
interpretierte sendmail die Empfängeradresse als Befehl, wodurch man
sich mit dem \textit{Empfänger} \texttt{/bin/sh} eine Shell mit den
jeweiligen Rechten öffnen lassen konnte. Obwohl auch dieser Fehler
schon seit langem bekannt war, gab es dennoch einige Systeme, die den
debug-Modus nicht deaktiviert haben \footnote{neuere Versionen von
sendmail haben diesen Modus standardmäßig deaktiviert}.

\paragraph{Passwortabfrage}

Nachdem eine erfolgreiche Infizierung stattfand, also alle
Objektdateien nachgeladen waren, durchsuchte der Wurm bestimmte
Konfigurationsdateien nach neuen Hosts und deren verschlüsselte
Passwörter. Zunächst wurden einfache Passwörter\footnote{wie z.B. der
Username, der Username rückwärts etc.} verschlüsselt und mit dem
gefunden verschlüsselten Passwort verglichen. Diese Methode wurde
angewandt, um so leise wie möglich zu sein und keine fehlgeschlagenen
Authentifizierungsversuche zu verursachen. Wenn das nicht
funktionierte wurde das Gleiche mit einem installiertem (falls
vorhanden) und mit einem vom Wurm mitgebrachten Wörterbuch versucht
\footnote{siehe cracksome.c in
ftp://coast.cs.purdue.edu/pub/doc/morris\_worm}.

\subsubsection {Sonstiges}

Der Wurm war nicht fertig und die Aussetzung erfolgte vermutlich
aufgrund einer Torschlusspanik des Autors, da kurz vor der Aussetzung
ein Sicherheitsloch bekannt wurde, das er ursprünglich ausnutzen
wollte. Sein ursprüngliches Ziel, die Größe des Arpanetes zu bestimmen
konnte gar nicht erreicht werden, da aufgrund eines Programmierfehlers
nichts \textit{Zuhause} ankam. Um zu verhindern, dass sich mehrere
Wurm-Instanzen installieren, wurde versucht eine Verbindung durch
einen Socket zu dem vermeintlich installierten Wurm zu etablieren und
falls dies gelang wurde eine bestimmte Variable auf 1 gesetzt, die
dazu führte, dass sich der Wurm deinstallierte. Nichtsdestotrotz war
jeder 7. Wurm \textit{unsterblich}, um gegen einen Fake-Wurm gewappnet
zu sein. Aus diesem Grund verursachte der Wurm eine DOS-Attacke, da
Robert Morris die Verbreitungsgeschwindgkeit unterschätzt hat.

\subsection {Erkenntnisse für \f}

Zunächst einmal können wir sehr viele technische Merkmale des
Morris-Wurms für unseren Wurm \footnote {wenn es denn ein Wurm werden
sollte} umsetzen, indem wir die entsprechenden Meterpreter-Scripts
suchen bzw. selber welche implementieren. (z.B. forked sich der Wurm
in regelmäßigen Abständen und löscht den Parent-Prozess, um nicht
entdeckt zu werden). Außerdem können wir viel von der Infrastruktur
lernen und den grundsätzlichen Aufbau eines etwas komplexeren Wurms
möglicherweise nachbauen, da der Quellcode zur Verfügung steht.

Wir können aus den begangenen Fehler lernen: unser Wurm bzw. unser
wurmartiges Brückenkopfprogramm \footnote {eine genaue Spezifikation
fand noch nicht statt} sollte sich keinesfalls mehrmals installieren,
aber gleichzeitig nicht einfach gestoppt werden können, falls ein
Fakeprogramm immer mit \texttt{ ich bin schon installiert} antwortet.







 % WF
\clearpage\section{Stuxnet}
\label{compositions:stuxnet}

\authors{\DM}{}

Die Folien zu diesem Referat sind im Projekt-Wiki unter
\url{http://rn.informatik.uni-bremen.de/wiki-fidius/Stuxnet?action=AttachFile&do=view&target=Referat-Stuxnet.pdf}
zu finden.\footnote{ggf. Zugangsdaten benötigt}

\subsection{Überblick}

Im Juni 2010 tauchte ein Computerwurm auf dem Radar der
Anti-Viren-Hersteller auf, der ein bisschen anders war als andere
Würmer. Es werden nicht -- wie sonst üblich -- Botnetze aufgebaut, die
dann Spam versenden oder Kreditkarteninformationen ausspähen. Auch die
Verbreitung des bald Stuxnet getauften Wurmes geschieht offenbar nicht
willkürlich, wenn man sich die Analysen der IP-Adresse der infizierten
Hosts ansieht (vgl. Abbildung~\ref{fig:comp:stuxnet:hosts}).
Schließlich beeindruckt die Gesamtzahl der ausgenutzten
Schwachstellen, insbesondere, weil dort viele so genannte
Zero-day-Exploits auf einmal ausgenutzt werden.

\begin{figure}[hp]
  \centering
  \includegraphics[width=\linewidth]{./images/comp-stuxnet-hosts.png}
  \caption[Geographische Verteilung der infizierten Hosts]
    {Geographische Verteilung der infizierten Hosts, Analysen von [Symantec]}
  \label{fig:comp:stuxnet:hosts}
\end{figure}

Nach weiteren Analysen, hier seien die von Symantec und ESET besonders
hervorzuheben, zeichnete sich ein neues Bild dieses Wurms. So
operierte Stuxnet offenbar schon ein Jahr verdeckt und nutzt bis dahin
unbekannte Techniken, um unerkannt in ein System einzudringen (vgl.
Tabelle~\ref{tab:comp:stuxnet:history}).


\begin{tabelle}{.175,.8}{Zeitlicher Ablauf (teilw. nachdatiert)}{tab:comp:stuxnet:history}
  \textbf{Datum} & \textbf{Ereignis} \\\endheadline
  
  Nov. 2008   & erste Trojaner-Variante, welche die Win-LNK-Lücke ausnutzt
                (später MS10-046) \\
  Juni 2009   & früheste gesehene Stuxnet-Variante (ohne signierte Treiber und
                MS10-046-Exploit) \\
  25.01.2010  & Stuxnet-Variante mit digitaler Unterschrift von Realtek \\
  März 2010   & Stuxnet-Variation nutzt MS10-046 aus \\
  17.06.10    & Stuxnet wird von \textit{VirusBlokAda} entdeckt; LNK-Lücke wird
                aufgedeckt (CVE-2010-2568) \\
  16.07.10    & von Microsoft kommen Informationen zu einer Schachstelle beim
                Verarbeiten von \textit{shortcut icons}/.lnk-Dateien, die
                Remote-Code-Execution ermöglicht\\
  16.07.10    & Verisign annuliert Realtek-Signatur \\
  17.07.10    & ESET entdeckt Stuxnet-Variante mit JMicron-Signatur \\
  19.07.10    & Siemens sorgt sich um WinCC-SCADA-Systeme infizierende Malware \\
  20.07.10    & Symantec beobachtet C\&C-Traffic \\
  22.07.10    & Verisign annuliert JMicron-Signatur \\
  02.08.10    & Microsoft patcht MS10-046 \\
  12.10.10    & Microsoft schließt weitere von Stuxnet genutzte Lücken
\end{tabelle}

\subsection[Exkurs: Speicherprogrammierbare Steuerungen] {Exkurs:
Speicherpro-\\grammierbare Steuerungen}

Um das Angriffsziel von Stuxnet verstehen zu können, soll an dieser
Stelle ein kurzer Abstecher in die Industrieautomatisierung und
Prozesssteuerung gemacht werden. Im Speziellen sollen hier SCADA und
PLC erklärt werden.

\subsubsection{SCADA}

Mit \emph{Supervisory Control and Data Acquisition}, kurz SCADA (bei
Siemens \enquote{SIMATIC WinCC}), wird ein Konzept zum Überwachen und
zur Steuerung insbesondere von Industrieanlagen beschrieben. Im
Prinzip visualisieren hierbei digitale Bildschirmanzeigen die zuvor
anaolgen (bzw. elektrisch geregelten) Anzeigen. Darüber hinaus erlaubt
dieses System auch, die in den Prozess eingebundenen Steuerungen
anzusprechen und zu manipulieren. Ein Beispiel einer solchen Anzeige
ist in Abbildung~\ref{fig:comp:stuxnet:scada} zu sehen.

\begin{figure}
  \centering
  \includegraphics[width=.7\linewidth]{./images/comp-stuxnet-scada.png}
  \caption[Typisches SCADA-Interface]
    {Typisches SCADA-Interface, [Wikimedia Commons]}
  \label{fig:comp:stuxnet:scada}
\end{figure}

Ein direkter Vorteil eines solchen Systems ist die natürliche
Darstellung von Prozessketten. Standardisierte Icons erlauben, sich
schnell einen Überblick über ein System zu verschaffen. Von Nachteil
ist dafür das Fehlen von mehreren, über die Anlage verteilten analogen
Anzeigen, welche zwar bei eventuellen widersprüchlichen Ausgaben
Notfallmaßnahmen nicht mehr behindern können, dafür aber ein volles
Vertrauen in die virtuellen Anzeigen verlangen. Die rein digitalen
Anzeigen bieten also eine Angriffsfläche, dessen Manipulation gar
nicht bemerkt werden kann.

\subsubsection{PLC}

Die im vorherigen Abschnitt genannten \emph{Steuerungen} stellen so
genannte \textit{Programmable Logic Controller} (PLC;
dt. Speicherprogrammierbare Steuerungen (SPS)) dar. Diese
Kleinstcomputer nehmen über Eingänge Messungen vor, verarbeiten diese
und regeln, wenn nötig, Ventile und ähnliches.

Über eine Ethernet-Schnittstelle\footnote{bzw. einem
sog. PROFINET-Adapter} kann eine solche Steuerung mit SCADA-Terminals
kommunizieren und von dort Befehle oder neue Verarbeitungsprogramme
empfangen. In Siemens SIMATIC-Anlagen übernimmt das Programm
\emph{WinCC} die Anzeige und Überwachung von PLCs, und mit einer
\emph{Step7} genannten Entwicklungsumgebung werden diese PLCs
programmiert.

\subsubsection{Anwendungen}

Wo immer technische Prozesse überwacht, Materialzuflüsse geregelt,
Temperaturen, Drucke oder Füllstände gemessen werden müssen, setzt man
heute PLCs ein.

Beispielanwendungen sind

\begin{itemize}
  \item Klärwerke/Wasserkraftwerke
  \item Herstellung von Waschmittel, Folien, chemische Werkstoffe
  \item Tunnelbohrmaschinen
  \item Urananreicherungsanlagen
\end{itemize}

sowie weitere.

\subsubsection{Ziel von Stuxnet}

Die Analysen von Stuxnet zeigen nun, dass Stuxnet nach ganz bestimmten
Systemkonfigurationen Ausschau hält. So bricht die Installation von
Stuxnet ab, wenn es keine WinCC-Installation\footnote{Siemens nennt
seine SCADA-Umsetzung Simatic WinCC} vorfindet, und die Verbreitung
stoppt nach 21 Tagen, wenn es feststellt, dass über den aktuell
infizierten Host keine Siemens-PLCs zu erreichen sind.

Dies allein zeigt schon, dass Stuxnet mindestens in die Kategorie der
gezielten Industriespionage fällt. Diese Ausarbeitung wird noch
zeigen, dass man durchaus von Industriesabotage sprechen muss.

\subsection{Aufbau von Stuxnet}

\subsubsection{Innere Organisation}

Im Wesentlichen besteht Stuxnet aus einer etwa 500\,kB großen
DLL-Datei. Darin enthalten sind die so genannten \textit{exports}
(also die von außen aufrufbaren, exportierten Funktionen) und
\textit{resources} (Konfigurations- und weitere DLL-Dateien,
Datei-Templates, (signierte) Treiber-Dateien und Strings). Wenn im
Folgenden von der \enquote{Stuxnet-DLL} gesprochen wird, ist damit
diese Datei gemeint.

Diese \textit{exports} und \textit{resources} sind nicht-linear
nummeriert und sollen im Folgenden nur zur Identifikation dienen. Eine
vollständige Liste aller \textit{exports} und \textit{resources}
findet sich sowohl im Symantec"=% Stuxnet"=Dossier, als auch im
Bericht von ESET.

\subsubsection{Kontrollfluss}

\paragraph{Windows-LNK-Lücke}

Stuxnet kennt verschiedene Wege, in ein System einzudringen. Die in
der Presse am meisten beachtete Variante, einen Windows-Rechner zu
kapern, nutzt die Windows-LNK-Lücke\footnote{CVE-2010-2568, MS10-046}
aus.

Hierbei handelt es sich um ein Fehler beim Verarbeiten von
\enquote{Verknüpfungen} (\texttt{lnk}-Dateien;
\textit{Link}). Speziell präparierte \texttt{lnk}-Dateien erlauben das
dynamische Laden von verschiedenen Icons. Gedacht ist diese Funktion
für \textit{Windows"=Systemsteuerungselemente}, die beispielsweise
abhängig von Gerätestati unterschiedliche Symbole anzeigen.  Damit das
wiederum funktioniert, muss zum Systemsteuerungsicon ein ausführbares
Programm hinterlegt sein, das beim Laden ausgeführt werden soll.

Da sich unter Windows die Systemsteuerung im Prinzip nicht von
ordinären Verzeichnissen unterscheidet, muss die Funktion vom
Nachladen von Code beim Anzeigen von Symbolen auch ausserhalb der
Systemsteuerung verfügbar sein. Und tatsächlich erlaubt der Header der
\texttt{lnk}-Format-Spezifikation die Angabe einer nachzuladenden
Bibliothek anzugeben, sofern eine bestimmte
\textit{Class-ID}\footnote{in der Windows-Regestry hinterlegte
GUID/UUID zum identifizieren von Systemobjekten} das Icon als
Systemsteuerungselement auszeichnet (siehe
Abbildung~\ref{fig:comp:stuxnet:lnk-vulnerability}, Punkt 1).  Findet
der \textit{Windows-Explorer}\footnote{oder ein anderes, Icons
anzeigendes Programm} nun ein solch präpariertes Symbol, wird eine
Funktionskaskade aufgerufen, die in der System-Funktion
\texttt{LoadLibraryW()}
(Abb.~\ref{fig:comp:stuxnet:lnk-vulnerability}, Punkt 2) mündet,
welcher der Pfad zu der DLL übergeben wird, die im \texttt{lnk}-Header
angegeben ist (in diesem Fall die Stuxnet-DLL).

Somit kann der Angreifer Code ausgeführen, ohne dass der Benutzer
etwas tun muss. Diese Lücke kann zudem nicht nur von USB-Sticks oder
anderen Datenträgern ausgenutzt werden, sondern auch via
Samba-Netzwerkfreigaben.

\begin{figure}
  \centering
  \begin{tikzpicture}[
    inner/.style={text width=7em,inner sep=2pt,anchor=center,text centered,minimum height=1em},
    box/.style={draw,very thin},
    proc/.style={box,inner sep=2pt,rounded corners=4pt}
  ]
    \node[inner]
      (lnk format) at (0,2em)
      {\textbf{.LNK File Format}};
    \node[inner,box,below=1pt of lnk format]
      (header)
      {Header};
    \node[inner,box,below=0pt of header]
      (id list)
      {Shell Item Id List};
    \node[inner,box,below=0pt of id list]
      (file info)
      {File Location Info};
    \node[inner,box,below=0pt of file info]
      (tmp)
      {Description};
    \node[inner,box,below=0pt of tmp]
      (tmp)
      {Relative Path};
    \node[inner,box,below=0pt of tmp]
      (tmp)
      {Working Directory};
    \node[inner,box,below=0pt of tmp]
      {…};
    \node[box,inner sep=1pt,right=10pt of file info] (clsid)
      {CLSID\_MyComputer};
    \node[box,inner sep=1pt,yshift=-2.6ex,right=10pt of file info] (media)
      {\char`\\\char`\\$\langle$path magic$\rangle$\char`\\foobar.dll};
    \node[below=1pt of media,xshift=-3em] {(1)};
    \draw[latex'-,densely dotted] (file info.0) -- (clsid.180);
    \draw[latex'-,densely dotted] (file info.0) -- (media.180);
    \node[proc]
      (findcplinfo) at (5.5,0)
      {\texttt{CPL\_FindCPLInfo()}};
    \draw[->,densely dashed,shorten >=1pt] (clsid) |- (findcplinfo);
    \node[proc,xshift=4em,below=1ex of findcplinfo]
      (loadandfindapplet)
      {\texttt{CPL\_LoadAndFindApplet()}};
    \node[proc,xshift=3em,below=1ex of loadandfindapplet]
      (loadcplmodule)
      {\texttt{\_LoadCPLModule()}};
    \node[proc,xshift=3em,below=1ex of loadcplmodule]
      (loadlibraryw)
      {\texttt{LoadLibraryW()}};
    \node[below=1pt of loadlibraryw,xshift=1em] {(2)};
    \draw[->] (findcplinfo.195) |- (loadandfindapplet);
    \draw[->] (loadandfindapplet.200) |- (loadcplmodule);
    \draw[->] (loadcplmodule.200) |- (loadlibraryw);
    \draw[->,thin] (media) |- (loadlibraryw);
  \end{tikzpicture}
  \caption[Eintrittsvektor Windows-LNK-Lücke]
    {Eintrittsvektor Windows-LNK-Lücke, [ESET, S.\,22ff.]}
  \label{fig:comp:stuxnet:lnk-vulnerability}
\end{figure}

\paragraph{Systemrechte erlangen}

\begin{figure}[p]
  \centering
  \includegraphics[width=\linewidth]{./images/comp-stuxnet-export-15.png}
  \caption{Kontrollfluss der Stuxnet-Installationsroutine (Export 15)}
  \label{fig:comp:stuxnet:export15}
\end{figure}

\begin{figure}[p]
  \centering
  \includegraphics[width=\linewidth]{./images/comp-stuxnet-export-16.png}
  \caption{Kontrollfluss der Stuxnet-Infektionsroutine (Export 16)}
  \label{fig:comp:stuxnet:export16}
\end{figure}

Die Stuxnet-DLL startet mit der Ausführung des Export 15 (dessen
Kontrollfluss ist in Abbildung~\ref{fig:comp:stuxnet:export15}
dargestellt). Hier werden zunächst einige Systemchecks durchgeführt
(Version und Bitigkeit des OS) und es wird überprüft, ob der aktuelle
Stuxnet-Prozess bereits mit Rechten des Systembenutzers läuft.

Ist dies nicht der Fall, wird für Windows 2000 und XP eine
Schwachstelle im für die Tastaturlayouts verantwortlichen
Systemtreiber \texttt{win32k.sys} ausgenutzt, um Systemrechte zu
erlangen. Um diese Schwachstelle auszunutzen, wird zunächst eine
präparierte Tastaturlayout-Datei durch \texttt{win32k.sys}
geladen. Eine ungültige Byte-Folge in dieser Datei führt nun beim
Parsen dazu, dass sich bestimmte Variablen innerhalb der
\texttt{win32k.sys} mit Adressdaten füllen, die auf einen von Stuxnet
kontrollierten Speicherbereich zeigen. Durch einen
\textit{use-after"=free}"=Fehler springt \texttt{win32k.sys}
schließlich in diesen Bereich und führt Stuxnet mit Systemrechten aus.

Unter Windows Vista und 7 nutzt Stuxnet einen Fehler im \textit{Task
Scheduler} aus. Hierzu liegen aber noch keine Informationen vor, da
sowohl Symantec als auch ESET \textit{full disclosure} betreiben und
erst dann Informationen bereitstellen wollen, wenn Microsoft diese
kritische Lücke gestopft hat.

\paragraph{Selbstschutz\dots}
\label{sec:comp:stuxnet:protection}

Sofern Stuxnet endlich maximale Rechte erworben hat, erstellt es aus
einer Resource und der kompletten Stuxnet-DLL eine ausführbare Datei,
die es in ein laufendes Anti-Viren-Programm oder den
Windows-Systemdienst injiziert (je nach Verfügbarkeit wählt Stuxnet
aus mehreren Anti-Viren-Programmen) und dort zur Ausführung bringt. Im
Kontext dieser Programme beginnt Stuxnet damit, sich durch die
Installation zweier Rootkits vor der Entdeckung zu schützen.

Der Selbstschutz-Mechanismus besteht im Wesentlichen aus einem digital
von \textit{Realtek Semiconductor Corp.} signiertem Systemtreiber,
welcher die von Stuxnet generierten Dateien (\texttt{lnk}-Dateien
bestimmter Größe und \texttt{tmp}-Dateien mit bestimmtem Namen) vor
dem Zugriff schützt und dabei als Low-Level-System-Hook auf
Dateisystemebene fungiert.

Darüber hinaus sorgt Stuxnet mit einem weiteren digital signiertem
Treiber dafür, dass die Stuxnet-DLL auch bei jedem Windows-Start
geladen wird, und in Prozessen eingeschleust wird, die zum anvisierten
Ziel gehören:

\begin{itemize}
  \item Die ausführbare Datei \texttt{s7tgtopx.exe} gehört zum
\textit{SIMATIC Manager}, also einem zentralen Programm zur Verwaltung
von SPS-Projekten.
  \item Die Datei \texttt{CCProjectMgr.exe} (WinCC-Projektmanagement)
bearbeitet Steuerprogramme, die schließlich mit demselben Tool auch
auf die Siemens-SPS geschrieben werden.
\end{itemize}

Die Injektion in diese beiden (und weitere) Prozesse soll
sicherstellen, dass auch der zuvor angelegte Treiber garantiert wieder
geladen wird, und die Anti-Viren"=Signaturerkennung fehlschlägt,
sollte es einmal ein Update der Signaturdatenbank gegeben haben.

Der zuletzt genannte Treiber weist ein erwähnenswertes Merkmal auf:
als Verisign die von Realtek ausgestellte Signatur am 16.07.2010 für
ungültig erklärte, tauchte bereits am 17.07.2010 eine Stuxnet-Variante
mit einer neuen Signatur von JMicron Technology auf. Die örtliche Nähe
der Büroräume von Realtek und JMicron im gleichen taiwanesischen
Industriepark legt einen vorbereitenden physichen Einbruch nahe.

\paragraph{\dots\ und Selbstverbreitung}

Nachdem sicher gestellt ist, dass Stuxnet unbemerkt arbeiten kann,
beginnt es, sich auf vier verschiedenen Wege zu verbreiten:

\begin{enumerate}[label=\textbf{\arabic*.}]
  \item \textbf{Peer-to-Peer-Netze.}  Stuxnet bringt einen RPC-Server
und -Client mit, und lauscht auf eingehende RPC-Nachrichten
bzw. verschickt diese im lokalen Netz. Diese Methode dient vor allem
dazu, die Versionen auf den jeweils infizierten Hosts zu vergleichen
und die ältere Version zu aktualisieren.
  
  \item \textbf{Offener MS SQL-Server.}  Bei der Installation von
WinCC wird ein Microsoft-SQL-Server installiert.  Dabei werden jedoch
die Zugangsdaten nicht durch zufällige Daten ersetzt, sondern es
werden die fest verdrahteten Datenbank-Benutzer und "~Passworte
weiterverwendet. Dadurch ist es Stuxnet möglich, den SQL-Server von
außen anzusprechen und mit mit vordefinierten \textit{Stored
Procedures} zuerst die Stuxnet-DLL als hexadezimalkodierten String in
einer Tabelle abzulegen, diesen String dann in eine Datei zu schreiben
und schließlich mit einem \texttt{EXEC}-Statement diese Datei
auszuführen. Dass der SQL-Server mit Systemrechten läuft, ist für
Stuxnet dabei nur von Vorteil.

  \item \textbf{Print Spooler Vulnerability.}  Ein Fehler im
Drucker-Subsystem (\texttt{spoolsvr.exe}) erlaubt es, beliebige
Dateien ins Systemverzeichnis \enquote{zu drucken}. Hier führt eine
mangelhafte Überprüfung der Parameter der Systemfunktion
\texttt{StartDocPrinter()} dazu, dass auch Benuter mit Gastrechten
über die \textit{Datei- und Druckerfreigabe} den Rechner befallen
können.
  
  \item \textbf{Conficker-Schwachstelle.}  Stuxnet nutzt auch eine
ältere Schwachstelle aus, die man schon zu Zeiten des Conficker-Wurmes
(erstes Auftreten etwa im Oktober 2008) kannte. Auch hier wird wieder
die Datei- und Druckerfreigabe ausgenutzt, um mit so genannten
\textit{Netzwerk-Jobs} eine Datei auszuführen, die zuvor über die
Standardfreigaben \texttt{ADMIN\$} und \texttt{C\$} auf ein System
geschleust wurde.
\end{enumerate}

Das primäre Ziel ist hierbei, die aktuelleste Stuxnet-Version in einem
Netz zu verteilen. Auf diese Weise können neue Stuxnet-Varianten auch
Hosts erreichen, die nicht direkt am Internet hängen, sondern -- wenn
überhaupt -- hinter diversen Firewalls abgeschirmt sind. Dadurch, dass
Stuxnet auch Datenträger infiziert, ließen sich theoretisch auch
Rechner erreichen, die überhaupt nicht mit anderen Rechnern verbunden
sind.

\subsection{Infektion von Siemens-Installationen}

Sobald Stuxnet in die ausführbaren WinCC- und Step7-Dateien injiziert
wurde (siehe~\ref{sec:comp:stuxnet:protection}), schreibt es in
weiteren ausführbaren Dateien die Funktionstabellen so um, dass die
darin enthaltenen Funktionen \texttt{CreateFileA()}
bzw. \texttt{StgOpenStorage()} auf Stuxnet-Wrapper zeigen. Diese
Funktionen werden beim Laden und Speichern von Step7-Projekten
ausgeführt und dienen Stuxnet dazu Projekte zu identifizieren, die
bestimmte Kriterien erfüllen:

\begin{itemize}
  \item Das WinCC-Projekt muss relativ aktiv genutzt werden. Stuxnet
erkennt inaktive Projekte durch den Zeitstempel des letzten Zugriffs
und setzt 42~Monate als Unterscheidungsmarke an.
  
  \item Dieses Projekt darf kein Projekt aus der
WinCC-Beispielsammlung sein.  Hier nutzt Stuxnet die Pfadangaben als
Indikator.
  
  \item Schließlich muss das Projekt eine \texttt{mcp}-Datei
(eigentlich ein Projekt-Metadaten-Container) an einer bestimmten
Stelle gespeichert haben.
\end{itemize}

In die \texttt{mcp}-Datei speichert Stuxnet eine verschlüsselte Kopie
seiner selbst, und in die Step7-Projektdateien wird ein DLL-Loader
platziert der diese \texttt{mcp}-Datei entschlüsselt und zur
Ausführung bringt. Somit verbreitet sich Stuxnet auch dann auf anderen
Hosts, wenn die Projektdateien bspw. aus einem Repository geholt oder
mit USB-Sticks verteilt werden, die nicht von der Windows-LNK-Lücke
betroffen sind.

\subsubsection{Step7/SPS-Rootkit}

In einem der letzten Schritte vor dem Ziel schiebt Stuxnet der
Step7-Installation eine DLL unter, die sich in die Kommunikation
zwischen Step7 und dem PLC einklinkt bzw. genauer die
Kommunikationskomponente ersetzt. Diese DLL (\texttt{s7otbxdx.dll})
bietet verschiedene Funktionen, um mit die PLC via PROFINET
anzusprechen, von dessen Programmspeicher Codeblöcke zu lesen und
wieder zu beschreiben.

Im Originalzustand wird der von dem PLC ausgelesene Code Eins-zu-Eins
im Step7-Editor angezeigt. Stuxnet manipuliert die Anzeige der Blöcke
nun so, dass eigene, auf die PLC gebrachte Schadroutinen beim Auslesen
ausgeblendet, bzw.  beim Schreiben auf die PLC (wieder) hinzugefügt
werden (vgl.  Abbildung~\ref{fig:comp:stuxnet:s7otbxdx}
vs.~\ref{fig:comp:stuxnet:s7otbxsx}).

\begin{figure}[p]
  \centering
  \includegraphics[width=.7\linewidth]{./images/comp-stuxnet-s7otbxdx.png}
  \caption{Normale Kommunikation zwischen Step7 und den PLCs}
  \label{fig:comp:stuxnet:s7otbxdx}
\end{figure}

\begin{figure}[p]
  \centering
  \includegraphics[width=.7\linewidth]{./images/comp-stuxnet-s7otbxsx.png}
  \caption{Manipulierte Kommunikation über eine von Stuxnet ersetzte DLL}
  \label{fig:comp:stuxnet:s7otbxsx}
\end{figure}



Bei den von Stuxnet geschriebenen Codeblöcken muss man im Wesentlichen
zwischen drei verschiedenen Sequenzen für zwei unterschiedliche
PLC-Typen sprechen. Es bliebt zunächst auch unklar, was diese
Sequenzen bei Ausführung \emph{genau} tun, da aus ihnen heraus
einerseits andere (unbekannte) Codeblöcke aufgerufen werden, und
andererseits Code-Obfuscation betrieben wurde -- die Funktionsnamen
lassen also keine Rückschlüsse auf die wirkliche Funktion zu.

\subsection{Fazit und Ausblick}

Die Anzahl der zum Teil stark verschiedenen Exploits zeigt sehr
deutlich, dass der Angreifer sicher sein wollte, sein Ziel zu
erreichen. Welches Ziel dies aber ist, oder wer der Angreifer sei, ist
bis heute nicht geklärt. Im Code finden sich einige Indizien, die
darauf hindeuten können, wer dafür Verantwortlich ist, bzw. wer oder
was das Ziel ist:

\begin{itemize}
  \item Eine Pfadangabe enthält die Zeichenfolge \emph{myrtus}. Diese
kann sowohl \enquote{My RTUs}\footnote{RTU = Remote Terminal Unit, ein
Synonym für PLCs} gelesen werden, als auch nach hebräischer
Übersetzung auf das Buch Ester verweisen, welches dem jüdischen Volk
die Verteilung und Tötung ihrer Feinde erlaubt.
  \item Liest Stuxnet aus einem bestimmten Windows-Registy-Key den
Wert \emph{19790509}, bricht es seine Installations- und
Infektionsroutine ab.  Neben vielen weiteren Ereignissen wurde mit
Habib Elghanian am 09.05.1979 einer der ersten Juden von der damals
neu gegründeten Islamischen Republik ermordet.
  \item Einer der Rücksprungadressen im PLC-Code hat die Adresse
\texttt{DEADF007}. Transliteriert man diese Zeichenfolge aus dem
\textit{Leetspeck} zu \enquote{dead foot}, so bezeichnet dies den
Ausfall einer (Flugzeug-) Turbine.
\end{itemize}

In jedem Fall sei hier angemerkt, dass diese gelegten Fährten nichts
anderes als Ablenkung sein können, um die wahren Täter zu schützen,
bzw. der Verdacht auf andere abzuwenden.

Zwischenzeitlich wurden auch neue Erkenntnisse zu Stuxnet gewonnen:

\begin{itemize}
  \item Symantec ließ verlauten\footnote{Symantec: \enquote{Stuxnet: A
Breakthrough},
\url{http://www.symantec.com/connect/blogs/stuxnet-breakthrough},
12.11.2010, zuletzt abgerufen am 14.11.2010}, dass sie verstünden,
welche Funktion die von Stuxnet auf die PLCs geschriebenen Codeblöcke
erfüllen. So verändert Stuxnet die Ausgangsfrequenz bestimmter
angeschlossener Frequenzumrichter, um damit schließlich die
Umdrehungsgeschwindigkeiten von Motoren so zu manipulieren, dass der
daran angeschlossene Fertigungsprozess sabotiert wird.
  
  \item Einen Tag darauf bestätigte\footnote{Langner: \enquote{Yeah,
it's a drive array for the 315 attack code},
\url{http://www.langner.com/english/?p=415}, 13.11.2010, zuletzt
abgerufen am 16.11.2010} die Langner Communications GmbH die Analysen
von Symantec. Ralph Langner veröffentlichte bereits im September die
hochspekulative Annahme\footnote{Langner: \enquote{Stuxnet logbook,
Sep 16 2010, 1200 hours MESZ},
\url{http://www.langner.com/english/?p=217} zuletzt abgerufen am
16.11.2010}, dass das Ziel das iranische Atomprogramm sei. Die
Analysen von Langner und Symantec lassen dazu nun kaum Alternativen.
\end{itemize}

Die spannendste Frage, wer sich für die Entwicklung von Stuxnet
verantwortlich zeichnet, wird aber wohl nie beantwortet werden.

\subsection{Materialien}

\begin{itemize}
  \item Aleksander Matrosov, Eugene Rodionov, David Harley, Jurai
Malcho: \enquote{Stuxnet under the Microscope}. ESET, 23.09.2010.
\url{http://www.eset.com/resources/white-papers/Stuxnet_Under_the_Microscope.pdf}

  \item Nicolas Falliere, Liam O Murchu, Eric Chien:
\enquote{W32.Stuxnet Dossier, Version 1.1}. Symantec, 12.10.2010.
\url{http://www.symantec.com/content/en/us/enterprise/media/security_response/whitepapers/w32_stuxnet_dossier.pdf}

  \item Bruce Schneier: \enquote{The Story Behind The Stuxnet
Virus}. Forbes-Kommentar, 07.10.2010.
\url{http://www.schneier.com/blog/archives/2010/10/stuxnet.html}

  \item Frank Rieger: \enquote{Der digitale Erstschlag ist
erfolgt}. FAZ, 22.09.2010.  \url{http://www.faz.net/-01i43d}

  \item Frank Rieger, Felix von Leitner: \enquote{Alternativlos, Folge
5}. Podcast, 26.09.2010.  \url{http://alternativlos.org/5/}

  \item Wikipedia: \enquote{Stuxnet}.
\url{https://secure.wikimedia.org/wikipedia/en/wiki/Stuxnet}

  \item Ralph Langner: \enquote{Stuxnet logbook}. Blog.
\url{http://www.langner.com/en/}
\end{itemize}
 % DM
\clearpage\section{Tunnel - Hole Punching}
\label{composition:tunnel} Im Rahmen der zweiten Hälfte des Projektes
\f ist das Referat \textbf{Tunnel - Hole-Punching} entstanden.
Das vorliegende Dokument ergänzt das Referat als entsprechende
Ausarbeitung.

Das Thema des Referates hatte seinen Ursprung in der Begründung der
Verwendung von Brückenköpfen, welche es gilt in der zweiten Hälfte des
Projektes einzusetzen. Brückenköpfe werden im Rahmen des Projektes,
als durch '\f' unter Kontrolle gebrachte Rechner verstanden, die
es ermöglichen, weitere Netze durch sich selber zu erschliessen und zu
benutzen. Im weitesten Sinne können Brückenköpfe als Gateways
verstanden werden.

Diese ermöglichte Kommunikation, auch über Brückenköpfe hinweg, findet
auf Rechnern statt, welche von '\f' erfolgreich kompromittiert
wurden. Um Kommunikationen zu Rechnern aufrecht erhalten zu können
müssen verbindungslose oder verbindungsorientierte Übertragungsdienste
bereitgestellt werden. Aus diesem Grund sah es das Referat, als auch
die Ausarbeitung vor sich auf VPN-Software zu konzentrieren. Da es im
Rahmen des Projektes jedoch schon Vorstellungen, Erfahrung und die
Nutzung von VPN-Software gab/gibt, wurde das Kernthema auf einen
anderen Schwerpunkt verlagert - dem des Hole Punching.

Hierbei handelt es sich um ein Verfahren welches beschreibt, wie
Verbindungen zwischen entfernten Netzen hergestellt werden können,
auch wenn die zu kommunizierenden Rechner jeweils hinter einer
restriktiven Firewall und/oder einem NAT-GW eingebunden sind - es
handelt sich um ein Spezialfall des Tunneling.

Im Folgenden wird kurz auf die Nachteile von VPN-Software in Bezug auf
die Verwendung innerhalb des Projektes \f eingegangen.  Es wird
ein demonstratives Beispiel aufgezeigt wie Firewalls und NATs durch
Hole Punching durchbrochen werden können.

Abschließend wird auf den möglichen Nutzen für FIDIUS näher
eingegangen.

\subsection{Problem} Um einen Datenaustausch zwischen zwei Rechnern
über das WAN zu ermöglichen, welche sich beide hinter einem NAT
befinden, benötigt es Maßnahmen, welche als nicht immer gegeben
angesehen werden können. Aus der Praxis kennt man gängige Verfahren
wie \texttt{portforwarding} oder \texttt{UPnP}, die es regelhaft
erlauben, gerichtete und bestimmte Kommunikation zuzulassen oder nicht
zu ermöglichen. Beim Einsatz dieser Techniken wird die korrekte
Konfiguration jedoch schon vorausgesetzt, um einen
Kommunikationsaustausch überhaupt zuzulassen.  Sind diese
Voraussetzungen erfüllt können gängige VPNs eingesetzt werden.
\subsection{Software als Lösung}
\label{sec:sw-solution} Gängige VPN-Software, so wie wir sie im
Projekt in Form von OpenVPN einsetzen, können Rechner sehr gut direkt
miteinander verbinden oder einen kontrollierten, integeren
Datenaustausch ermöglichen. Jedoch müsste diese Software auf den zu
kompromittierenden Rechnern installiert und konfiguriert werden. Die
Funktionalität ist, da es sich im Fall von OpenVPN um den
Quasi-Standard handelt, im Szenario '\f-Brückenkopf' von
ausreichend bis überdimensioniert zu bewerten.

Die hohe Funktionalität erhöht zudem die Fehlerwahrscheinlichkeit und
mindert nicht die Notwendigkeit der anzupassenden
Konfiguration. Hierzu gehören, neben der eventuellen Installation von
Tap- oder Tun-Devices, auch das Routing, so wie die schon erwähnte
Ausgangslage, dass Firewalls und NATs entsprechend angepasst sind.

Dies bezüglich gibt es auch Software wie \texttt{campagnol
\url{http://campagnol.sourceforge.net/}}, welche ohne entsprechende
Einstellung von Firewalls eine Kommunikation zwischen zwei Rechnern
ermöglicht.

\textit{Campagnol is a distributed IP-based VPN software able to open
new connections through NATs or firewalls without any configuration.}

Die Installation vorausgesetzt könnte diese spezielle Software für
eine Kommunikation zwischen Rechnern, jeweils hinter einem NAT,
erfolgreich eingesetzt werden, ohne dass Informationen oder
Konfigurationen bezüglich einer bestehenden Firewall oder eines NATs,
notwendig sind. Ausschlaggebend dabei ist die Technik und das bekannte
Prinzip, nicht die Software selber, so dass sich der Schwerpunkt des
Referates und dieser Ausarbeitung auf eben diese Technik \textbf{Hole
Punching} konzentriert, als weitere Software zu erläutern.

\subsection{Technik als Lösung}
\subsubsection{NAT - Network Address Translation} Spricht man vom
\textit{Hole Punching} und stützt sich auf dementsprechende Literatur,
beginnen sehr viele Quellen mit der Vorstellung der verschiedenen
NAT-Typen, so dass diese grundlegende Information über die bestehenden
Klassen der NATs an dieser Stelle nachgeholt werden soll.

Gemäß ihres Verhaltens, werden NATs durch vier Typen klassifiziert

\begin{enumerate}
\item Full Cone
\item Restricted Cone
\item Port Restricted Cone
\item Symmetric
\end{enumerate}

\paragraph{Full Cone NAT}

Bezogen auf Abbildung~\ref{fig:nat-full-cone} sendet ein privater Host
(PH) einen initialen Request zum Remote Host A.  Hierfür öffnet der
NAT-Router einen öffentlichen Endpunkt (öffentliche Adresse mit
Port). Jegliche Kommunikation oder Verbindung zu einem entfernten
Rechner vom Port des privaten Hosts wird auf den gleichen Port des
NAT-Routers \texttt{gemappt}. Bei einem Full Cone NAT kann nun jeder
Remote Host von jedem Quellport zum Private Host A kommunizieren, in
dem er den öffentlichen Endpunkt (öffentliche Adresse mit Port) des
NAT-Routers benutzt.

\begin{figure}
  \centering
  \includegraphics[width=9cm]{images/nat_full_cone.png}
  \caption{Full Cone NAT}
  \label{fig:nat-full-cone}  
  \small{Quelle: \url{http://sarwiki.informatik.hu-berlin.de/Image:NAT_full_cone.png}}
\end{figure}

\paragraph{Restricted Cone NAT}

Das Verhalten des Restricted Cone NATs~\ref{fig:nat-restricted-cone}
ist dem des Full Cone NATs~\ref{fig:nat-full-cone} nahezu
identisch. Es unterscheidet sich lediglich dadurch, dass nur eine
Kommunikation zu genau dem Remote Hosts erfolgen darf, zu dem der
initiale Request gesendet wurde.  Alle anderen, eventuellen Pakete von
Remote Hosts, würden je nach Konfirguration des NAT-Routers
\texttt{gedroped} oder \texttt{rejected}.

\begin{figure}
  \centering
  \includegraphics[width=9cm]{images/nat_restricted_cone.png}
  \caption{Restricted Cone NAT}
  \label{fig:nat-restricted-cone}
  \small{Quelle: \url{http://sarwiki.informatik.hu-berlin.de/Image:NAT_restricted_cone.png}}
\end{figure}

\paragraph{Port Restricted Cone NAT}

Diese Klassifizierung des NATs~\ref{fig:nat-port-restricted-cone}
agiert noch restriktivier, als die des Restricted Cone NATs, indem vom
Remote Host zusätzlich gefordert wird, dass jegliche Antwort vom
angegebenen Ziel-Port aus erfolgt.

\begin{figure}
  \centering
  \includegraphics[width=9cm]{images/nat_port_restricted_cone.png}
  \caption{Port Restricted Cone NAT}
  \label{fig:nat-port-restricted-cone}
  \small{Quelle: \url{http://sarwiki.informatik.hu-berlin.de/Image:NAT_port_restricted_cone.png}}
\end{figure}

\paragraph{Symmetric NAT}

Das Symmetric NAT~\ref{fig:nat-sym} verwendet die höchste
Restriktion. Hier wird zu jeder angeforderten verbindungsorientierten
oder verbindungslosen Kommunikation des Private Hosts A zu einem
Remote Host B, ein anderer gemappter Port als der Source-Port, auf dem
NAT-Router benutzt (vergleiche hierzu Full Cone NAT).

\begin{figure}
  \centering
  \includegraphics[width=9cm]{images/nat_symmetric.png}
  \caption{Symmetric NAT}
  \label{fig:nat-sym}
  \small{Quelle: \url{http://sarwiki.informatik.hu-berlin.de/Image:NAT_symmetric.png}}
\end{figure}


\subsubsection{Hole Punching}
Beim Hole Punching unterscheidet man zwischen UDP Hole Punching und
TCP Hole Punching. Da UDP auf einer verbindungslosen Kommunikation
beruht und dadurch das gängigere Hole Punching Verfahren beschreibt,
wird im Folgenden darauf eingegangen.

UDP Hole Punching erlaubt es zwischen zwei Rechnern eine direkte
peer-to-peer UDP Session aufzubauen. Unter Zuhilfenahme eines
Rendezvous Servers ist es darüber hinaus möglich eine Session
aufzubauen, obgleich sich beide Rechner hinter einem NAT befinden.
Diese Technik wurde laut RFC 3027 (Protocol Complications with the IP
Network Address Translator) bereits vor dem heute bekannten STUN
(siehe~\ref{sec:stun}) durch das proprietäre \texttt{Activision gaming
  protocol} angewandt.  Auch ohne Verwendung eines Rendezvous Servers
kann man das Prinzip und die Technik des Hole Punchings beispielhaft
aufzeigen.
\paragraph{Hole Punching ohne Rendezvous Server}
Im Folgenden sei ein Host A und Host B gegeben. Host B ist dabei
hinter einem Port Restricted NAT~\ref{fig:nat-port-restricted-cone}.
Wenn Host A eine Verbindung zu B aufbauen soll ist dies auf Grund des
NATs nicht möglich, da kein Port in das LAN zu Host B weitergeleitet
wird. Selbst wenn Host B einen Listener~\ref{lst:netcat} durch
\texttt{netcat} auf Port 69 UDP laufen läßt, ist dieser vom WAN nicht
erreichbar, da dass NAT dies nicht zulässt.

\begin{lstlisting}[language=bash]
nc -u -l -p 69
\end{lstlisting}
\label{lst:netcat}

Wenn nun Host B ein UDP-Datagramm von Port 69 zu Host A
(74.125.39.104) schickt
\begin{lstlisting}[language=bash]
hping3 -c 1 -2 -s 69 -p 54 74.125.39.104
\end{lstlisting}
\label{lst:hping}
wird dadurch genau Port 69 zu Host B in das LAN geöffnet, falls es
sich um ein \texttt{(Port) Restricted Cone NAT }handelt.  Versucht
darauf hin Host A nochmals ein UDP-Datagramm zu Host B zu schicken
gelingt dies, da Host B diesen Port selber in das eigene LAN geöffnet
hat.

\paragraph{Hole Punching mit Rendezvous Server}
\label{sec:hole-rendezvous}
Die weitaus praktikablere Variante des Einsatzes des UDP Hole
Punching, ist die unter der Verwendung eines Rendezvous Servers.  Der
Rendezvous Server agiert in dem beispielhaften Szenario als
Mittelsmann zwischen zwei Hosts A und B, welche jeweils hinter einem
NAT in einem LAN eingebunden sind. Abbildung~\ref{fig:rendezvous} soll
diesen Sachverhalt visualisieren.

\begin{figure}
  \centering
  \includegraphics[width=6cm]{images/rendezvous.png}
  \caption{Zwei Hosts, jeweils hinter NAT}
  \label{fig:rendezvous}
\end{figure}

Diese Konstellation von zwei verschiedenen Hosts/Clients und einem
Server ist ein Beispiel, welches in vielen Audio-Anwendungen in Form
von VOIP-Software wiederzufinden ist. Eines der bekannteren Programme
ist sicherlich Skype.  Skype, wie auch andere peer-to-peer-Anwendungen
bedienen sich genau dieser Technik und \textit{bohren} dadurch Löcher
in die eigene Firewall und übergehen ebenfalls multiple NATs.

Angenommen Client A möchte Client B anrufen. Beide sind schon jeweils
am Rendezvous Server (in diesem Fall der Skype Server) mit ihren Daten
angemeldet. Der Skype Server weiss somit wie die jeweils öffentlichen
IP-Adressen von Client A und B sind, sowie über welchen Port jeweils
die Audiodaten von Client A und B übertragen werden würden, da dies im
Anwendungsclient eingestellt ist.

Sobald Client A dem Skype Server seinen Wunsch mitgeteilt hat Client B
anzurufen, initiiert dieser einen Sessionaufbau von Client B zu
A. Client B öffnet dadurch einen Port in der eigenen Firewall
bzw. leitet alles was an diesen Port des NATs adressiert ist und von
der öffentlichen Adresse des Clients A kommt, an die private Adresse
des Clients B weiter.

Der Versuch des Sessionaufbaus von Client B zu Client A wird an der
Firewall oder an dem NAT des Clients A scheitern, da diese(r) an den
Port adressierte Pakete dropen oder rejecten wird. Doch ab diesen
Zeitpunkt kann Client A, durch die gewonnenen Informationen des Skype
Servers über Client B selber die Session beginnen, indem er die
öffentliche Adresse von B, sowie den von Client B selber geöffneten
Port als Ziel angibt. Das Gespräch kann somit zu stande kommen.

Diese Technik wird unter anderem auch als RFC-Standard unter dem Namen
STUN eingesetzt.

\subsection{STUN}
\label{sec:stun} Früher \texttt{Simple traversal of UDP through NATs}
durch RFC 3489 definiert, steht STUN heute für \texttt{Session
Traversal Utilities for NATs} (RFC 5389) und ist der Standard für
VOIP-Anwendungen wie sip-basierte Telefonie. Es bietet eine
Client-Server Struktur und basiert auf \texttt{requests} der Clients,
welche durch STUN-\texttt{responses} beantwortet werden. Viele grosse
Anbieter wie sipgate.de haben öffentliche STUN-Server welche das
unter~\ref{sec:hole-rendezvous} exemplarisch gezeigte Vorgehen
unterstützen und genau dafür ausgelegt sind. Da sich das eigentliche
Vorgehen bei der Verwendung von STUN nicht wesentlich von dem Beispiel
abhebt, soll an dieser Stelle nicht weiterführend darauf eingegangen
werden. Nichtsdestotrotz zeigt es die aktuelle Praktikabilität.

\subsection{Nutzen für FIDIUS} Unter Punkt~\ref{sec:sw-solution} wurde
bereits herausgestellt, dass die Verwendung von VPN-Software schwierig
zu realisieren sein wird, da Installations- und Konfigurationsprozesse
zu komplex für kompromittierte Rechner wirken, um Brückenköpfe
aufzubauen. Herausgestochen bei der Software ist \texttt{campagnol}
durch die Verwendung der Technik des Hole Punchings.

Hole Punching ist als Technik dann für \f interessant, falls NATs
oder Firewalls überwunden werden müssen, ohne Konfigurationsprozesse
durchführen zu müssen. In wie weit es konkret einsetzbar ist, wird
sich im Laufe der zweiten Halbzeit des Projektes zeigen.


\subsection{Quellen} 
\cite{rfc3027},
\cite{rfc3489},
\cite{rfc5389},
\cite{tunnel-p2pnat},
\cite{tunnel-skype},
\cite{tunnel-nat-traversal},
\cite{en-wiki:udp-hole},
\cite{en-wiki:nat-hole}
 % CA: Tunnel
\clearpage\section{Zeitreihen}
\label{sec:compositions:timeanalysis}
\subsection{Einführung} Diese Ausarbeitung wird einen Einstieg in das
Thema Zeitreihenverarbeitung geben.

In vielen Domänen ist die zeitliche Betrachtung von Daten von hoher
Bedeutung. Bei der Analyse von Aktienmärkten, DNA Sequenzen und
Sprache findet man zeitlich zusammenhängende Daten ( Zeitreihen ), die
nicht punktuell betrachtet werden können. Auch in Rechnernetzen trifft
man auf zeitlich zusammenhängende Daten wie Paketströme, Anzahl von
Anfragen ( gemessen über die Zeit ) oder Anzahl der TCP Pakete (
gemessen über die Zeit ).

Um eine Einführung in die Verarbeitung solcher Daten zu geben wird der
Begriff der Zeitreihe formal eingeführt und gegen punktuelle
Lernverfahren abgegrenzt. Grundlegende statistische Verfahren zur
Verarbeitung von Zeitreihen wie ``Sampling'', ``Sliding Window'',
``Sample Mean'' und ``Sample Variance'' werden erläutert.

Nachfolgend wird das Distanzmaß ``Dynamic Time Warping'' (DTW)
vorgestellt und die Möglichkeit Zeitreihen damit zu klassifizieren (
Nächste Nachbarn / SVM mit DTW Kernel ) grundlegend dargestellt. Zum
indizieren und finden werden die Zeitreihenrepräsentation ``Symbolic
Aggregate approXimation'' und ``indexable Symbolic Aggregate
approXimation'' (SAX / iSax) vorgestellt mit denen es möglich ist
große Mengen Zeitreihen zu durchsuchen.

\subsection{Zeitreihen und Statistik} In diesem Abschnitt sollen die
grundlegenden statistischen Verfahren auf Zeitreihen eingeführt werden
für eine genauere Einführung sei auf \cite{ts, gama} verwiesen.  Eine
Zeitreihe beschreibt eine Reihung realwertiger Zahlen im zeitlichen
Kontext definiert als (\cite{shieh08}):

\begin{align}
  X = x_1, x_2, \ldots,  x_t \text{ mit $x_i \in \mathbb{R}$ und der
    Zeitpunkt $t \in \mathbb{N}$ }
\end{align}

Um solch eine Zeitreihe zu erhalten wird oft \textbf{Sampling}
genutzt.  Sampling beschreibt periodisches Messen einer Variable.
Beispiele sind stündliches Messen der Temperatur oder das stündliche
Messen der Anzahl der UDP Pakete in einem Subnetz.

Da oft nicht alle vorherigen Ereignissen interessant sind, sondern nur
die in der nahen Vergangenheit, wird oft ein \textbf{Sliding Window}
genutzt. Solch ein Fenster ist ein Bereich fester Größe (mit
$length(window) << length(data)$) der durch den Datensatz elementweise
geschoben wird. Es wird folglich pro resultierendem Fenster
ausgewertet (siehe Beispiel~\ref{img:sw}).

\begin{figure} [ht]                                   
\centering
\includegraphics[width=0.5 \textwidth]{images/sliding_window}  
\caption{Eine Zeitreihe $\{1,2,3,4,5,6,7,8,1\}$der Länge 8 und ein
  Sliding Window der Länge 6.}         
\label{img:sw}
\end{figure}

Auf diesen Fenstern können nun ``lokale'' Statistiken geführt werden
(lokal da nur ein Ausschnitt betrachtet wird und nicht die gesamte
Zeit).  Zwei der häufigsten Statistiken sind der Mittelwert $\mu$
(eng. mean) und die Varianz $\sigma^2$(eng. variance). Der Mittelwert
über ein Fenster ist definiert als:

\begin{align}
\mu(X) = \frac{1}{N}\Sigma_{i = start} ^ {N} x_i \text{ mit } N = length(window)
\end{align}

Die Varianz gibt die Streuung oder Abweichung vom Mittelwert an und
ist definiert als:

\begin{align} \sigma^2(X) = \Sigma_{i = start} ^ {N} (x_i - \mu )^2
\end{align}

Für weiterführende Literatur empfehlen sich \cite{gama, ts}.

\subsection{Dynamic Time Warping Distanz}

Dynamic Time Warping (DTW) ist ein Distanzmaß für Zeitreihen ähnlich
der Levenstein oder Editierdistanz.  In Abbildung~\ref{img:dtw} a)
sind zwei beispielhafte Zeitreihen angegeben zu denen die DTW Distanz
gesucht ist.  Im ersten Schritt werden diese mit Dynamic Programming
justiert oder in Deckung gebracht (siehe~\ref{img:dtw} b) ).
Nachfolgend werden die justierten Zeitreihen so übereinander gelegt,
dass die Länge der Abweichungen (graue Linien in~\ref{img:dtw} c))
minimal ist. Die Distanz ergibt sich dann aus der Summe der
Abweichungen.


Die Varianz gibt die Streuung oder Abweichung vom Mittelwert an und
ist definiert als:

\begin{align} \sigma^2(X) = \Sigma_{i = start} ^ {N} (x_i - \mu )^2
\end{align}

Für weiterführende Literatur empfehlen sich \cite{gama, ts}.

\subsection{Dynamic Time Warping Distanz}

Dynamic Time Warping (DTW) ist ein Distanzmaß für Zeitreihen ähnlich
der Levenstein oder Editierdistanz.  In Abbildung~\ref{img:dtw} a)
sind zwei beispielhafte Zeitreihen angegeben zu denen die DTW Distanz
gesucht ist.  Im ersten Schritt werden diese mit Dynamic Programming
justiert oder in Deckung gebracht (siehe~\ref{img:dtw} b) ).
Nachfolgend werden die justierten Zeitreihen so übereinander gelegt,
dass die Länge der Abweichungen (graue Linien in~\ref{img:dtw} c))
minimal ist. Die Distanz ergibt sich dann aus der Summe der
Abweichungen.

\begin{figure}[ht]
  \centering
  \includegraphics[width= 0.5 \textwidth]{images/dtw}
  \caption{Berechnen der DTW Distanz, Übernommen mit Berechtigung von
    Daniel Ashbrook}
  \label{img:dtw}
\end{figure}

Formal definiert ist die Dynamic Time Warping Distanz wir folgt.  Für
zwei Zeitreihen $C = C_1, C_2, ..., C_n $ und $Q = Q_1, Q_2, ..., Q_m$
ist Dynamic Time Warping rekursiv definiert als \cite{wei08, keogh02}:
\begin{eqnarray} DTW([ ],[ ]) & = & 0 \\ DTW(C, [ ]) & = & \infty \\
DTW([ ], Q) & = & \infty \\ DTW(C,Q) & = & abs(First(C), First(Q)) \\
& & + min(\\ & & DTW(C, Rest(Q)) \text{ insertion} \\ & &
DTW(Rest(C),Q) \text{ deletion} \\ & & DTW(Rest(C), Rest(Q)) \text{
match}
\end{eqnarray}

Bildlich wird eine Distanzmatrix der Größe $n x m$ aufgebaut ( x-Achse
repräsentier $C$ und y-Achse $Q$) und der Pfad mit den minimalen
Kosten durch die Matrix gesucht (siehe~\ref{img:wp}).
\begin{figure}[ht] \centering
  \includegraphics[width= 0.5 \textwidth]{images/dtw2}
  \caption{Warping Path der DTW Distanz, Übernommen mit Berechtigung
von Eamonn Keogh\cite{keogh02}}
  \label{img:wp}
\end{figure}

\subsection{Klassifizierung mit Dynamic Time Warping}
\subsubsection{Nächster Nachbar} Nächster Nachbar (eng. Nearest
Neighbour) ist ein instanzbasiertes Verfahren zum Klassifizieren. Im
Gegensatz zu Klassifizierern wie Support Vector Machines oder naive
Bayes muss der Algorithmus nicht trainiert werden. Die
Trainingsbeispiele werden vorgehalten bis eine Instanz klassifiziert
werden soll. Für jede nicht klassifizierte Instanz gibt die
Klassifizierung des Trainingsbeispiels mit der minimalen Distanz zur
zu klassifizierenden Instanz die Klassifizierung an.  Für Zeitreihen
wird meist die Dynamic Time Warping Distanz genutzt.
\subsubsection{Support Vector Machines} % ref Eine generelle
Einführung in die Klassifizierung und das Trainieren von Support
Vector Machines befindet sich im ersten Halbjahresbericht des
Projektes \f. Grundlegend wird ein linearer Klassifikator im
sogenannten ``Kernel Space'' gelernt. Ein Kernel $K$ ist eine Funktion
die gegeben zweier Instanzen $x$ und $y$ so in $\mathbb{R}$ abbildet,
dass eine lineare Trennung möglich ist (Schreibweise: $K(x, y)$). Oft
verwendete Kernel sind der lineare Kernel $K(x,y) = x * y$ und der
polinomielle Kernel $K(x,y) = ( x * y) ^d$ (hier ist $*$ das
Scalarprodukt). Um Zeitreihen zu Lernen und Klassifizieren kann ein
Dynamic Time Warping Kernel verwendet werden. Seien hier $x$ und $y$
Zeitreihen (mit $length(x) = n$ und $length(y) = m$), dann ist der
Kernel $K(x,y) = DTW(x,y)$. Es wird also im Dynamic Time Warping Raum
gelernt \cite{svm_dtw}. Zu beachten sei, dass die SVM hier als eine
Art gewichteter instanzbasierter Klassifikator ist.

\subsection{Symbolic Aggregate Approximation} Symbolic Aggregate
approXimation (SAX) \cite{lin07} überführt eine Zeitreihe in eine
symbolische Approximation (einen String).  Da Suche oder Indizierung
von Strings eine gut verstandene Disziplin in Algorithmik und Bio-
Informatik sind, können bestehende Verfahren auf Zeitreihenanalyse
angewendet werden.  Die Transformation einer Zeitreihe in einen String
wird im Folgenden erläutert.

Eine Zeitreihe $X = x_1, x_2, ..., x_t$ wird zunächst mittels
Piecewise Aggregate Approximation (PAA) \cite{keogh00} auf eine
gewählte Länge $N$ kompremiert. Das Ergebnis ist die kompremierte
Zeitreihe $\bar{X} = \bar{x_1}, \bar{x_2}, ..., \bar{x_N}$ wobei für
$\bar{x_i}$ gilt:

\begin{align}
\bar{x_i} = \frac{t}{N} \Sigma_{j = t / N(i-1)+1}^{n/w*i} X_i
\end{align}

Folglich teilt die PAA die Zeitreihe $X$ in $N$ gleich große Blöcke
und speichert den Mittelwert pro Block.  Um die Zeitreihe in einen
String zu überführen wird zusätzlich zur Wortlänge eine Alphabetgröße
(genannt Kardinalität, eng cardinality) $\alpha = |\Sigma|$
eingeführt. $SAX(N, \alpha)$ teilt die Y-Achse in $\alpha$
Regionen. Jeder PAA - Wert wird durch das Symbol ersetzt, das mit der
Region assoziiert ist, in der dieser Wert fällt.  Diese Regionen sind
so gewählt, das die Wahrscheinlichkeit in jeder Region gleich ist.
Für gleich verteilte Zeitreihen ist dies $\frac{1}{\alpha}$ und für
normalverteilte Zeitreihen eine Region, sodass die Fläche unter einer
Gausskurve $\frac{1}{\alpha}$ ist. Die Regionen für Normalverteilungen
können in statistischen Tabellen nachgeschlagen werden.  Das Ergebnis
ist ein String $S = s_1,..., s_N$ mit $s_i \in \Sigma$.  Sei unsere
Zeitreihe beispielsweise: $X = [1,4,1,4,1,2,1,2,6,8]$ und unsere
Alphabetlänge $\alpha = 3$ (also $\Sigma = \{A,B,C\}$).  Die Anfänge
der Regionen seien $[1,2,7]$ (in diesem Beispiel weder Normal noch
Gaussverteilt). Die $PAA(5,X)$ ist dann $[2.5,2.5,1.5,1.5,7]$ und der
String ``BBAAC''.

Der gesamte Prozess ist in Abbildung~\ref{img:sax} abgebildet.
\begin{figure}[ht] \centering
\includegraphics[width= 0.8\textwidth]{images/sax_minnen}
\caption{Von der Zeitreihe zu SAX. Übernommen mit Berechtigung von
David Minnen}
\label{img:sax}
\end{figure}

\subsection{indexable Symbolic Aggregate Approximation} Um zu einer
Zeitreihe schnell nächste Nachbarn zu finden soll iSAX, als
Erweiterung zu SAX, vorgestellt werden.  Die Idee ist alle Zeitreihen
in SAX zu transformieren und den resultierenden String als Index zu
nutzen. Zeitreihen mit dem gleichen Index werden dann unter diesem
gemeinsam gehalten. Meist wird dies durch Dateien mit dem Index als
Namen und den Zeitreihen als Inhalt realisiert.  In iSax
\cite{shieh08, shiehPHD} wird zusätzlich zu den SAX Parametern noch
eine sogenannte ``bucket size'' angegeben. Diese gibt an wie viele
Zeitreihen pro Index gehalten werden können. Übersteigt die Anzahl der
Zeitreihen in einem Index diesen Schwellwert wird die Alphabetgröße an
einer Position des Strings (mittels Round Robin) erhöht, die Datei
durch einen Ordner mit gleichem Namen ersetzt und alle Zeitreihen
werden mit der höheren Alphabetgröße( in dem Ordner) neu indiziert.
Durch das erhöhen der Alphabetgröße an einer Position werden die
Zeitreihen besser unterscheidbar, da es nun an einer Position mehr
Symbole und daher schmalere Regionen gibt.

Aus diesem Verfahren ergibt sich dann ein Suchbaum~\ref{img:sax_tree}.
\begin{figure}[ht] \centering
\includegraphics[width= 0.8\textwidth]{images/isax_tree}
\caption{Ein iSAX Suchbaum }
\label{img:sax_tree}
\end{figure}

Um zu einer Zeitreihe nächste Nachbarn zu finden, wird diese in SAX
Form unter den Bedingungen des Wurzelknotens transformiert und dann
den Indizes rekursiv gefolgt, bis eine Enddatei erreicht ist. In
dieser befinden sich dann die Kandidaten.

\subsection{Zusammenfassung} In dieser Ausarbeitung wurden
grundlegenden Statistiken und Definitionenen von Zeitreihen
beschrieben.  Außerdem wurde Dynamic Time Warping als Distanzmaß und
Basis für Klassifikatoren eingeführt. Im letzten Abschnitt wurde eine
Transformation von Zeitreihen in Strings beschrieben mit der es
möglich ist Suchindizes für große Datensätze aufzubauen.

 % DK: Algorithmen
