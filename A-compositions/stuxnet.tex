\section{Stuxnet}
\label{compositions:stuxnet}

\authors{\DM}{}

Die Folien zu diesem Referat sind im Projekt-Wiki unter
\url{http://rn.informatik.uni-bremen.de/wiki-fidius/Stuxnet?action=AttachFile&do=view&target=Referat-Stuxnet.pdf}
zu finden.\footnote{ggf. Zugangsdaten benötigt}

\subsection{Überblick}

Im Juni 2010 tauchte ein Computerwurm auf dem Radar der
Anti-Viren-Hersteller auf, der ein bisschen anders war als andere
Würmer. Es werden nicht -- wie sonst üblich -- Botnetze aufgebaut, die
dann Spam versenden oder Kreditkarteninformationen ausspähen. Auch die
Verbreitung des bald Stuxnet getauften Wurmes geschieht offenbar nicht
willkürlich, wenn man sich die Analysen der IP-Adresse der infizierten
Hosts ansieht (vgl. Abbildung~\ref{fig:comp:stuxnet:hosts}).
Schließlich beeindruckt die Gesamtzahl der ausgenutzten
Schwachstellen, insbesondere, weil dort viele so genannte
Zero-day-Exploits auf einmal ausgenutzt werden.

\begin{figure}[hp]
  \centering
  \includegraphics[width=\linewidth]{./images/comp-stuxnet-hosts.png}
  \caption[Geographische Verteilung der infizierten Hosts]
    {Geographische Verteilung der infizierten Hosts, Analysen von [Symantec]}
  \label{fig:comp:stuxnet:hosts}
\end{figure}

Nach weiteren Analysen, hier seien die von Symantec und ESET besonders
hervorzuheben, zeichnete sich ein neues Bild dieses Wurms. So
operierte Stuxnet offenbar schon ein Jahr verdeckt und nutzt bis dahin
unbekannte Techniken, um unerkannt in ein System einzudringen (vgl.
Tabelle~\ref{tab:comp:stuxnet:history}).


\begin{tabelle}{.175,.8}{Zeitlicher Ablauf (teilw. nachdatiert)}{tab:comp:stuxnet:history}
  \textbf{Datum} & \textbf{Ereignis} \\\endheadline
  
  Nov. 2008   & erste Trojaner-Variante, welche die Win-LNK-Lücke ausnutzt
                (später MS10-046) \\
  Juni 2009   & früheste gesehene Stuxnet-Variante (ohne signierte Treiber und
                MS10-046-Exploit) \\
  25.01.2010  & Stuxnet-Variante mit digitaler Unterschrift von Realtek \\
  März 2010   & Stuxnet-Variation nutzt MS10-046 aus \\
  17.06.10    & Stuxnet wird von \textit{VirusBlokAda} entdeckt; LNK-Lücke wird
                aufgedeckt (CVE-2010-2568) \\
  16.07.10    & von Microsoft kommen Informationen zu einer Schachstelle beim
                Verarbeiten von \textit{shortcut icons}/.lnk-Dateien, die
                Remote-Code-Execution ermöglicht\\
  16.07.10    & Verisign annuliert Realtek-Signatur \\
  17.07.10    & ESET entdeckt Stuxnet-Variante mit JMicron-Signatur \\
  19.07.10    & Siemens sorgt sich um WinCC-SCADA-Systeme infizierende Malware \\
  20.07.10    & Symantec beobachtet C\&C-Traffic \\
  22.07.10    & Verisign annuliert JMicron-Signatur \\
  02.08.10    & Microsoft patcht MS10-046 \\
  12.10.10    & Microsoft schließt weitere von Stuxnet genutzte Lücken
\end{tabelle}

\subsection[Exkurs: Speicherprogrammierbare Steuerungen] {Exkurs:
Speicherpro-\\grammierbare Steuerungen}

Um das Angriffsziel von Stuxnet verstehen zu können, soll an dieser
Stelle ein kurzer Abstecher in die Industrieautomatisierung und
Prozesssteuerung gemacht werden. Im Speziellen sollen hier SCADA und
PLC erklärt werden.

\subsubsection{SCADA}

Mit \emph{Supervisory Control and Data Acquisition}, kurz SCADA (bei
Siemens \enquote{SIMATIC WinCC}), wird ein Konzept zum Überwachen und
zur Steuerung insbesondere von Industrieanlagen beschrieben. Im
Prinzip visualisieren hierbei digitale Bildschirmanzeigen die zuvor
anaolgen (bzw. elektrisch geregelten) Anzeigen. Darüber hinaus erlaubt
dieses System auch, die in den Prozess eingebundenen Steuerungen
anzusprechen und zu manipulieren. Ein Beispiel einer solchen Anzeige
ist in Abbildung~\ref{fig:comp:stuxnet:scada} zu sehen.

\begin{figure}
  \centering
  \includegraphics[width=.7\linewidth]{./images/comp-stuxnet-scada.png}
  \caption[Typisches SCADA-Interface]
    {Typisches SCADA-Interface, [Wikimedia Commons]}
  \label{fig:comp:stuxnet:scada}
\end{figure}

Ein direkter Vorteil eines solchen Systems ist die natürliche
Darstellung von Prozessketten. Standardisierte Icons erlauben, sich
schnell einen Überblick über ein System zu verschaffen. Von Nachteil
ist dafür das Fehlen von mehreren, über die Anlage verteilten analogen
Anzeigen, welche zwar bei eventuellen widersprüchlichen Ausgaben
Notfallmaßnahmen nicht mehr behindern können, dafür aber ein volles
Vertrauen in die virtuellen Anzeigen verlangen. Die rein digitalen
Anzeigen bieten also eine Angriffsfläche, dessen Manipulation gar
nicht bemerkt werden kann.

\subsubsection{PLC}

Die im vorherigen Abschnitt genannten \emph{Steuerungen} stellen so
genannte \textit{Programmable Logic Controller} (PLC;
dt. Speicherprogrammierbare Steuerungen (SPS)) dar. Diese
Kleinstcomputer nehmen über Eingänge Messungen vor, verarbeiten diese
und regeln, wenn nötig, Ventile und ähnliches.

Über eine Ethernet-Schnittstelle\footnote{bzw. einem
sog. PROFINET-Adapter} kann eine solche Steuerung mit SCADA-Terminals
kommunizieren und von dort Befehle oder neue Verarbeitungsprogramme
empfangen. In Siemens SIMATIC-Anlagen übernimmt das Programm
\emph{WinCC} die Anzeige und Überwachung von PLCs, und mit einer
\emph{Step7} genannten Entwicklungsumgebung werden diese PLCs
programmiert.

\subsubsection{Anwendungen}

Wo immer technische Prozesse überwacht, Materialzuflüsse geregelt,
Temperaturen, Drucke oder Füllstände gemessen werden müssen, setzt man
heute PLCs ein.

Beispielanwendungen sind

\begin{itemize}
  \item Klärwerke/Wasserkraftwerke
  \item Herstellung von Waschmittel, Folien, chemische Werkstoffe
  \item Tunnelbohrmaschinen
  \item Urananreicherungsanlagen
\end{itemize}

sowie weitere.

\subsubsection{Ziel von Stuxnet}

Die Analysen von Stuxnet zeigen nun, dass Stuxnet nach ganz bestimmten
Systemkonfigurationen Ausschau hält. So bricht die Installation von
Stuxnet ab, wenn es keine WinCC-Installation\footnote{Siemens nennt
seine SCADA-Umsetzung Simatic WinCC} vorfindet, und die Verbreitung
stoppt nach 21 Tagen, wenn es feststellt, dass über den aktuell
infizierten Host keine Siemens-PLCs zu erreichen sind.

Dies allein zeigt schon, dass Stuxnet mindestens in die Kategorie der
gezielten Industriespionage fällt. Diese Ausarbeitung wird noch
zeigen, dass man durchaus von Industriesabotage sprechen muss.

\subsection{Aufbau von Stuxnet}

\subsubsection{Innere Organisation}

Im Wesentlichen besteht Stuxnet aus einer etwa 500\,kB großen
DLL-Datei. Darin enthalten sind die so genannten \textit{exports}
(also die von außen aufrufbaren, exportierten Funktionen) und
\textit{resources} (Konfigurations- und weitere DLL-Dateien,
Datei-Templates, (signierte) Treiber-Dateien und Strings). Wenn im
Folgenden von der \enquote{Stuxnet-DLL} gesprochen wird, ist damit
diese Datei gemeint.

Diese \textit{exports} und \textit{resources} sind nicht-linear
nummeriert und sollen im Folgenden nur zur Identifikation dienen. Eine
vollständige Liste aller \textit{exports} und \textit{resources}
findet sich sowohl im Symantec"=% Stuxnet"=Dossier, als auch im
Bericht von ESET.

\subsubsection{Kontrollfluss}

\paragraph{Windows-LNK-Lücke}

Stuxnet kennt verschiedene Wege, in ein System einzudringen. Die in
der Presse am meisten beachtete Variante, einen Windows-Rechner zu
kapern, nutzt die Windows-LNK-Lücke\footnote{CVE-2010-2568, MS10-046}
aus.

Hierbei handelt es sich um ein Fehler beim Verarbeiten von
\enquote{Verknüpfungen} (\texttt{lnk}-Dateien;
\textit{Link}). Speziell präparierte \texttt{lnk}-Dateien erlauben das
dynamische Laden von verschiedenen Icons. Gedacht ist diese Funktion
für \textit{Windows"=Systemsteuerungselemente}, die beispielsweise
abhängig von Gerätestati unterschiedliche Symbole anzeigen.  Damit das
wiederum funktioniert, muss zum Systemsteuerungsicon ein ausführbares
Programm hinterlegt sein, das beim Laden ausgeführt werden soll.

Da sich unter Windows die Systemsteuerung im Prinzip nicht von
ordinären Verzeichnissen unterscheidet, muss die Funktion vom
Nachladen von Code beim Anzeigen von Symbolen auch ausserhalb der
Systemsteuerung verfügbar sein. Und tatsächlich erlaubt der Header der
\texttt{lnk}-Format-Spezifikation die Angabe einer nachzuladenden
Bibliothek anzugeben, sofern eine bestimmte
\textit{Class-ID}\footnote{in der Windows-Regestry hinterlegte
GUID/UUID zum identifizieren von Systemobjekten} das Icon als
Systemsteuerungselement auszeichnet (siehe
Abbildung~\ref{fig:comp:stuxnet:lnk-vulnerability}, Punkt 1).  Findet
der \textit{Windows-Explorer}\footnote{oder ein anderes, Icons
anzeigendes Programm} nun ein solch präpariertes Symbol, wird eine
Funktionskaskade aufgerufen, die in der System-Funktion
\texttt{LoadLibraryW()}
(Abb.~\ref{fig:comp:stuxnet:lnk-vulnerability}, Punkt 2) mündet,
welcher der Pfad zu der DLL übergeben wird, die im \texttt{lnk}-Header
angegeben ist (in diesem Fall die Stuxnet-DLL).

Somit kann der Angreifer Code ausgeführen, ohne dass der Benutzer
etwas tun muss. Diese Lücke kann zudem nicht nur von USB-Sticks oder
anderen Datenträgern ausgenutzt werden, sondern auch via
Samba-Netzwerkfreigaben.

\begin{figure}
  \centering
  \begin{tikzpicture}[
    inner/.style={text width=7em,inner sep=2pt,anchor=center,text centered,minimum height=1em},
    box/.style={draw,very thin},
    proc/.style={box,inner sep=2pt,rounded corners=4pt}
  ]
    \node[inner]
      (lnk format) at (0,2em)
      {\textbf{.LNK File Format}};
    \node[inner,box,below=1pt of lnk format]
      (header)
      {Header};
    \node[inner,box,below=0pt of header]
      (id list)
      {Shell Item Id List};
    \node[inner,box,below=0pt of id list]
      (file info)
      {File Location Info};
    \node[inner,box,below=0pt of file info]
      (tmp)
      {Description};
    \node[inner,box,below=0pt of tmp]
      (tmp)
      {Relative Path};
    \node[inner,box,below=0pt of tmp]
      (tmp)
      {Working Directory};
    \node[inner,box,below=0pt of tmp]
      {…};
    \node[box,inner sep=1pt,right=10pt of file info] (clsid)
      {CLSID\_MyComputer};
    \node[box,inner sep=1pt,yshift=-2.6ex,right=10pt of file info] (media)
      {\char`\\\char`\\$\langle$path magic$\rangle$\char`\\foobar.dll};
    \node[below=1pt of media,xshift=-3em] {(1)};
    \draw[latex'-,densely dotted] (file info.0) -- (clsid.180);
    \draw[latex'-,densely dotted] (file info.0) -- (media.180);
    \node[proc]
      (findcplinfo) at (5.5,0)
      {\texttt{CPL\_FindCPLInfo()}};
    \draw[->,densely dashed,shorten >=1pt] (clsid) |- (findcplinfo);
    \node[proc,xshift=4em,below=1ex of findcplinfo]
      (loadandfindapplet)
      {\texttt{CPL\_LoadAndFindApplet()}};
    \node[proc,xshift=3em,below=1ex of loadandfindapplet]
      (loadcplmodule)
      {\texttt{\_LoadCPLModule()}};
    \node[proc,xshift=3em,below=1ex of loadcplmodule]
      (loadlibraryw)
      {\texttt{LoadLibraryW()}};
    \node[below=1pt of loadlibraryw,xshift=1em] {(2)};
    \draw[->] (findcplinfo.195) |- (loadandfindapplet);
    \draw[->] (loadandfindapplet.200) |- (loadcplmodule);
    \draw[->] (loadcplmodule.200) |- (loadlibraryw);
    \draw[->,thin] (media) |- (loadlibraryw);
  \end{tikzpicture}
  \caption[Eintrittsvektor Windows-LNK-Lücke]
    {Eintrittsvektor Windows-LNK-Lücke, [ESET, S.\,22ff.]}
  \label{fig:comp:stuxnet:lnk-vulnerability}
\end{figure}

\paragraph{Systemrechte erlangen}

\begin{figure}[p]
  \centering
  \includegraphics[width=\linewidth]{./images/comp-stuxnet-export-15.png}
  \caption{Kontrollfluss der Stuxnet-Installationsroutine (Export 15)}
  \label{fig:comp:stuxnet:export15}
\end{figure}

\begin{figure}[p]
  \centering
  \includegraphics[width=\linewidth]{./images/comp-stuxnet-export-16.png}
  \caption{Kontrollfluss der Stuxnet-Infektionsroutine (Export 16)}
  \label{fig:comp:stuxnet:export16}
\end{figure}

Die Stuxnet-DLL startet mit der Ausführung des Export 15 (dessen
Kontrollfluss ist in Abbildung~\ref{fig:comp:stuxnet:export15}
dargestellt). Hier werden zunächst einige Systemchecks durchgeführt
(Version und Bitigkeit des OS) und es wird überprüft, ob der aktuelle
Stuxnet-Prozess bereits mit Rechten des Systembenutzers läuft.

Ist dies nicht der Fall, wird für Windows 2000 und XP eine
Schwachstelle im für die Tastaturlayouts verantwortlichen
Systemtreiber \texttt{win32k.sys} ausgenutzt, um Systemrechte zu
erlangen. Um diese Schwachstelle auszunutzen, wird zunächst eine
präparierte Tastaturlayout-Datei durch \texttt{win32k.sys}
geladen. Eine ungültige Byte-Folge in dieser Datei führt nun beim
Parsen dazu, dass sich bestimmte Variablen innerhalb der
\texttt{win32k.sys} mit Adressdaten füllen, die auf einen von Stuxnet
kontrollierten Speicherbereich zeigen. Durch einen
\textit{use-after"=free}"=Fehler springt \texttt{win32k.sys}
schließlich in diesen Bereich und führt Stuxnet mit Systemrechten aus.

Unter Windows Vista und 7 nutzt Stuxnet einen Fehler im \textit{Task
Scheduler} aus. Hierzu liegen aber noch keine Informationen vor, da
sowohl Symantec als auch ESET \textit{full disclosure} betreiben und
erst dann Informationen bereitstellen wollen, wenn Microsoft diese
kritische Lücke gestopft hat.

\paragraph{Selbstschutz\dots}
\label{sec:comp:stuxnet:protection}

Sofern Stuxnet endlich maximale Rechte erworben hat, erstellt es aus
einer Resource und der kompletten Stuxnet-DLL eine ausführbare Datei,
die es in ein laufendes Anti-Viren-Programm oder den
Windows-Systemdienst injiziert (je nach Verfügbarkeit wählt Stuxnet
aus mehreren Anti-Viren-Programmen) und dort zur Ausführung bringt. Im
Kontext dieser Programme beginnt Stuxnet damit, sich durch die
Installation zweier Rootkits vor der Entdeckung zu schützen.

Der Selbstschutz-Mechanismus besteht im Wesentlichen aus einem digital
von \textit{Realtek Semiconductor Corp.} signiertem Systemtreiber,
welcher die von Stuxnet generierten Dateien (\texttt{lnk}-Dateien
bestimmter Größe und \texttt{tmp}-Dateien mit bestimmtem Namen) vor
dem Zugriff schützt und dabei als Low-Level-System-Hook auf
Dateisystemebene fungiert.

Darüber hinaus sorgt Stuxnet mit einem weiteren digital signiertem
Treiber dafür, dass die Stuxnet-DLL auch bei jedem Windows-Start
geladen wird, und in Prozessen eingeschleust wird, die zum anvisierten
Ziel gehören:

\begin{itemize}
  \item Die ausführbare Datei \texttt{s7tgtopx.exe} gehört zum
\textit{SIMATIC Manager}, also einem zentralen Programm zur Verwaltung
von SPS-Projekten.
  \item Die Datei \texttt{CCProjectMgr.exe} (WinCC-Projektmanagement)
bearbeitet Steuerprogramme, die schließlich mit demselben Tool auch
auf die Siemens-SPS geschrieben werden.
\end{itemize}

Die Injektion in diese beiden (und weitere) Prozesse soll
sicherstellen, dass auch der zuvor angelegte Treiber garantiert wieder
geladen wird, und die Anti-Viren"=Signaturerkennung fehlschlägt,
sollte es einmal ein Update der Signaturdatenbank gegeben haben.

Der zuletzt genannte Treiber weist ein erwähnenswertes Merkmal auf:
als Verisign die von Realtek ausgestellte Signatur am 16.07.2010 für
ungültig erklärte, tauchte bereits am 17.07.2010 eine Stuxnet-Variante
mit einer neuen Signatur von JMicron Technology auf. Die örtliche Nähe
der Büroräume von Realtek und JMicron im gleichen taiwanesischen
Industriepark legt einen vorbereitenden physichen Einbruch nahe.

\paragraph{\dots\ und Selbstverbreitung}

Nachdem sicher gestellt ist, dass Stuxnet unbemerkt arbeiten kann,
beginnt es, sich auf vier verschiedenen Wege zu verbreiten:

\begin{enumerate}[label=\textbf{\arabic*.}]
  \item \textbf{Peer-to-Peer-Netze.}  Stuxnet bringt einen RPC-Server
und -Client mit, und lauscht auf eingehende RPC-Nachrichten
bzw. verschickt diese im lokalen Netz. Diese Methode dient vor allem
dazu, die Versionen auf den jeweils infizierten Hosts zu vergleichen
und die ältere Version zu aktualisieren.
  
  \item \textbf{Offener MS SQL-Server.}  Bei der Installation von
WinCC wird ein Microsoft-SQL-Server installiert.  Dabei werden jedoch
die Zugangsdaten nicht durch zufällige Daten ersetzt, sondern es
werden die fest verdrahteten Datenbank-Benutzer und "~Passworte
weiterverwendet. Dadurch ist es Stuxnet möglich, den SQL-Server von
außen anzusprechen und mit mit vordefinierten \textit{Stored
Procedures} zuerst die Stuxnet-DLL als hexadezimalkodierten String in
einer Tabelle abzulegen, diesen String dann in eine Datei zu schreiben
und schließlich mit einem \texttt{EXEC}-Statement diese Datei
auszuführen. Dass der SQL-Server mit Systemrechten läuft, ist für
Stuxnet dabei nur von Vorteil.

  \item \textbf{Print Spooler Vulnerability.}  Ein Fehler im
Drucker-Subsystem (\texttt{spoolsvr.exe}) erlaubt es, beliebige
Dateien ins Systemverzeichnis \enquote{zu drucken}. Hier führt eine
mangelhafte Überprüfung der Parameter der Systemfunktion
\texttt{StartDocPrinter()} dazu, dass auch Benuter mit Gastrechten
über die \textit{Datei- und Druckerfreigabe} den Rechner befallen
können.
  
  \item \textbf{Conficker-Schwachstelle.}  Stuxnet nutzt auch eine
ältere Schwachstelle aus, die man schon zu Zeiten des Conficker-Wurmes
(erstes Auftreten etwa im Oktober 2008) kannte. Auch hier wird wieder
die Datei- und Druckerfreigabe ausgenutzt, um mit so genannten
\textit{Netzwerk-Jobs} eine Datei auszuführen, die zuvor über die
Standardfreigaben \texttt{ADMIN\$} und \texttt{C\$} auf ein System
geschleust wurde.
\end{enumerate}

Das primäre Ziel ist hierbei, die aktuelleste Stuxnet-Version in einem
Netz zu verteilen. Auf diese Weise können neue Stuxnet-Varianten auch
Hosts erreichen, die nicht direkt am Internet hängen, sondern -- wenn
überhaupt -- hinter diversen Firewalls abgeschirmt sind. Dadurch, dass
Stuxnet auch Datenträger infiziert, ließen sich theoretisch auch
Rechner erreichen, die überhaupt nicht mit anderen Rechnern verbunden
sind.

\subsection{Infektion von Siemens-Installationen}

Sobald Stuxnet in die ausführbaren WinCC- und Step7-Dateien injiziert
wurde (siehe~\ref{sec:comp:stuxnet:protection}), schreibt es in
weiteren ausführbaren Dateien die Funktionstabellen so um, dass die
darin enthaltenen Funktionen \texttt{CreateFileA()}
bzw. \texttt{StgOpenStorage()} auf Stuxnet-Wrapper zeigen. Diese
Funktionen werden beim Laden und Speichern von Step7-Projekten
ausgeführt und dienen Stuxnet dazu Projekte zu identifizieren, die
bestimmte Kriterien erfüllen:

\begin{itemize}
  \item Das WinCC-Projekt muss relativ aktiv genutzt werden. Stuxnet
erkennt inaktive Projekte durch den Zeitstempel des letzten Zugriffs
und setzt 42~Monate als Unterscheidungsmarke an.
  
  \item Dieses Projekt darf kein Projekt aus der
WinCC-Beispielsammlung sein.  Hier nutzt Stuxnet die Pfadangaben als
Indikator.
  
  \item Schließlich muss das Projekt eine \texttt{mcp}-Datei
(eigentlich ein Projekt-Metadaten-Container) an einer bestimmten
Stelle gespeichert haben.
\end{itemize}

In die \texttt{mcp}-Datei speichert Stuxnet eine verschlüsselte Kopie
seiner selbst, und in die Step7-Projektdateien wird ein DLL-Loader
platziert der diese \texttt{mcp}-Datei entschlüsselt und zur
Ausführung bringt. Somit verbreitet sich Stuxnet auch dann auf anderen
Hosts, wenn die Projektdateien bspw. aus einem Repository geholt oder
mit USB-Sticks verteilt werden, die nicht von der Windows-LNK-Lücke
betroffen sind.

\subsubsection{Step7/SPS-Rootkit}

In einem der letzten Schritte vor dem Ziel schiebt Stuxnet der
Step7-Installation eine DLL unter, die sich in die Kommunikation
zwischen Step7 und dem PLC einklinkt bzw. genauer die
Kommunikationskomponente ersetzt. Diese DLL (\texttt{s7otbxdx.dll})
bietet verschiedene Funktionen, um mit die PLC via PROFINET
anzusprechen, von dessen Programmspeicher Codeblöcke zu lesen und
wieder zu beschreiben.

Im Originalzustand wird der von dem PLC ausgelesene Code Eins-zu-Eins
im Step7-Editor angezeigt. Stuxnet manipuliert die Anzeige der Blöcke
nun so, dass eigene, auf die PLC gebrachte Schadroutinen beim Auslesen
ausgeblendet, bzw.  beim Schreiben auf die PLC (wieder) hinzugefügt
werden (vgl.  Abbildung~\ref{fig:comp:stuxnet:s7otbxdx}
vs.~\ref{fig:comp:stuxnet:s7otbxsx}).

\begin{figure}[p]
  \centering
  \includegraphics[width=.7\linewidth]{./images/comp-stuxnet-s7otbxdx.png}
  \caption{Normale Kommunikation zwischen Step7 und den PLCs}
  \label{fig:comp:stuxnet:s7otbxdx}
\end{figure}

\begin{figure}[p]
  \centering
  \includegraphics[width=.7\linewidth]{./images/comp-stuxnet-s7otbxsx.png}
  \caption{Manipulierte Kommunikation über eine von Stuxnet ersetzte DLL}
  \label{fig:comp:stuxnet:s7otbxsx}
\end{figure}



Bei den von Stuxnet geschriebenen Codeblöcken muss man im Wesentlichen
zwischen drei verschiedenen Sequenzen für zwei unterschiedliche
PLC-Typen sprechen. Es bliebt zunächst auch unklar, was diese
Sequenzen bei Ausführung \emph{genau} tun, da aus ihnen heraus
einerseits andere (unbekannte) Codeblöcke aufgerufen werden, und
andererseits Code-Obfuscation betrieben wurde -- die Funktionsnamen
lassen also keine Rückschlüsse auf die wirkliche Funktion zu.

\subsection{Fazit und Ausblick}

Die Anzahl der zum Teil stark verschiedenen Exploits zeigt sehr
deutlich, dass der Angreifer sicher sein wollte, sein Ziel zu
erreichen. Welches Ziel dies aber ist, oder wer der Angreifer sei, ist
bis heute nicht geklärt. Im Code finden sich einige Indizien, die
darauf hindeuten können, wer dafür Verantwortlich ist, bzw. wer oder
was das Ziel ist:

\begin{itemize}
  \item Eine Pfadangabe enthält die Zeichenfolge \emph{myrtus}. Diese
kann sowohl \enquote{My RTUs}\footnote{RTU = Remote Terminal Unit, ein
Synonym für PLCs} gelesen werden, als auch nach hebräischer
Übersetzung auf das Buch Ester verweisen, welches dem jüdischen Volk
die Verteilung und Tötung ihrer Feinde erlaubt.
  \item Liest Stuxnet aus einem bestimmten Windows-Registy-Key den
Wert \emph{19790509}, bricht es seine Installations- und
Infektionsroutine ab.  Neben vielen weiteren Ereignissen wurde mit
Habib Elghanian am 09.05.1979 einer der ersten Juden von der damals
neu gegründeten Islamischen Republik ermordet.
  \item Einer der Rücksprungadressen im PLC-Code hat die Adresse
\texttt{DEADF007}. Transliteriert man diese Zeichenfolge aus dem
\textit{Leetspeck} zu \enquote{dead foot}, so bezeichnet dies den
Ausfall einer (Flugzeug-) Turbine.
\end{itemize}

In jedem Fall sei hier angemerkt, dass diese gelegten Fährten nichts
anderes als Ablenkung sein können, um die wahren Täter zu schützen,
bzw. der Verdacht auf andere abzuwenden.

Zwischenzeitlich wurden auch neue Erkenntnisse zu Stuxnet gewonnen:

\begin{itemize}
  \item Symantec ließ verlauten\footnote{Symantec: \enquote{Stuxnet: A
Breakthrough},
\url{http://www.symantec.com/connect/blogs/stuxnet-breakthrough},
12.11.2010, zuletzt abgerufen am 14.11.2010}, dass sie verstünden,
welche Funktion die von Stuxnet auf die PLCs geschriebenen Codeblöcke
erfüllen. So verändert Stuxnet die Ausgangsfrequenz bestimmter
angeschlossener Frequenzumrichter, um damit schließlich die
Umdrehungsgeschwindigkeiten von Motoren so zu manipulieren, dass der
daran angeschlossene Fertigungsprozess sabotiert wird.
  
  \item Einen Tag darauf bestätigte\footnote{Langner: \enquote{Yeah,
it's a drive array for the 315 attack code},
\url{http://www.langner.com/english/?p=415}, 13.11.2010, zuletzt
abgerufen am 16.11.2010} die Langner Communications GmbH die Analysen
von Symantec. Ralph Langner veröffentlichte bereits im September die
hochspekulative Annahme\footnote{Langner: \enquote{Stuxnet logbook,
Sep 16 2010, 1200 hours MESZ},
\url{http://www.langner.com/english/?p=217} zuletzt abgerufen am
16.11.2010}, dass das Ziel das iranische Atomprogramm sei. Die
Analysen von Langner und Symantec lassen dazu nun kaum Alternativen.
\end{itemize}

Die spannendste Frage, wer sich für die Entwicklung von Stuxnet
verantwortlich zeichnet, wird aber wohl nie beantwortet werden.

\subsection{Materialien}

\begin{itemize}
  \item Aleksander Matrosov, Eugene Rodionov, David Harley, Jurai
Malcho: \enquote{Stuxnet under the Microscope}. ESET, 23.09.2010.
\url{http://www.eset.com/resources/white-papers/Stuxnet_Under_the_Microscope.pdf}

  \item Nicolas Falliere, Liam O Murchu, Eric Chien:
\enquote{W32.Stuxnet Dossier, Version 1.1}. Symantec, 12.10.2010.
\url{http://www.symantec.com/content/en/us/enterprise/media/security_response/whitepapers/w32_stuxnet_dossier.pdf}

  \item Bruce Schneier: \enquote{The Story Behind The Stuxnet
Virus}. Forbes-Kommentar, 07.10.2010.
\url{http://www.schneier.com/blog/archives/2010/10/stuxnet.html}

  \item Frank Rieger: \enquote{Der digitale Erstschlag ist
erfolgt}. FAZ, 22.09.2010.  \url{http://www.faz.net/-01i43d}

  \item Frank Rieger, Felix von Leitner: \enquote{Alternativlos, Folge
5}. Podcast, 26.09.2010.  \url{http://alternativlos.org/5/}

  \item Wikipedia: \enquote{Stuxnet}.
\url{https://secure.wikimedia.org/wikipedia/en/wiki/Stuxnet}

  \item Ralph Langner: \enquote{Stuxnet logbook}. Blog.
\url{http://www.langner.com/en/}
\end{itemize}
