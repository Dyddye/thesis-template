\section{Würmer}
\label{compositions:wuermer}

\subsection{Einführung}

\subsubsection {Definition}

Ein Computerwurm bezeichnet ein Programm, dass vorhandene
Schwachstellen im Betriebssystem oder in Anwendungen, aber auch oft
die Naivität und Unwissenheit der Nutzer ausnutzt, um sich
unberechtigt auf dem Zielsystem Zugang zu Ressourcen zu
verschaffen. Nach einer erfolgreichen Infiltration durchsucht der Wurm
auf dem Zielsystem bestimmte Dateien nach neuen Hosts, IP- oder
Emailadressen, also neuen Opfern, um diese anschließend mit bereits
bestehenden bzw. vom Wurm \textit{mitgebrachten} Netzwerkdienste
anzugreifen.

Im RFC 4949 wird ein Wurm wie folgt definiert:

\begin{quote} \itshape \foreignquote{english}{ computer program that
can run independently, can propagate a complete working version of
itself onto other hosts on a network, and may consume system resources
destructively.}  ~\cite{wuermer-rfc4949}
\end{quote}

Würmer werden der Kategorie der Maleware zugeordnet, da Würmer
unerwünschte Aktionen auf dem Zielsystem ausführen, verdeckt
operieren, Daten stehlen, Ressourcen verbrauchen und oftmals auch
absichtlich Schaden verursachen. Ein Wurm ist einem Virus in vielen
Belangen (Ziele, Verbreitungswege, Eigenschaften) sehr ähnlich, wobei
es auch essentielle Unterschiede gibt: Viren sind lediglich
Code-Fragmente, die sich an andere Daten anhängen und sich nur bei
deren Ausführung oder Verarbeitung vermehren
~\cite{wuermer-viren}. Würmer sind hingegen komplexer und verbreiten
sich selbst.

\subsubsection {Analogie im Tierreich}

Zwischen den Pedanten im Tierreich gibt es eine ähnliche
Relation. Viren sind kleine Krankheitserreger, bestehend aus
Erbmaterial, das von einer schützenden Hülle aus Fetten oder Eiweißen
umgeben ist. Sie besitzen keinen eigenen Stoffwechsel und können sich
nur innerhalb lebender Wirtszellen (Analogie: Wirtsdatei)
vermehren. ~\cite{wuermer-bioviren}.  Im Gegensatz dazu sind
parasitäre Würmer höherentwickelte und selbständige Lebewesen
(\textit{wobei es in der Wissenschaft unterschiedliche Definition von
Leben gibt} ~\cite{wuermer-leben}), die sich in einem Wirt einnisten,
sich von diesem ernähren und Eier legen. Die Eier werden über die
Fäkalien ausgeschieden und verbreiten sich dadurch.

Eine direkte Übereinstimmung liegt natürlich nicht vor, da biologische
Organismen ungleich komplexer sind. Doch gewisse Analogien lassen sich
ableiten: Z.B. sind aggressive Viren (z.B. der Ebolavirus) nicht sehr
verbreitet, da die Inkubationszeit gering ist und die Sterberate zu
hoch ist, wodurch potenzielle Überträger schnell dezimiert werden
~\cite{wuermer-ebola}. Die erfolgreichsten Viren, wie z.B. der
Herpesvirus 1, von dem 90\% der Menschen infiziert sind, sind meisten
harmlos oder können sogar eine symbiotische Verbindung mit dem Wirt
eingehen ~\cite{wuermer-herpes}.  Diese Beobachtungen aus dem
Tierreich können auch als Inspiration für künftige Wurmprogrammierer
dienen.

\subsubsection {Verbreitungswege}

Es gibt eine Vielzahl von Verbreitungswegen für Würmern. Grundsätzlich
können Würmer jede Art von Anwendung dafür verwenden neue Opfern zu
finden. Eine beliebte z.B. ist der E-Mail Dienst. Der Wurm wird
entweder im Anhang mitgeschickt oder ein Hyperlink in der Mail
verweist auf eine attackierende Seite, die den Angriff bzw. die
Infektion startet. Nach der Infektion wird die Wurm-Mail an alle auf
dem Zielsystem gefunden Adressaten mit Hilfe eines installierten
E-Mail Programms bzw. eines vom Wurm mitgelieferten verschickt. Analog
dazu können auch Instant-Messaging Dienste verwendet werden, die
ebenfalls entweder einen Link an alle lokal gespeicherten Adressaten
(\textit{in der Fachliteratur auch Buddies genannt}) auf eine
attackierende Seite verschicken oder den Wurm direkt allen anderen zum
Download anbieten. Weitere beliebte Anwendungen für Würmer sind IRC-
und P2P-Clients. Eine unübliche Verbreitunsgart sind
Wechseldatenträger, da bei diesem Medium die Reproduktionsrate äußert
gering ist ~\cite{wuermer-ways}.

\subsubsection {Angriffstaktiken}

Würmer können theoretisch alle vorhanden Angriffstaktiken verwenden,
um sich einen unberechtigten Zugang zu einem System zu verschaffen
(wenn man jedoch für die Verbreitung Wechselmedien verwendet, werden
menschliche Helfer benötigt). Z.B. werden sehr häufig Pufferüberläufe
erzeugt, um in das System zu gelangen. Ein weiterer Schwachpunkt ist
das Opfer selber: die Naivität, Ahnungslosigkeit, aber auch oft
Faulheit bzw. Gleichgültigkeit machen es dem Wurm leichter in das
System zu gelangen.  Z.B. konnte sich der \textit{I-love-you-Wurm} so
schnell verbreiten, weil die Opfer den Anhang öffneten.  Ein andere
Schwachpunkt sind unsichere Passwörter: Manche Würmer probieren
Wörterbuch-Passwörter aus oder kombinieren auf dem System gefundene
Nach- und Vornamen zu entsprechenden Hosts, um sich zu z.B. per SSH zu
authentifizieren.

\subsection {Morris-Wurm}
\subsubsection {Einführung} Um zu verstehen wie Würmer aufgebaut sind,
welche Motivation dahinter steckt und welche Konsequenzen die
Aussetzung eines Wurmes haben können wird im folgenden näher auf den
Morris-Wurm eingegangen.  Der erste Internet- bzw. Arpawurm mit dem
Namen Morris wurde vom gleichnamigen Autor Robert Tappan Morris
programmiert und 1988 ausgesetzt. Der Wurm infizierte ungefähr 6000
Systeme, also 10\% des gesamten Arpanets und verursachte einen Schaden
von bis zu 100 Millionen US- Dollar. Als Reaktion auf den Wurm wurde
das \textit{Computer Emergency Response Team Coordiantion Center}
(kurz CERT) gegründet ~\cite{wuermer-overview}.

\subsubsection {Angriffstaktiken}

Der Wurm konnte vor allem VAX and Sun-Maschinen infizieren, die mit
einem BSD-Unix betrieben wurden.  Die folgenden Schwächen
~\cite{wuermer-morris} bzw. Angriffsstrategien wurden vom Wurm
ausgenutzt:

\paragraph{Pufferüberlauf vom Programm fingerd}

Der Daemon \texttt{fingerd} verwendete die Funktion \texttt{gets}, um
eine Usereingabe entgegenzunehmen, ohne die Länge zu
überprüfen. Dadurch konnte ein Pufferüberlauf verursacht werden, um
die Rücksprungadresse auf den überschriebenen Puffer zeigen zu
lassen. Der eingeschleuste Code führte ein \texttt{chmk}
(\textit{change-mode-to-kernel}) aus, wodurch eine neue Shell-Session
erstellt werden konnte. Die Shell \textit{kontaktierte} ein bereits
infiziertes System und holt sich die benötigten Dateien zum
installieren des Wurmes. Bei den Sun-Rechnern führte diese Funktion
aufgrund eines Programmierfehlers zu einem Core Dump.  Obwohl die
Existenz dieses Sicherheitsloch bekannt war, wurden die verfügbaren
Fixes nicht von allen Administratoren eingespielt.

\paragraph{Bug/Backdoor im Programm sendmail}

Ein Bug bzw. eine Backdoor von sendmail ermöglichte es Remotebefehle
abzusetzen; und das ohne jegliche Authentifizierung. Im Debug-Modus
interpretierte sendmail die Empfängeradresse als Befehl, wodurch man
sich mit dem \textit{Empfänger} \texttt{/bin/sh} eine Shell mit den
jeweiligen Rechten öffnen lassen konnte. Obwohl auch dieser Fehler
schon seit langem bekannt war, gab es dennoch einige Systeme, die den
debug-Modus nicht deaktiviert haben \footnote{neuere Versionen von
sendmail haben diesen Modus standardmäßig deaktiviert}.

\paragraph{Passwortabfrage}

Nachdem eine erfolgreiche Infizierung stattfand, also alle
Objektdateien nachgeladen waren, durchsuchte der Wurm bestimmte
Konfigurationsdateien nach neuen Hosts und deren verschlüsselte
Passwörter. Zunächst wurden einfache Passwörter\footnote{wie z.B. der
Username, der Username rückwärts etc.} verschlüsselt und mit dem
gefunden verschlüsselten Passwort verglichen. Diese Methode wurde
angewandt, um so leise wie möglich zu sein und keine fehlgeschlagenen
Authentifizierungsversuche zu verursachen. Wenn das nicht
funktionierte wurde das Gleiche mit einem installiertem (falls
vorhanden) und mit einem vom Wurm mitgebrachten Wörterbuch versucht
\footnote{siehe cracksome.c in
ftp://coast.cs.purdue.edu/pub/doc/morris\_worm}.

\subsubsection {Sonstiges}

Der Wurm war nicht fertig und die Aussetzung erfolgte vermutlich
aufgrund einer Torschlusspanik des Autors, da kurz vor der Aussetzung
ein Sicherheitsloch bekannt wurde, das er ursprünglich ausnutzen
wollte. Sein ursprüngliches Ziel, die Größe des Arpanetes zu bestimmen
konnte gar nicht erreicht werden, da aufgrund eines Programmierfehlers
nichts \textit{Zuhause} ankam. Um zu verhindern, dass sich mehrere
Wurm-Instanzen installieren, wurde versucht eine Verbindung durch
einen Socket zu dem vermeintlich installierten Wurm zu etablieren und
falls dies gelang wurde eine bestimmte Variable auf 1 gesetzt, die
dazu führte, dass sich der Wurm deinstallierte. Nichtsdestotrotz war
jeder 7. Wurm \textit{unsterblich}, um gegen einen Fake-Wurm gewappnet
zu sein. Aus diesem Grund verursachte der Wurm eine DOS-Attacke, da
Robert Morris die Verbreitungsgeschwindgkeit unterschätzt hat.

\subsection {Erkenntnisse für \f}

Zunächst einmal können wir sehr viele technische Merkmale des
Morris-Wurms für unseren Wurm \footnote {wenn es denn ein Wurm werden
sollte} umsetzen, indem wir die entsprechenden Meterpreter-Scripts
suchen bzw. selber welche implementieren. (z.B. forked sich der Wurm
in regelmäßigen Abständen und löscht den Parent-Prozess, um nicht
entdeckt zu werden). Außerdem können wir viel von der Infrastruktur
lernen und den grundsätzlichen Aufbau eines etwas komplexeren Wurms
möglicherweise nachbauen, da der Quellcode zur Verfügung steht.

Wir können aus den begangenen Fehler lernen: unser Wurm bzw. unser
wurmartiges Brückenkopfprogramm \footnote {eine genaue Spezifikation
fand noch nicht statt} sollte sich keinesfalls mehrmals installieren,
aber gleichzeitig nicht einfach gestoppt werden können, falls ein
Fakeprogramm immer mit \texttt{ ich bin schon installiert} antwortet.







