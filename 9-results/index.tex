\chapter{Fazit}
\label{chp:results}
\section{Fazit}
\authors{\DK \and \LM \and \MW}{\DH \and \JF \and \HM \and \DE}

Der erreichte Funktionsumfang des \f-Systems und die erfüllten Ziele
wurden bereits im Kapitel~\ref{chp:conclusion} auf
Seite~\pageref{chp:conclusion} vorgestellt. Im folgenden Kapitel wird nun
auf die verbleibenden Ziele des Kapitel~\ref{chp:objectives}
(Seite~\pageref{chp:objectives}) eingegangen und ein allgemeines Fazit über das
Projekt und sein Ergebnis gefasst.
 
\subsection{FIDIUS-System}
Ein wesentliches Ziel des \f-Systems war der Vergleich und die
Evaluierung von verschiedenen \acr{ids}. Wesentliche Idee war hierbei, eine
Reihe von definierten Angriffen und gesendeten Datenströmen auszuführen und
das Verhalten des \acr{ids} während dieser Zeit zu protokollieren. Zum
einen sollten die protokollierten Daten als Bericht zur Verfügung gestellt
werden, zum anderen sollte der Bericht Informationen wie \zB erzeugte False-Positives
oder richtig erkannte Angriffe beinhalten. Mit diesen Daten hätten außerdem
verschiedene \acrpl{ids} miteinander verglichen werden können.

Im jetzigen Zustand kann das \f-System zwar das Verhalten von \acr{ids}
protokollieren, wenn eine Schnittstelle zwischen \acr{msf} und dem
\acr{ids} bereitgestellt wird, aber es erfolgt keine automatische
Auswertung der erhobenen Daten, um einen Bericht zu erstellen. Eine
Auswertung der protokollierten Daten findet zwar beim Snortor statt,
allerdings wird hier nur das Ergebnis hinsichtlich eines \acr{ids}
optimiert, es werden nicht verschiedene \acrpl{ids} miteinander
verglichen.

Die Host-Reconnaissance ist ein weiteres Ziel, welches nicht komplett gelöst wurde.
Die benötigte Funktionalität wurde zwar unter Verwendung von \acr{metap}
umgesetzt, allerdings wurde diese Funktionalität nach der Umstellung auf
die \f-Core Architektur, siehe Abschnitt~\ref{sec:core} auf
Seite~\pageref{sec:core}, nicht mehr in ins \f-System integriert. Dieses
Schicksal teilen auch die andere, im Rahmen von \f entwickelten \acr{metap}-Skripte.

Wie bereits im Kapitel~\ref{chp:conclusion} auf
Seite~\pageref{chp:conclusion} beschrieben wurde, wurde die
\f-Core Architektur eingeführt, um den modularen Aufbau des
Systems zu gewährleisten. Während der modulare Aufbau zwischen
der \acr{gui} und dem \f-Core gut funktioniert hat, ist dies zwischen
dem \f-Core und \acr{msf} nicht gut gelungen, da die beiden Komponenten
durch \acr{drb} eng miteinander gekoppelt sind. Das widerspricht der
ursprünglichen Planung, dass das \f-System mit
beliebigen Angriffsframeworks kompatibel ist und somit \acr{msf}
austauschbar sein sollte. Genauere Informationen zur Kommunikation mit dem
\f-Core können im Abschnitt~\ref{subsec:core:communication} auf
Seite~\pageref{subsec:core:communication} gefunden werden.

Allerdings gibt es keine gut getrennte
Schnittstelle zwischen \f-Core und \acr{msf}. Durch die Verwendung von
\acr{drb} zwischen \f-Core und \acr{msf} ist es möglich, direkt auf
\acr{msf}-Methoden zuzugreifen. Damit ging die Anbindung über \acr{drb}
schnell vonstatten, hat aber auch dazu geführt, dass \acr{msf} mit dem
\f-Core, genauer gesagt mit der Action-Komponenten, stark gekoppelt ist
und nicht einfach ausgetauscht werden kann.  

Neben der nicht vorhandenen Schnittstelle hat die Verwendung von \acr{drb}
zu einigen Fehlern im System geführt. Die Fehler entstanden
hauptsächlich beim Weiterreichen von Referenzen auf \acr{msf}-Objekten und
der anschließenden Verwendung dieser Objekte im \f-Core. Die Fehlerfindung
musste über \acr{drb} hinweg sowohl im \f-Core als auch in \acr{msf}
durchgeführt werden und wurde entsprechend stark erschwert, dadurch wurde
auch einen Großteil des erhaltenen Zeitgewinns negiert.

Letzendlich hat sich gezeigt, dass die Verwendung
von \acr{drb} als Kommunikationsschnittstelle zwischen \f-Core und
\acr{msf} eine gute Wahl war, da alle anderen Lösungen noch
aufwändiger gewesen wären und selbst mit \acr{drb} die Umstellung aufgrund
von Zeitmangel nicht komplett durchgeführt werden konnte. Die
starke Kopplung von \acr{msf} kann auch verkraftet werden, da uns kein
vergleichbares Framework bekannt ist, für das man \acr{msf}
austauschen wollte.

Grundsätzlich lief die Projektarbeit sehr flüssig und bis auf einige
größere Umstellung in der Architektur, sind nur sehr wenige Stellen
die bei einem erneuten Beginn anders gemacht werden
sollten. Dabei ist der wichtigste Punkt, das Definieren eines genauen Zieles 
und nicht nur einer ungefähren Richtung. Zusätzlich war die Aufteilung im 
ersten Projektjahr (KI und RN) nicht sehr förderlich und hätte im Nachhinein 
anders gelöst werden sollen.

Zusammenfassend kann man sagen, dass
das Projekt, obwohl es nach einem Jahr eine Zielveränderung gab, sehr
erfolgreich war.